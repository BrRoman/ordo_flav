\documentclass[twoside]{article}

% Géométrie de la page :
\usepackage{geometry}
\geometry{paperwidth=10.5cm, paperheight=16.5cm, inner=0.75cm, outer=0.75cm, tmargin=0.5cm, bmargin=0.5cm, includehead}

% Choix d'une police principale :
\usepackage{luatextra}
\setmainfont{Garamond Premier Pro}% Linux serveur : mettre 'Amiri'.
% Polices de page de garde :
\newfontfamily{\FontFlyLeaf}{Silentium Pro Roman I}

% Césures etc. :
\usepackage[latin]{babel}
\usepackage{array}
\usepackage{enumitem}

% Mise en forme des titres de sections (? Janvier, février etc. ?) :
\usepackage[explicit]{titlesec}
\titleformat{\section}{}{}{0cm}{\center\MakeUppercase{#1}}

% Entêtes et pieds de pages :
\usepackage{fancyhdr}
\pagestyle{fancy}
\fancyhf{}
\fancyhead[RO,LE]{\thepage}
\fancyhead[CE]{\textbf{\rightmark}}
% Le [CO] étant variable sera défini dans le fichier maître par programmation.
\renewcommand{\sectionmark}[1]{\markright{#1 \CurrentYear{}}}
\renewcommand{\headrulewidth}{0pt}
\renewcommand{\footrulewidth}{0pt}
\setlength{\parindent}{0cm}
\setlength{\headsep}{0.3cm} % Distance entre le header ("Cyclus liturgicus etc.") et le corps du texte.

% Tableaux spéciaux sur plusieurs pages :
\usepackage{longtable}

% Pour forcer l'usage des césures (cf. mail Thierry Masson), ATTENTION DANGER (entêtes et autres blocs bizarres) :
\pretolerance = -1 
\tolerance = 2000 

% Pour augmenter l'approche des caractères :
\usepackage{soul}



%%%%%%%%%%%%%%%%%%%%%%%%%%%%%%%
% Commandes et environnements :
%%%%%%%%%%%%%%%%%%%%%%%%%%%%%%%

%% Espace fine :
\DeclareRobustCommand{\mynobreakthinspace}{%
\leavevmode\nobreak\hspace{0.08em}}
\def~{\mynobreakthinspace{}}

% Environnement Boîte (utile chaque fois qu'on veut commencer par un espace vertical avant le paragraphe) :
\newenvironment{ParBox}[2]
{
\setlength{\parindent}{0cm}
\begin{center}
\parbox{8.5cm}{\vspace{#1} #2}}
{
\end{center}
\par}
\newcommand{\ApplyParBox}[2]{\begin{ParBox}{#1}{#2}\end{ParBox}}

% Style de paragraphe NewMonthTitres :
\newenvironment{ParStyleNewMonthTitles}[1]
{
\setlength{\parskip}{0cm}
\fontsize{14}{13}\selectfont
\bfseries
\section{#1}}{}
\newcommand{\ApplyNewMonthTitles}[1]{\begin{ParStyleNewMonthTitles}{#1}\end{ParStyleNewMonthTitles}}

% Style de paragraphe NewMonthSousTitres :
\newenvironment{ParStyleNewMonthSubTitles}[1]
{
\fontsize{12}{13}\selectfont
\bfseries
\itshape
\begin{center}
\MakeUppercase{#1}}
{
\end{center}}
\newcommand{\ApplyNewMonthSubTitles}[1]{\begin{ParStyleNewMonthSubTitles}{#1}\end{ParStyleNewMonthSubTitles}}

% Commande nouveau mois pour entêtes :
\newcommand{\NewMonthForHeader}[1]{\renewcommand{\leftmark}{\normalfont{#1}}}

% Style de paragraphe GeneralitiesTitres1 :
\newenvironment{ParStyleGenerTitleHuge}[1]
{
\fontsize{15}{19}\selectfont
\setlength{\parskip}{-0.5cm}
\begin{center}
\MakeUppercase{\textbf{#1}}}
{
\end{center}}
\newcommand{\ApplyGenerTitleHuge}[1]{\begin{ParStyleGenerTitleHuge}{#1}\end{ParStyleGenerTitleHuge}}

% Style de paragraphe GeneralitiesTitres2 :
\newenvironment{ParStyleGenerTitleLarge}[1]
{
\fontsize{12}{14.5}\selectfont
\setlength{\parskip}{-0.5cm}
\begin{center}
\MakeUppercase{\textbf{#1}}}
{
\end{center}}
\newcommand{\ApplyGenerTitleLarge}[1]{\begin{ParStyleGenerTitleLarge}{#1}\end{ParStyleGenerTitleLarge}}

% Style de paragraphe GeneralitiesSousTitres :
\newenvironment{ParStyleGenerSubTitle}[1]
{
\fontsize{11.5}{14}\selectfont
\begin{center}
\textbf{#1}}
{
\end{center}
\par}
\newcommand{\ApplyGenerSubTitle}[1]{\begin{ParStyleGenerSubTitle}{#1}\end{ParStyleGenerSubTitle}}

% Style de paragraphe GeneralitiesListes :
\newenvironment{ParStyleGenerList}
{
\fontsize{9.5}{10.5}\selectfont
\setlength{\parskip}{0.5cm}
\begin{itemize}
[label = -,
labelwidth = 0.3cm,
partopsep = -0.5cm,
itemsep = 0cm,
leftmargin = 0.5cm]}{
\end{itemize}}
\newcommand{\ApplyGenerList}[1]{\begin{ParStyleGenerList}#1\end{ParStyleGenerList}}

% Style de paragraphe Anniv :
\newenvironment{ParStyleAnniv}
{
\setlength{\parindent}{0cm}
\setlength{\parskip}{0.5cm}
\fontsize{9.5}{10.5}\selectfont
\itshape}
{
\par}
\newcommand{\ApplyAnniv}[1]{\begin{ParStyleAnniv}#1\end{ParStyleAnniv}}

% Style de paragraphe HebdoPsalt :
\newenvironment{ParStyleHebdoPsalt}
{
\fontsize{11.5}{13}\selectfont
\setlength{\parindent}{0cm}
\setlength{\parskip}{0cm}
\itshape
\begin{center}}
{
\end{center}
\par}
\newcommand{\ApplyHebdoPsalt}[1]{\begin{ParStyleHebdoPsalt}#1\end{ParStyleHebdoPsalt}}

% Style de paragraphe Header :
\newenvironment{ParStyleHeader}
{
\fontsize{11}{12}\selectfont
\setlength{\parindent}{0cm}
\setlength{\parskip}{0.3cm}
\setlength{\tabcolsep}{0cm}
\begin{tabular}{p{0.7cm} p{8.3cm}}}
{
\end{tabular}
\par}
\newcommand{\ApplyHeader}[1]{\begin{ParStyleHeader}#1\end{ParStyleHeader}}

% Style de paragraphe Body :
\newenvironment{ParStyleBody}
{
\fontsize{10}{11.5}\selectfont
\begin{itemize}
[label = -,
labelwidth = 0.3cm,
partopsep = 0cm,
itemsep = 0cm,
leftmargin = 0.7cm]}
{
\end{itemize}}
\newcommand{\ApplyBody}[1]{\begin{ParStyleBody}#1\end{ParStyleBody}}

% Style de paragraphe LecturesHeader :
\newenvironment{ParStyleLectHeader}
{
\fontsize{10}{11}\selectfont
\setlength{\parindent}{0cm}
\setlength{\parskip}{0cm}
\bfseries
\begin{center}}
{
\end{center}
\par}
\newcommand{\ApplyLectHeader}[1]{\begin{ParStyleLectHeader}#1\end{ParStyleLectHeader}}

% Style de paragraphe LecturesBody :
\newenvironment{ParStyleLectBody}
{
\fontsize{9}{10.5}\selectfont
\begin{list}{}{
\setlength{\labelwidth}{1.2cm}% largeur de la boîte englobant le label
\setlength{\labelsep}{0.1cm}% espace entre paragraphe et l’étiquette
\setlength{\leftmargin}{1.8cm}% marge de gauche - Normalement \labelwidth+\labelsep, mais ça marche pas. - En pratique mettre \labelwidth + \labelsep + la \leftmargin du Style Body. Ainsi les lectures sont alignées sur les items des paragraphes Body.
\setlength{\itemsep}{-0.05cm}% espace entre les items.
\renewcommand{\makelabel}[1]{\raggedleft{##1}\hfill}}}
{
\end{list}}
\newcommand{\ApplyLectBody}[1]{\begin{ParStyleLectBody}#1\end{ParStyleLectBody}}

% Style de paragraphe LecturesTriduum :
\newenvironment{ParStyleLectTrid}
{
\fontsize{9}{10.5}\selectfont
\setlength{\parskip}{-0.2cm}}
{}
\newcommand{\ApplyLectTriduum}[1]{\begin{ParStyleLectTrid}#1\end{ParStyleLectTrid}}

% Style de paragraphe Préface féries :
\newenvironment{ParStylePrefaceFeries}
{
\fontsize{9}{10.5}\selectfont
\begin{itemize}
[label = -,
labelwidth = 0.3cm,
partopsep = 0cm,
itemsep = 0cm,
leftmargin = 0.5cm]}
{
\end{itemize}}
\newcommand{\ApplyPrefaceFeries}[1]{\begin{ParStylePrefaceFeries}#1\end{ParStylePrefaceFeries}}


% Symboles spéciaux :
% Abstinence :
\catcode`\ł=\active
\defł{{\fontspec{Menlo} \symbol{10063}}}
% Jeûne et abstinence :
\catcode`\µ=\active
\defµ{{\fontspec{Menlo} \symbol{9724}}}
% 1er vendredi du mois :
\catcode`\£=\active
\def£{{\fontspec{Menlo} \symbol{10070}}}
% 1er samedi du mois :
\catcode`\§=\active
\def§{{\fontspec{Menlo} \symbol{10037}}}
% 1er dimanche du mois :
\catcode`\ŧ=\active
\defŧ{{\fontspec{Menlo} \symbol{10016}}}
% Fête précepte :
\catcode`\¬=\active
\def¬{{\fontspec{Menlo} \symbol{9670}}}
% Fête demi-précepte :
\catcode`\þ=\active
\defþ{{\fontspec{Menlo} \symbol{10061}}}
% Antienne :
\catcode`\ø=\active
\defø{{\fontspec{FlavGaramond} \symbol{8721}}}
% Répons :
\catcode`\¶=\active
\def¶{{\fontspec{FlavGaramond} \symbol{164}}}
% Verset :
\catcode`\ß=\active
\defß{{\fontspec{FlavGaramond} \symbol{8730}}}
% Croix :
\catcode`\†=\active
\def†{{\fontspec{FlavGaramond} \symbol{8224}}}


\fancyhead[CO]{\textbf{Cyclus liturgicus C/I}}
\newcommand{\CurrentYear}{2018}

\begin{document}

\thispagestyle{empty}

\begin{center}
\par\parbox{7.5cm}{\vspace{1cm}}\par
\makebox[7.5cm][s]{\fontsize{68}{0}\selectfont\FontFlyLeaf{\MakeUppercase{O R D O}}\par}
\vspace{1cm}\par
\makebox[7.5cm][s]{\fontsize{24}{0}\selectfont\FontFlyLeaf{\MakeUppercase{missæ celebrandæ}}\par}
\vspace{0.4cm}\par
\makebox[7.5cm][s]{\fontsize{24}{0}\selectfont\FontFlyLeaf{\MakeUppercase{et divini officii}}\par}
\vspace{0.4cm}\par
\makebox[7.5cm][s]{\fontsize{24}{0}\selectfont\FontFlyLeaf{\MakeUppercase{p e r s o l v e n d i}}\par}
\vspace{1cm}\par
\makebox[7.5cm][s]{\fontsize{17}{0}\selectfont\FontFlyLeaf{\MakeUppercase{iuxta ritum romano-}}\par}
\vspace{0.4cm}\par
\makebox[7.5cm][s]{\fontsize{17}{0}\selectfont\FontFlyLeaf{\MakeUppercase{monasticum in abbatia}}\par}
\vspace{0.4cm}\par
\makebox[7.5cm][s]{\fontsize{17}{0}\selectfont\FontFlyLeaf{\MakeUppercase{sancti ioseph claræ vallis}}\par}
\vspace{1cm}\par
\makebox[6cm][s]{\fontsize{28}{0}\selectfont\FontFlyLeaf{\MakeUppercase{\so{pro anno}}}\par}
\vspace{0.4cm}\par
\makebox[6cm][s]{\fontsize{28}{0}\selectfont\FontFlyLeaf{\MakeUppercase{l i t u r g i c o}}\par}
\vspace{0.4cm}\par
\makebox[6cm][s]{\fontsize{28}{0}\selectfont\FontFlyLeaf{\MakeUppercase{d o m i n i}}\par}
\vspace{0.4cm}\par
\makebox[6cm][c]{\fontsize{28}{0}\selectfont\FontFlyLeaf{\MakeUppercase{2018-2019}}}

\end{center}
\newpage
\thispagestyle{empty}\null

\newpage
\thispagestyle{empty}
\ApplyParBox{0.5cm}{\ApplyGenerTitleHuge{Anno liturgico 2018-2019}
\smallskip
\ApplyGenerTitleLarge{Celebrationes mobiles}}
\medskip
\fontsize{12}{12}\selectfont
\setlength{\parskip}{0.1cm}
Dominica I Adventus \dotfill d. 2 decembris\par
Sanctæ Familiæ \dotfill d. 30 decembris\par
Baptisma Domini \dotfill d. 13 ianuarii\par
Feria IV Cinerum \dotfill d. 6 martii\par
Dominica in Palmis \dotfill d. 14 aprilis\par
Dominica Paschæ \dotfill d. 21 aprilis\par
Ascensio Domini \dotfill d. 30 maii\par
Dominica Pentecostes \dotfill d. 9 iunii\par
Ss.mæ Trinitatis \dotfill d. 16 iunii\par
Ss.mi Corporis et Sanguinis Domini \dotfill d. 20 iunii\par
Ss.mi Cordis Iesu \dotfill d. 28 iunii\par
D.N.I.C. universorum Regis \dotfill d. 24 novembris\par
\vspace{1cm}
\ApplyGenerTitleLarge{Cyclus liturgicus}
\ApplyGenerSubTitle{\textnormal{Dominicalis et festivus~: }C}
\ApplyGenerSubTitle{\textnormal{Ferialis~: }I}
\newpage
\thispagestyle{empty}
\ApplyParBox{0cm}{\ApplyGenerTitleLarge{Clavis signorum}
 \ApplyGenerTitleLarge{et abbreviationum}}
\setlength{\tabcolsep}{0cm}
\begin{tabular}{p{1cm} p{7.5cm}}
¬ & \small{Festum de præcepto.}\\
þ & \small{Festum cum tempore usque ad Nonam de præcepto, postmeridiem vero laborandum est.}\\
µ & \small{Ieiunium regulare cum abstinentia carnium.}\\
ł & \small{Abstinentia carnium servanda secundum normas Declarationum nostrarum.}\\
ŧ & \small{Prima dominica in mense~: ad Completorium, dicitur antiphona Beatæ Mariæ Virginis in honorem Dominæ Nostræ Ephesini ad intentiones diocesis Smyrnensis (Izmir).}\\
£ & \small{Prima feria VI in mense~: fit expositio Sanctissimi Sacramenti, in honorem Sacratissimi Cordis Iesu, a Vigiliis usque ad Completorium.}\\
§ & \small{Primum sabbatum in mense.}\\
 & \\
\small{AM} & \small{Antiphonale monasticum.}\\
\small{CM} & \small{Collectio Missarum de Beata Maria Virgine 1987 (numerus designat ordinem missarum).}\\
\small{GR} & \small{Graduale romanum.}\\
\small{LS} & \small{Lectionnaire de semaine.}\\
\small{MC} & \small{Missa conventualis.}\\
\small{ML} & \small{Missæ lectæ 1962.}\\
\small{MP} & \small{Missæ Propriæ OSB 1976.}\\
\small{MR} & \small{Missale romanum 2002.}\\
\small{MS} & \small{Missel de semaine Jounel.}\\
\small{Or.} & \small{Oratio universalis.}\\
\small{SO} & \small{Lectiones de Scriptura occurenti in I nocturno.}\\
\end{tabular}

\newpage
\thispagestyle{empty}
\setlength{\parskip}{0cm}

\ApplyParBox{0cm}{\ApplyGenerTitleLarge{Abbreviationes librorum}
\ApplyGenerTitleLarge{Sacræ Scripturæ}}
\setlength{\tabcolsep}{0cm}
\renewcommand{\arraystretch}{0.8}
\begin{longtable}{>{\small\bf}p{1.5cm}<{} >{\small}p{7cm}<{}}
Abd & Liber Abdiæ prophetæ\\
Act & Actus Apostolorum\\
Ag & Liber Aggæi prophetæ\\
Am & Liber Amos prophetæ\\
Ap & Apocalypsis beati Ioannis apostoli\\
Bar & Liber Baruch prophetæ\\
Cant & Canticum canticorum\\
1 et 2 Chr & Libri I et II Chronicorum ( = I et II Paralipomenon)\\
Col & Epistola beati Pauli apostoli ad Colossenses\\
1 et 2 Cor & Epistolæ I et II beati Pauli apostoli ad Corinthios\\
Dan & Liber Danielis prophetæ\\
Deut & Liber Deuteronomii\\
Eph & Epistola beati Pauli apostoli ad Ephesios\\
Esd & Liber Esdræ ( = I Esdræ)\\
Est & Liber Esther\\
Ex & Liber Exodus\\
Ez & Liber Ezechielis prophetæ\\
Gal & Epistola beati Pauli apostoli ad Galatas\\
Gen & Liber Genesis\\
Hab & Liber Habacuc prophetæ\\
Hebr & Epistola ad Hebræos\\
Iac & Epistola beati Iacobi apostoli\\
Ier & Liber Ieremiæ prophetæ\\
Io & Evangelium secundum Ioannem\\
1, 2 et 3 Io & Epistolæ I, II et III beati Ioannis apostoli\\
Iob & Liber Iob\\
Ioel & Liber Ioelis prophetæ\\
Ion & Liber Ionæ prophetæ\\
Ios & Liber Iosue\\
Is & Liber Isaiæ prophetæ\\
Iud & Epistola beati Iudæ apostoli\\
Iudic & Liber Iudicum\\
Iudt & Liber Iudith\\
Lam & Lamentationes ( = Threni)\\
Lc & Evangelium secundum Lucam\\
Lev & Liber Leviticus\\
1 et 2 Mac & Libri I et II Machabæorum\\
Mal & Liber Malachiæ prophetæ\\
Mc & Evangelium secundum Marcum\\
Mic & Liber Michææ prophetæ\\
Mt & Evangelium secundum Matthæum\\
Nah & Liber Nahum prophetæ\\
Neh & Liber Nehemiæ ( = II Esdræ)\\
Num & Liber Numeri\\
Os & Liber Oseæ prophetæ\\
1 et 2 Petr & Epistolæ I et II beati Petri apostoli\\
Phil & Epistola beati Pauli apostoli ad Philippenses\\
Phm & Epistola beati Pauli apostoli ad Philemonem\\
Prov & Liber Proverbiorum\\
Ps & Liber Psalmorum\\
Qoh & Liber Qohelet ( = Ecclesiastes)\\
1 et 2 Reg & Libri I et II Regum ( = III et IV Regum)\\
Rom & Epistola beati Pauli apostoli ad Romanos\\
Rut & Liber Ruth\\
1 et 2 Sam & Libri I et II Samuelis ( = I et II Regum)\\
Sap & Liber Sapientiæ\\
Sir & Liber Siracidæ (Ben Sira = Ecclesiasticus)\\
Soph & Liber Sophoniæ prophetæ\\
1 et 2 Th & Epistolæ I et II beati Pauli apostoli ad Thessalonicenses\\
1 et 2 Tim & Epistolæ I et II beati Pauli apostoli ad Timotheum\\
Tit & Epistola beati Pauli apostoli ad Titum\\
Tob & Liber Tobiæ\\
Zac & Liber Zachariæ prophetæ\\
\thispagestyle{empty}
\end{longtable}

\newpage
\thispagestyle{empty}
\ApplyGenerTitleLarge{Notanda}
\ApplyGenerSubTitle{in Officio~:}
\ApplyGenerList{
\item In officiis sanctorum, quando sumitur antiphona propria ad Benedictus vel ad Magnificat, dicendum est versiculum e Laudibus vel e II Vesperis de communi sanctorum aut de officio proprio, nisi aliter notetur.
\item In memoriis maioribus a dominica prima novembris usque ad Pascha, post tres psalmos primi nocturni Vigiliarum, leguntur tres lectiones de Scriptura occurrenti (nisi propriæ sint) quibus adiungitur sine mora responsorium sine Gloria Patri ~; deinde legitur lectio de sancto, post quam aliquantulum servatur sacrum silentium~; signo dato a Superiore, dicitur responsorium de sancto cum versiculo Gloria Patri.}
\ApplyGenerSubTitle{in ML~:}
\ApplyGenerList{
\item Paramenta celebrantis debent esse coloris convenientis Missæ diei aut alteræ Missæ celebrandæ (cf. Rubricæ generales Missalis romani n. 117).
\item Color paramentorum, in Missis votivis, debet esse cuique Missæ conveniens~; sed in missis votivis lectis IV classis non conventualibus, adhiberi potest etiam color Officii diei, servato tamen colore violaceo et nigro unice pro Missis quibus per se competit (Rubricæ generales Missalis romani n. 323).
\item Missa votiva IV classis est Missa votiva quæ celebrari potest tantum in diebus liturgicis IV classis (Rubricæ generales Missalis romani n. 387).
\item Missæ defunctorum IV classis sunt Missæ defunctorum «cotidianæ», quæ celebrari possunt, loco Missæ Officio diei respondentis, in feriis IV classis tantum, extra tempus natalicium. Maxime convenit ut hæ Missæ defunctorum IV classis tunc tantum dicantur cum revera pro defunctis, aut in genere aut certo designatis, applicantur (Rubricæ generales Missalis Romani n. 423). In Missis defunctorum «cotidianæ» nigro colore utendum est.}
\ApplyGenerSubTitle{in MC~:}
\ApplyGenerList{
\item In feriis cantatur præfatio in tono simplici, sed in memoriis minoribus in tono sollemni.
\item Per hebdomadam dicitur semper eadem præfatio, nisi aliter notetur.}
\thispagestyle{empty}
\vspace{2cm}\ApplyGenerTitleLarge{Advertenda}\par
\setlength{\parindent}{0.5cm}
\small{Memoriæ sanctorum sunt obligatoriæ vel ad libitum. Memoriæ obligatoriæ designantur litteris rectis (memoria maior - memoria minor), memoriæ ad libitum litteris italicis (\textit{memoria maior} - \textit{memoria minor}).\par E.g., Memoria S. Claræ, virginis, quamvis celebretur gradu memoriæ minoris, est celebratio obligatoria. Memoria Beatæ Virginis Mariæ Reginæ, quamvis sit gradu memoriæ maioris, est tamen celebratio ad libitum.}

\newpage
\thispagestyle{empty}
\ApplyParBox{1cm}{\ApplyGenerTitleHuge{Tempus Adventus}}
\ApplyGenerSubTitle{in Adventu~:}
\ApplyGenerList{
\item in Missis de tempore silent organa, excepta dominica \textit{Gaudete} in Missa.
\item ad Completorium~: ø \textit{Alma Redemptoris}.}
\ApplyGenerSubTitle{in Officio de tempore~:}
\ApplyGenerList{
\item omnes genua flectunt in Officio feriali ad \textit{Kyrie eleison}.
\item ad Vigilias~: invitatorium \textit{Regem venturum} usque ad diem 16 decembris.
\item ad Laudes~: dicuntur cantica ferialia, quod servatur etiam in memoriis.
\item ad Benedictus et Magnificat~: antiphonæ propriæ.
\item ad Horas minores~: antiphonæ dominicæ præcedentis (usque ad diem 17 decembris)~; in memoriis cantatur hymnus in tono Adventus.}
\ApplyGenerSubTitle{in MC~:}
\ApplyGenerList{
\item Missa propria singulis diebus.
\item a prima dominica Adventus usque ad diem 16 decembris dicitur præfatio I de Adventu in Missis de tempore et in ceteris Missis, quæ celebrantur eodem tempore et præfatione propria carent.
\item in feriis Adventus~: \textit{Pater} in tono B~; \textit{Mysterium fidei} et \textit{Per ipsum} in tono simplici.}
\ApplyGenerSubTitle{in ML~:}
\ApplyGenerList{
\item dicitur præfatio de Adventu ut supra.
\item quando resumitur Missa de dominica, post graduale omittuntur \textit{Alleluia} et versiculus sequens.
\item Missæ votivæ IV classis et Missæ defunctorum «cotidianæ» non permittuntur.}
\ApplyHebdoPsalt{\textbf{- pro breviario 62~: tomus prior -}}

\ApplyNewMonthTitles{December}
\ApplyHeader{\textbf{1} & Sabbato § - de eo - \textit{Viol.}}
\ApplyBody{
\item I Vesperæ dominicæ sequentis.}
\ApplyHebdoPsalt{- hebdomada I psalterii -}
\ApplyHeader{\textbf{2} & ŧ \textbf{\textsc{Dominica I Adventus}} - de ea - \textit{Viol.}}
\ApplyBody{\item ad Vigilias~: post lectiones I nocturni, dicitur ¶ lectionis 1 \textit{Aspiciens a longe}.}
\ApplyLectHeader{
Hebdomada I Adventus~: lectiones de tempore}
\ApplyLectBody{
\item[Dom. C] Ier \textbf{33}, 14-16 / 1 Th \textbf{3}, 12–4, 2 Lc \textbf{21}, 25-28.34-36
\item[Feria II] Is \textbf{2}, 1-5 / Mt \textbf{8}, 5-11
\item[Feria III] Is \textbf{11}, 1-10 / Lc \textbf{10}, 21-24
\item[Feria IV] Is \textbf{25}, 6-10a / Mt \textbf{15}, 29-37
\item[Feria V] Is \textbf{26}, 1-6 / Mt \textbf{7}, 21.24-27
\item[Feria VI] Is \textbf{29}, 17-24 / Mt \textbf{9}, 27-31
\item[Sabbato] Is \textbf{30}, 19-21.23-26 / Mt \textbf{9}, 35–10, 1.6-8}
\ApplyHeader{\textbf{3} & Feria II - \textsc{S. Francisci Xavier}, presbyteri - \textbf{memoria maior} - \textit{Alb.}}
\ApplyBody{
\item ad Vigilias~: lectio de memoria in supplemento 60.
\item ad Benedictus~: ø \textit{Euntes} (AM 484).
\item in MC~: præfatio de sanctis virginibus et religiosis.
\item ad Magnificat~: ø \textit{Alias oves} (AM 486).}
\ApplyHeader{\textbf{4} & Feria III - de ea - \textit{Viol.}}
\ApplyHeader{\textbf{5} & Feria IV ł - de ea - \textit{Viol.}}
\ApplyHeader{\textbf{6} & Feria V - de ea - \textit{Viol.}}
\ApplyHeader{\textbf{7} & Feria VI µ £ - \textsc{S. Ambrosii}, episcopi et Ecclesiæ doctoris - \textbf{memoria maior} - \textit{Alb.}}
\ApplyBody{
\item ad Benedictus~: ø \textit{Paraclitus} (AM 531).
\item \textit{in ML~: Missa in proprio sanctorum vel in PAL.}
\item in MC~: præfatio de sanctis pastoribus.
\item I Vesperæ sollemnitatis sequentis.}
\ApplyHeader{\textbf{8} & Sabbato - ¬ \textbf{\MakeUppercase{In Conceptione Immaculata Beatæ Mariæ Virginis} - sollemnitas maior} - \textit{Alb.}}
\ApplyBody{
\item ad Vigilias~: in nocturno II~: lectiones 5, 6 et 7 cum ¶ lectionis 8~; in nocturno III~: lectiones 9 et 10.
\item in MC~: lectiones propriæ~: Gen \textbf{3}, 9-15.20 / Eph \textbf{1}, 3-6.11-12 / Lc \textbf{1}, 26-38~; præfatio propria.\item I Vesperæ dominicæ sequentis.}
\ApplyHebdoPsalt{- hebdomada II psalterii -}
\ApplyHeader{\textbf{9} & \textbf{\textsc{Dominica II Adventus}} - de ea - \textit{Viol.}}
\ApplyLectHeader{
Hebdomada II Adventus~: lectiones de tempore}
\ApplyLectBody{
\item[Dom. C] Bar \textbf{5}, 1-9 / Phil \textbf{1}, 4-6.8-11 / Lc \textbf{3}, 1-6
\item[Feria II] Is \textbf{35}, 1-10 / Lc \textbf{5}, 17-26
\item[Feria III] Is \textbf{40}, 1-11 / Mt \textbf{18}, 12-14
\item[Feria IV] Is \textbf{40}, 25-31 / Mt \textbf{11}, 28-30
\item[Feria V] Is \textbf{41}, 13-20 / Mt \textbf{11}, 11-15
\item[Feria VI] Is \textbf{48}, 17-19 / Mt \textbf{11}, 16-19
\item[Sabbato] Sir \textbf{48}, 1-4.9-11 / Mt \textbf{17}, 10-13}
\ApplyHeader{\textbf{10} & Feria II - de ea - \textit{Viol.}}
\ApplyHeader{\textbf{11} & Feria III - de ea - \textit{Viol.}}
\ApplyHeader{\textbf{12} & Feria IV ł - Beatæ Mariæ Virginis de Guadalupe - \textit{memoria minor} - \textit{Viol.}}
\ApplyBody{
\item ad Benedictus~: ø \textit{Viderunt eam} (AM 1073)~; oratio in supplemento 60*.
\item \textit{in ML~: Alb.}
\item in MC \textit{(Alb.)}~: Commune Beatæ Mariæ Virginis (MR 905)~; præfatio I de Beata Maria Virgine.}
\ApplyAnniv{\textup{†} Cras recurrit anniversarium obitus Reverendissimi Patris Fulgentii Mariæ \textsc{Lagrace}, prioris, qui die 13 decembris 1966 in Abbatia Dominæ Nostræ Mayliliensis obdormivit in Domino.}

\ApplyHeader{\textbf{13} & Feria V - \textsc{S. Luciæ}, virginis et martyris - \textbf{memoria maior} - \textit{Rub.}}
\ApplyBody{
\item ad Vigilias pro breviario vetere~: in supplemento 60.
\item ad Laudes, Vesperas et Horas minores~: antiphonæ propriæ.
\item ad Laudes~: hymnus \textit{Iesu corona} (AM 677).
\item in MC~: Commune virginis martyris (MR 924)~; præfatio de sanctis martyribus.
\item ad Vesperas~: hymnus \textit{Iesu corona} ut supra~; ad Magnificat~: ø \textit{In tua patientia} (AM 769).}
\ApplyAnniv{\textup{†} Cras recurrit anniversarium obitus Reverendissimi Patris Emmanuel Mariæ \textsc{Sarramagnan}, prioris, qui die 14 decembris 2005 in Abbatia Dominæ Nostræ Mayliliensis obdormivit in Domino.}

\ApplyHeader{\textbf{14} & Feria VI ł - \textsc{S. Ioannis a Cruce}, presbyteri et Ecclesiæ doctoris - \textbf{memoria maior} - \textit{Alb.} (olim die 24 novembris).}
\ApplyBody{
\item ad Vigilias~: lectio de memoria in supplemento 61.
\item ad Benedictus~: ø \textit{Qui vult} (AM 644).
\item \textit{in ML~: Missa in PAL.}
\item in MC~: præfatio de sanctis virginibus et religiosis.}
\ApplyHeader{\textbf{15} & Sabbato - de eo - \textit{Viol.}}
\ApplyBody{
\item I Vesperæ dominicæ sequentis.}
\ApplyHebdoPsalt{- hebdomada I psalterii -}
\ApplyHeader{\textbf{16} & \textbf{\textsc{Dominica III Adventus}, Gaudete} - de ea - \textit{Viol.}}
\ApplyBody{\item ad Vigilias~: continuatur invitatorium \textit{Regem venturum} usque ad diem 16 decembris.\item hodie pulsantur organa ad Missam, non vero in aliis Horis.}
\ApplyLectHeader{
Hebdomada III Adventus~: lectiones de tempore}
\ApplyLectBody{
\item[Dom. C] Soph \textbf{3}, 14-18a / Phil \textbf{4}, 4-7 / Lc \textbf{3}, 10-18
\item[17 dec.] Gen \textbf{49}, 2.8-12 / Mt \textbf{1}, 1-17
\item[18 dec.] Ier \textbf{23}, 5-8 / Mt \textbf{1}, 18-24
\item[19 dec.] Iudic \textbf{13}, 2-7.24-25a / Lc \textbf{1}, 5-25
\item[20 dec.] Is \textbf{7}, 10-14 / Lc \textbf{1}, 26-38
\item[21 dec.] Cant \textbf{2}, 8-14 / Lc \textbf{1}, 39-45
\item[22 dec.] 1 Sam \textbf{1}, 24-28 / Lc \textbf{1}, 46-56}
\ApplyGenerSubTitle{a die 17 ad diem 23 decembris inclusive~:}
\ApplyGenerList{
\item ad Vigilias~: invitatorium \textit{Prope est}~; lectiones SO in supplemento 9 et sequentibus pro singulis feriis.
\item ad Laudes~: antiphonæ propriæ.
\item ad Benedictus~: antiphonæ propriæ.
\item ad Horas et Vesperas~: antiphonæ e Laudibus.
\item ad Magnificat~: antiphonæ “O” (AM 208 - 211).
\item in Missa conventuali~: præfatio II de Adventu~; in feriis dicuntur orationes et lectiones diei mensis adsignatæ.}
\ApplyAnniv{\textup{†} Cras recurrit anniversarium obitus fratris nostri Ioseph Mariæ \textsc{Bumat}, qui die 17 decembris 1997 obdormivit in Domino.}

\ApplyHeader{\textbf{17} & Feria II - de ea - \textit{Viol.}}
\ApplyBody{
\item ad omnes Horas ø \textit{Ecce veniet} (AM 212).
\item ad Benedictus~: ø \textit{Egredietur} (AM 220).}
\ApplyHeader{\textbf{18} & Feria III - de ea - \textit{Viol.}}
\ApplyBody{
\item ad omnes Horas ø \textit{Rorate} (AM 213).
\item ad Benedictus~: ø \textit{Tu Bethlehem} (AM 221).}
\ApplyHeader{\textbf{19} & Feria IV ł - de ea - \textit{Viol.}}
\ApplyBody{
\item ad omnes Horas ø \textit{Prophetæ} (AM 215).\item in Officio~: oratio Quatuor Temporum Adventus, præter ad Vesperas.
\item ad Benedictus~: ø \textit{Missus est} (AM 222).\item \textit{in ML~: Quatuor Temporum Adventus}.}
\ApplyHeader{\textbf{20} & Feria V - de ea - \textit{Viol.}}
\ApplyBody{
\item ad omnes Horas ø \textit{De Sion} (AM 216).
\item ad Benedictus~: ø \textit{Vigilate} (AM 223).\item ad Vesperas~: divisiones psalmorum considerantur ut integri psalmi, ita ut eis antiphonæ propriæ sint attribuendæ.}
\ApplyHeader{\textbf{21} & Feria VI ł - de ea - \textit{Viol.}}
\ApplyBody{
\item ad omnes Horas ø \textit{Constantes} (AM 217).\item in Officio~: oratio Quatuor Temporum Adventus, præter ad Vesperas.
\item ad Benedictus~: ø \textit{Nolite timere} (AM 219).\item \textit{in ML~: Quatuor Temporum Adventus}.\item ad Vesperas~: divisiones psalmorum considerantur ut integri psalmi, ita ut eis antiphonæ propriæ sint attribuendæ.}
\ApplyHeader{\textbf{22} & Sabbato - de eo - \textit{Viol.}}
\ApplyBody{
\item ad Laudes et Horas minores~: ø \textit{Intuemini} in variationibus 14.\item in Officio~: oratio Quatuor Temporum Adventus.
\item ad Benedictus~: ø \textit{Quomodo fiet istud} (AM 225).\item \textit{in ML~: Quatuor Temporum Adventus \textit{(formula brevior)}}.\item I Vesperæ dominicæ sequentis (AM 226).}
\ApplyHebdoPsalt{- hebdomada II psalterii -}
\ApplyHeader{\textbf{23} & \textbf{\textsc{Dominica IV Adventus}} - de ea - \textit{Viol.}}
\ApplyBody{\item ad Vigilias~: in I nocturno lectiones e dominica IV Adventus.
\item ad Benedictus~: ø Ecce completa sunt (AM 220).}
\ApplyLectHeader{
Hebdomada IV Adventus~: lectiones de tempore}
\ApplyLectBody{
\item[Dom. C] Mic \textbf{5}, 2-5a / Hebr \textbf{10}, 5-10 / Lc \textbf{1}, 39-45
\item[24 dec.] 2 Sam \textbf{7}, 1-5.8b-12.16 / Lc \textbf{1}, 67-79}
\ApplyAnniv{\textup{†} Cras recurrit anniversarium obitus Reverendissimi Patris Bernardi Mariæ \textsc{Maréchaux}, abbatis, qui die 24 decembris 1927 in Monasterio Dominæ Nostræ Sanctæ Spei Mesnili obdormivit in Domino.}

\ApplyHeader{\textbf{24} & Feria II - de ea - \textsc{Vigilia Nativitatis Domini} - \textit{Viol.}}
\ApplyBody{\item ad Vigilias~: invitatorium \textit{Hodie scietis}~; lectiones SO in supplemento 20.
\item ad Laudes~: antiphonæ propriæ cum psalmis festivis.
\item \textit{in ML~: Missa de Vigilia.}
\item in MC~: cantus sumuntur e Missa \textit{Hodie} de Vigilia (GR 38).\item I Vesperæ sollemnitatis sequentis.
\item Completorium omittitur ab his qui solemnem Vigiliam et Missam in nocte intersunt.}

\ApplyGenerSubTitle{ad mensam~:}
\ApplyGenerList{\item benedictio de Nativitate.}
\vspace{1cm}
\ApplyHeader{\textbf{25} & Feria III - ¬ \textbf{\MakeUppercase{In Nativitate Domini Nostri Iesu Christi} - sollemnitas maior cum octava} - \textit{Alb}.}
\ApplyBody{\item ad Vigilias~: psalmi hebdomadæ I.
\item hodie omnes sacerdotes tres Missas celebrare possunt, dummodo hæ suo tempore celebrentur.
\item in omnibus Missis Nativitatis ad verba symboli \textit{Et incarnatus est} omnes genua flectunt.
\item ad Missam in nocte~: ad hymnum \textit{Gloria} pulsantur campanæ~; præfatio I de Nativitate Domini~; \textit{Communicantes} proprium hodie et per totam octavam.
\item Laudes in aurora celebrantur.
\item ad Missam in die~: præfatio II de Nativitate Domini.
\item Vesperæ sollemnitatis~; benedictio Sanctissimi Sacramenti.}
\ApplyLectHeader{
In Nativitate Domini Nostri~: lectiones de tempore}
\ApplyLectBody{
\item[In nocte] Is \textbf{9}, 2-4.6-7 / Tit \textbf{2}, 11-14 / Lc \textbf{2}, 1-14
\item[In aurora] Is \textbf{62}, 11-12 / Tit \textbf{3}, 4-7 / Lc \textbf{2}, 15-20
\item[In die] Is \textbf{52}, 7-10 / Hebr \textbf{1}, 1-6 / Io \textbf{1}, 1-18
\item[29 dec.] 1 Io \textbf{2}, 3-11 / Lc \textbf{2}, 22-35
\item[31 dec.] 1 Io \textbf{2}, 18-21 / Io \textbf{1}, 1-18}
\ApplyHeader{\textbf{26} & Feria IV ł - \textbf{\textsc{S. Stephani, protomartyris}} - \textbf{festum} - \textit{Rub.}}
\ApplyBody{
\item in MC~: lectiones propriæ~: Act \textbf{6}, 8-10~; \textbf{7}, 54-60 / Mt \textbf{10}, 17-22~; præfatio I de Nativitate.
\item ad Vesperas~: antiphonæ et psalmi de Nativitate~; a capitulo de festo.}
\ApplyHeader{\textbf{27} & Feria V - \textbf{\textsc{S. Ioannis, apostoli et evangelistæ}} - \textbf{festum} - \textit{Alb.}}
\ApplyBody{
\item in MC~: lectiones propriæ~: 1 Io \textbf{1}, 1-4 / Io \textbf{20}, 2-8~; præfatio I de Nativitate.
\item ad Vesperas~: antiphonæ et psalmi de Nativitate~; a capitulo de festo.}
\ApplyHeader{\textbf{28} & Feria VI ł - \textbf{\textsc{Ss. Innocentium, martyrum}} - \textbf{festum} - \textit{Rub.}}
\ApplyBody{
\item in MC~: lectiones propriæ~: 1 Io \textbf{1}, 5–2, 2 / Mt \textbf{2}, 13-18~; præfatio I de Nativitate.
\item ad Vesperas~: antiphonæ et psalmi de Nativitate~; a capitulo de festo.}
\ApplyHeader{\textbf{29} & Sabbato - \textbf{\textsc{De Octava Nativitatis}} - \textit{Alb}.}
\ApplyBody{\item in Officio~: omnia dicuntur sicut in Nativitate præter Vigilias.\item ad Laudes~: ß \textit{Benedicamus Domino} III.\item in MC~: \textit{Gloria}~; præfatio II de Nativitate.\item I Vesperæ festi sequentis (AM 265).}
\ApplyHebdoPsalt{- hebdomada I psalterii -}
\ApplyHeader{\textbf{30} & \textbf{\textsc{Dominica I post nativitatem - Sanctæ Familiæ Iesu, Mariæ et Ioseph}} - \textbf{festum} - \textit{Alb}.}
\ApplyBody{\item Officium dicitur de dominica infra octavam Nativitatis, præter orationem in Supplemento 22*~; invitatorium proprium in supplemento 59.
\item ad Vigilias~: psalmi hebdomadæ II.\item ad Benedictus~: ø \textit{Remansit} (AM 301).\item \textit{in ML~: Missa dominicæ I post Epiphaniam, Sanctæ Familiæ, cum præfatione et \emph{Communicantes} de Nativitate.}
\item in MC~: lectiones propriæ~: 1 Sam \textbf{1}, 20-22.24-28 / 1 Jo \textbf{3}, 1-2.21-24 / Lc \textbf{2}, 41-52~; præfatio II de Nativitate.\item ad Magnificat~: ø \textit{Fili, quid fecisti} (AM 303).}
\ApplyHeader{\textbf{31} & Feria II - \textbf{\textsc{De Octava Nativitatis}} - \textit{Alb}.}
\ApplyBody{\item in Officio~: omnia dicuntur sicut in Nativitate præter Vigilias.\item ad Laudes et Vesperas~: ß \textit{Benedicamus Domino} III.\item in MC~: \textit{Gloria}~; præfatio II de Nativitate.\item Vesperæ sollemnitatis Sanctæ Dei Genetricis Mariæ.
\item ante Completorium~: cantatur hymnus \textit{Te Deum} cum versiculis et oratione in gratiarum actionem (GR 841)~: indulgentia plenaria.}
\renewcommand{\CurrentYear}{2019}

\ApplyNewMonthTitles{Ianuarius}
\ApplyHeader{\textbf{1} & Feria III - ¬ \textbf{\MakeUppercase{sollemnitas Sanctæ Dei Genetricis Mariæ}} - \textbf{sollemnitas maior} - \textit{Alb.}}
\ApplyBody{
\item Officium et Missæ lectæ dicuntur de octava Nativitatis (vel Circumcisione) præter invitatorium in supplemento 58.
\item in MC~: cantatur hymnus \textit{Veni Creator} sine versiculo neque oratione (AM 1254), prima stropha dicitur flexis genibus post formulam salutationis et ante Tertiam (indulgentia plenaria)~; lectiones propriæ~: Num \textbf{6}, 22-27 / Gal \textbf{4}, 4-7 / Lc \textbf{2}, 16-21~; præfatio I de Beata Maria Virgine (\textit{Et te in maternitate})~; \textit{Communicantes} proprium.
\item Vesperæ sollemnitatis~; benedictio Sanctissimi Sacramenti.}

\newpage
\ApplyParBox{1cm}{\ApplyGenerTitleHuge{Tempus Nativitatis I}}
\ApplyGenerTitleLarge{Usque ad nonam diei 5 ianuarii}
\ApplyGenerSubTitle{in Officio~:}
\ApplyGenerList{\item ad Vigilias~: omnia ut in ordinario officii ferialis tempore Nativitatis (in supplemento 28 pro breviario vetere), præter capitulum Hebr. 1, 10 ut in psalterio~; lectiones SO (in supplemento 30 pro breviario vetere).
\item ad Laudes et Vesperas~: antiphonæ et psalmi de feria, reliqua ut in die 1 ianuarii, nisi aliter notetur.
\item ß \textit{Benedicamus Domino}~: ad Laudes VI$_2$~; ad Vesperas VI$_1$.
\item ad Horas minores~: antiphonæ et reliqua ut in die 1 ianuarii.
\item continuatur tonus Nativitatis ad Horas minores et Completorium.}
\ApplyGenerSubTitle{in ML~:}
\ApplyGenerList{\item præfatio de Nativitate.
\item Missæ defunctorum «cotidianæ» non permittuntur.}
\ApplyGenerSubTitle{in MC~:}
\ApplyGenerList{\item præfatio de Nativitate III, nisi aliter notetur.}
\ApplyGenerSubTitle{ad mensam~:}
\ApplyGenerList{\item benedictio de Nativitate.}
\ApplyLectHeader{Lectiones de tempore}
\ApplyLectBody{
\item[2 ian.] 1 Io \textbf{2}, 22–28 / Io \textbf{1}, 19-28
\item[3 ian.] 1 Io \textbf{2}, 29–3, 6 / Io \textbf{1}, 29-34
\item[4 ian.] 1 Io \textbf{3}, 7-10 / Io \textbf{1}, 35-42
\item[5 ian.] 1 Io \textbf{3}, 11-21 / Io \textbf{1}, 43-51}
\medskip
\ApplyHeader{\textbf{2} & Feria IV ł - \textsc{Ss. Basilii Magni et Gregorii Nazianzeni}, episcoporum et Ecclesiæ doctorum - \textbf{memoria maior} - \textit{Alb.}}
\ApplyBody{
\item in Officio~: oratio in supplemento 65*.
\item ad Vigilias~: hymnus \textit{Inclitos Christi} in supplemento 66~; lectiones SO~; lectio de memoria in supplemento 63*.
\item ad Laudes et Vesperas~: antiphonæ et psalmi de feria, a capitulo ut in variationibus 26 et sequentibus.
\item ad Benedictus~: ø \textit{Qui sperant} in variationibus 31.
\item ad Horas minores~: antiphonæ, capitulum et versiculi ut in variationibus 25.
\item \textit{in ML~: Missa plurium confessorum pontificum in PAL.}
\item in MC~: lectiones feriales~; præfatio de sanctis pastoribus.
\item ad Magnificat~: ø \textit{Sapientiam sanctorum} (AM 981).}
\ApplyHeader{\textbf{3} & Feria V - \textsc{Sanctissimi Nominis Iesu} - \textbf{memoria maior} - \textit{Alb.}}
\ApplyBody{
\item in Officio~: oratio in supplemento 67*.
\item ad Vigilias~: invitatorium, hymnus et lectio de memoria in supplemento 65*.
\item ad Laudes et Horas minores~: AM 279.
\item \textit{in ML~: olim dominica a die 2 ad diem 5 ianuarii occurrente (non dicitur \emph{Credo}).} 
\item in MC~: lectiones propriæ~: Phil \textbf{2}, 1-11 / Lc \textbf{2}, 21-24.\item ad Vesperas~: AM 276~; ad Magnificat~: ø \textit{Vocabis} (AM 283).}
\ApplyHeader{\textbf{4} & Feria VI µ £ - de ea - \textit{Alb}.}
\ApplyBody{
\item \textit{in ML~: Missa de sacratissimo Corde Iesu \emph{(Gloria)}.}}
\ApplyHeader{\textbf{5} & Sabbato § - de eo - \textit{Alb}.}
\ApplyBody{
\item ad Benedictus~: ø \textit{Illuminare} (AM 586).
\item \textit{in ML (Alb.) ~: Immaculati Cordis Beatæ Mariæ Virginis.}\item I Vesperæ sollemnitatis sequentis.}

\ApplyGenerSubTitle{ad mensam~:}
\ApplyGenerList{
\item benedictio de Epiphania.}
\vspace{0.5cm}
\ApplyHebdoPsalt{- hebdomada II psalterii -}
\ApplyHeader{\textbf{6} & ŧ \textbf{\textsc{Dominica II post Nativitatem}} - ¬  \textbf{\MakeUppercase{In Epiphania Domini}} - \textbf{sollemnitas maior} - \textit{Alb.}}
\ApplyBody{
\item in MC~: lectiones propriæ~: Is \textbf{60}, 1-6 / Eph \textbf{3}, 2-3a 5-6 / Mt \textbf{2}, 1-12~; præfatio et \textit{Communicantes} de Epiphania.
\item Vesperæ sollemnitatis~; benedictio Sanctissimi Sacramenti.}
\ApplyLectHeader{
Dominica II post Nativitatem~: lectiones de tempore}
\ApplyLectBody{
\item[Dom. C]Sir \textbf{24}, 1-4.12-16 / Eph \textbf{1}, 3-6.15-18 / Io \textbf{1}, 1-18}

\newpage
\ApplyParBox{0cm}{\ApplyGenerTitleHuge{Tempus Nativitatis II}}\ApplyGenerTitleLarge{Ab Epiphania usque ad}
\ApplyGenerTitleLarge{festum Baptismatis Domini}
\ApplyGenerSubTitle{in Officio~:}
\ApplyGenerList{
\item oratio de Epiphania \textit{Deus, qui hodierna die}.
\item ad Vigilias~: sumitur ordinarium officii ferialis tempore Epiphaniæ (in supplemento 33 pro breviario vetere)~; lectiones SO.
\item ad Laudes et Vesperas~: antiphonæ et psalmi de feria, reliqua ut in festo Epiphaniæ~; ad Benedictus et Magnificat~: antiphonæ propriæ (AM 297 et sequentibus).
\item ß \textit{Benedicamus Domino}~: ad Laudes VI$_2$~; ad Vesperas VI$_1$.
\item ad Horas minores~: antiphonæ de Epiphania.
\item continuatur tonus Epiphaniæ ad Horas minores et Completorium.}
\ApplyGenerSubTitle{in ML~:}
\ApplyGenerList{
\item præfatio de Epiphania.
\item Missæ defunctorum «cotidianæ» non permittuntur.}
\ApplyGenerSubTitle{in MC~:}
\ApplyGenerList{
\item præfatio de Epiphania.}
\ApplyGenerSubTitle{ad mensam~:}
\ApplyGenerList{
\item benedictio de Epiphania.}
\vspace{1cm}
\ApplyLectHeader{Post Epiphaniam Domini~: lectiones de tempore}
\ApplyLectBody{
\item[7 ian.] 1 Io \textbf{3}, 22–4, 6 / Mt \textbf{4}, 12-17.23-25
\item[8 ian.] 1 Io \textbf{4}, 7-10 / Mc \textbf{6}, 34-44
\item[9 ian.] 1 Io \textbf{4}, 11-18 / Mc \textbf{6}, 45-52
\item[10 ian.] 1 Io \textbf{4}, 19–5, 4 / Lc \textbf{4}, 14-22a
\item[11 ian.] 1 Io \textbf{5}, 5-6.8-13 / Lc \textbf{5}, 12-16
\item[12 ian.] 1 Io \textbf{5}, 14-21 / Io \textbf{3}, 22-30}
\medskip
\ApplyHeader{\textbf{7} & Feria II - de ea - \textit{Alb}.}
\ApplyBody{
\item ad Vigilias~: lectio \textit{Veritatem} de epistola ad Romanos (pro breviario vetere in supplemento 36) et sic usque ad sabbatum.\par
\item ad Benedictus~: ø \textit{Ab Oriente} (AM 297).
\item in MC~: lectiones feriales post Epiphaniam.
\item ad Magnificat~: ø \textit{Videntes} (AM 298).}
\ApplyHeader{\textbf{8} & Feria III - de ea - \textit{Alb}.}
\ApplyBody{
\item ad Benedictus~: ø \textit{Tria sunt munera} (AM 298).
\item ad Magnificat~: ø \textit{Lux de luce} (AM 298).}
\ApplyHeader{\textbf{9} & Feria IV ł - de ea - \textit{Alb}.}
\ApplyBody{
\item ad Benedictus~: ø \textit{Vidimus} (AM 299).
\item ad Magnificat~: ø \textit{Interrogabat} (AM 299).}
\ApplyHeader{\textbf{10} & Feria V - de ea - \textit{Alb}.}
\ApplyBody{
\item ad Benedictus~: ø \textit{Omnes nationes} (AM 299).
\item ad Magnificat~: ø \textit{Omnes de Saba} (AM 300).}
\ApplyHeader{\textbf{11} & Feria VI ł - de ea - \textit{Alb}.}
\ApplyBody{
\item ad Benedictus~: ø \textit{Venient ad te} (AM 300).
\item ad Magnificat~: ø \textit{Admoniti Magi} (AM 300).}
\ApplyHeader{\textbf{12} & Sabbato - de eo - \textit{Alb}.}
\ApplyBody{
\item ad Benedictus~: ø \textit{Mirabile} (AM 717).\item I Vesperæ festi sequentis~: antiphonæ propriæ in variationibus 37, reliqua ut in festo Epiphaniæ~; ad Magnificat~: ø \textit{Et ecce vox} (AM 1000)~; oratio in variationibus 38.}
\ApplyHebdoPsalt{- hebdomada I psalterii -}
\ApplyHeader{\textbf{13} &  \textbf{\textsc{Dominica III post Nativitatem - Dominica in Baptismate Domini}} (I post Epiphaniam) - \textbf{festum} - \textit{Alb}.}
\ApplyBody{\item ad Vigilias~: olim die 13 ianuarii pro breviario 62~; in octava Epiphaniæ vel in supplemento 41 pro breviario vetere~; in I nocturno lectiones e dominica I post Epiphaniam cum responsorio \textit{Hodie in Iordane} (post lectionem I).
\item ad Laudes et Horas minores~: antiphonæ propriæ in variationibus 37, reliqua ut in festo Epiphaniæ excepta oratione in variationibus 38 vel in die octava Epiphaniæ.
\item \textit{in ML~: Missa in Commemoratione Baptismatis Domini Nostri Iesu Christi.}
\item in MC~: lectiones propriæ~: Is \textbf{40}, 1-5.9-11 / Tit \textbf{2}, 11-14~; \textbf{3}, 4-7 / Lc \textbf{3}, 15-16.21-22~; præfatio propria.
\item ad Vesperas~: antiphonæ e Laudibus, reliqua ut in festo Epiphaniæ excepta oratione in variationibus 38 vel in die octava Epiphaniæ.}
\ApplyParBox{1cm}{\begin{center}\large{\textit{Post Completorium explicit}}\par\large{\textit{tempus Nativitatis.}}\end{center}}

\newpage
\ApplyParBox{1cm}{\ApplyGenerTitleHuge{Tempus Per annum}}
\ApplyGenerSubTitle{in Officio~:}
\ApplyGenerList{
\item in feriis~: hebdomada I per annum vel I post Epiphaniam.}
\ApplyGenerSubTitle{ad mensam~:}
\ApplyGenerList{
\item benedictio de tempore per annum.}
\ApplyLectHeader{Hebdomada I per Annum~: lectiones de tempore}
\ApplyLectBody{
\item[Feria II] Hebr \textbf{1}, 1-6 / Mc \textbf{1}, 14-20
\item[Feria III] Hebr \textbf{2}, 5-12 / Mc \textbf{1}, 21-28
\item[Feria IV] Hebr \textbf{2}, 14-18 / Mc \textbf{1}, 29-39
\item[Feria V] Hebr \textbf{3}, 7-14 / Mc \textbf{1}, 40-45
\item[Feria VI] Hebr \textbf{4}, 1-5.11 / Mc \textbf{2}, 1-12
\item[Sabbato] Hebr \textbf{4}, 12-16 / Mc \textbf{2}, 13-17}
\ApplyGenerList{
\item præfatio communis I, nisi aliter notetur.}
\medskip
\ApplyHeader{\textbf{14} & Feria II - S. Remigii, episcopi - \textit{memoria minor} - \textit{Vir.}}
\ApplyBody{
\item ad Benedictus~: ø \textit{Sacerdos} (AM 656)~; oratio in supplemento 66.
\item \textit{in ML (Alb.)~: olim die 1 octobris.}
\item in MC \textit{(Alb.)}~: omnia in MP.}
\ApplyHeader{\textbf{15} & Feria III - \textsc{Ss. Mauri et Placidi}, discipulorum SPN Benedicti - \textbf{memoria maior} - \textit{Alb.} (olim die 5 octobris).}
\ApplyBody{
\item in Officio~: oratio in supplemento 70 vel in variationibus 28.
\item ad Vigilias~: hymnus proprius in supplemento 66~; lectio de memoria in supplemento 67.
\item ad Laudes, Vesperas et Horas minores~: Officium dicitur ut in AM 778.\item ad Benedictus~: ø \textit{Cum Placidus} (AM 965).\item \textit{in ML~: Missa plurium confessorum non pontificum in PAL cum Evangelio S. Mauri (die 15 ianuarii in supplemento OSB).}
\item in MC~: collecta in MP~; Commune sanctorum et sanctarum (MR 962)~; lectiones propriæ~: Sir \textbf{51}, 12-19a.20.27 / Mt \textbf{14}, 22-33~; præfatio de sanctis virginibus et religiosis.
\item ad Magnificat~: ø \textit{Hodie} (AM 782).}
\ApplyAnniv{\textup{†} Cras recurrit anniversarium obitus Reverendissimi ac dilectissimi Patris Augustini Mariæ \textsc{Joly}, abbatis et fundatoris Monasterii nostri, qui die 16 ianuarii 2006 obdormivit in Domino.}

\ApplyHeader{\textbf{16} & Feria IV ł - de ea - \textit{Vir.}}
\ApplyAnniv{Cras recurrit anniversarium erectionis S. Ioseph Claræ Vallis in titulum abbatialem (1992).}

\ApplyHeader{\textbf{17} & Feria V - \textsc{S. Antonii}, abbatis - \textbf{memoria maior} - \textit{Alb.}}
\ApplyBody{
\item in MC~: præfatio de sanctis virginibus et religiosis.
\item ad Magnificat~: ø \textit{Vos qui reliquistis} (AM 624).}
\ApplyAnniv{Cras incipit hebdomada precibus pro christianorum unitate dedicata.}

\ApplyHeader{\textbf{18} & Feria VI ł - de ea - \textit{Vir.}}
\ApplyBody{
\item \textit{in ML (Viol)~: Missa votiva pro Ecclesiæ unitate n. 20.}
\item in MC \textit{(Viol)}~: Missa pro unitate christianorum A (MR 1111)~; lectiones propriæ~: 1 Io \textbf{4}, 9-15 / Io \textbf{17}, 20-26~; præfatio propria.}
\ApplyHeader{\textbf{19} & Sabbato  - \textsc{De Beata} - \textit{\textbf{memoria maior}} - \textit{Alb.}}
\ApplyBody{
\item ad Vigilias~: lectio de memoria sabbato 3 (in supplemento 197 pro breviario vetere)~; oratio \textit{Deus, qui salutis æternæ}.
\item ad Laudes et Horas minores~: officium proprium post Nativitatem Domini (AM 716).
\item ad Benedictus~: ø \textit{Magnum} (AM 716).
\item in MC~: \textit{Beatæ Mariæ Virginis, Matris Salvatoris} (CM 5)~; præfatio I de Beata Maria Virgine.
\item I Vesperæ dominicæ sequentis.}
\ApplyHebdoPsalt{- hebdomada II psalterii -}
\ApplyHeader{\textbf{20} & \textbf{\textsc{Dominica II per annum}} (II post Epiphaniam) - de ea - \textit{Vir.}}
\ApplyBody{
\item in MC~: præfatio I de dominicis.
\item ad benedictionem Sanctissimi Sacramenti, post canticum expositionis, cantatur \textit{Ubi Caritas} in Besnier 275 (hebdomada pro christianorum unitate).}
\ApplyLectHeader{Hebdomada II per Annum~: lectiones de tempore}
\ApplyLectBody{
\item[Dom. C] Is \textbf{62}, 1-5 / 1 Cor \textbf{12}, 4-11 / Io \textbf{2}, 1-12
\item[Feria II] Hebr \textbf{5}, 1-10 / Mc \textbf{2}, 18-22
\item[Feria III] Hebr \textbf{6}, 10-20 / Mc, \textbf{2}, 23-28
\item[Feria IV] Hebr \textbf{7}, 1-3.15-17 / Mc \textbf{3}, 1-6
\item[Feria V] Hebr \textbf{7}, 25–8, 6 / Mc \textbf{3}, 7-12
\item[Feria VI] Hebr \textbf{8}, 6-13 / Mc \textbf{3}, 13-19
\item[Sabbato] Hebr \textbf{9}, 2-3.11-14 / Mc \textbf{3}, 20-21}
\ApplyPrefaceFeries{
\item in feriis~: præfatio communis II, nisi aliter notetur.}
\ApplyAnniv{\textup{†} Cras recurrit anniversarium obitus Henrici \textsc{Vergez}, sacerdotis et benefactoris, qui die 21 ianuarii 1973 obdormivit in Domino.}

\ApplyHeader{\textbf{21} & Feria II - \textsc{S. Agnetis}, virginis et martyris - \textbf{memoria maior} - \textit{Rub.}}
\ApplyBody{
\item ad Vigilias pro breviario vetere~: lectio de memoria in supplemento 71.
\item ad Laudes, Vesperas et Horas minores~: antiphonæ propriæ.
\item in MC~: Commune virginis martyris (MR 924)~; lectiones propriæ~: 1 Co \textbf{1}, 26-31 / Mt \textbf{13}, 44-46~; præfatio de sanctis martyribus.
\item ad Magnificat~: ø \textit{Beata Agnes} (AM 785).}
\ApplyHeader{\textbf{22} & Feria III - S. Vincentii, diaconi et martyris - \textit{memoria minor} - \textit{Vir.}}
\ApplyBody{
\item ad Benedictus~: ø \textit{Si mihi} (AM 320).
\item \textit{in ML (Rub.)~: Missa in Supplemento OSB.}
\item in MC \textit{(Rub.)}~: Commune martyrum (MR 915).}
\ApplyHeader{\textbf{23} & Feria IV ł - de ea - \textit{Vir.}}
\ApplyHeader{\textbf{24} & Feria V - \textsc{S. Francisci de Sales}, episcopi et Ecclesiæ doctoris - \textbf{memoria maior} - \textit{Alb.} (olim die 29 huius).}
\ApplyBody{
\item ad Vigilias~: lectio de memoria in supplemento 73.
\item ad Benedictus~: ø \textit{Sapientiam} (AM 981).
\item \textit{in ML~: Missa in PAL.}
\item in MC~: præfatio de sanctis pastoribus.}
\ApplyHeader{\textbf{25} & Feria VI ł - \textbf{\textsc{In Conversione S. Pauli apostoli}} - \textbf{festum} - \textit{Alb.}}
\ApplyBody{
\item ad Vigilias~: in supplemento 74 pro breviario 62~; invitatorium proprium.
\item ad Benedictus~: ø \textit{Vade Anania} (AM 791).
\item \textit{in ML~: non dicitur \emph{Credo}.}
\item in MC~: lectiones propriæ~: Act \textbf{22}, 3-16 / Mc \textbf{16}, 15-18~; præfatio I de Apostolis.}
\ApplyHeader{\textbf{26} & Sabbato  - \textsc{De Beata} - \textit{\textbf{memoria maior}} - \textit{Alb.}}
\ApplyBody{
\item ad Vigilias~: lectio de memoria sabbato 4 (in supplemento 197 pro breviario vetere)~; oratio \textit{Deus, qui salutis æternæ}.
\item ad Laudes et Horas minores~: officium proprium post Nativitatem Domini (AM 716).
\item ad Benedictus~: ø \textit{Mirabile} (AM 717).
\item in MC~: \textit{Sanctæ Mariæ de Nazareth} (CM 8)~; præfatio I de Beata Maria Virgine.
\item I Vesperæ dominicæ sequentis.}
\ApplyHebdoPsalt{- hebdomada I psalterii -}
\ApplyHeader{\textbf{27} & \textbf{\textsc{Dominica III per annum}} (III post Epiphaniam) - de ea - \textit{Vir.}}
\ApplyBody{
\item in MC~: præfatio II de dominicis.}
\ApplyLectHeader{Hebdomada III per Annum~: lectiones de tempore}
\ApplyLectBody{
\item[Dom. C] Neh \textbf{8}, 2-4a 5-6.8-10 / 1 Cor \textbf{12}, 12-30 / Lc \textbf{1}, 1-4~; \textbf{4}, 14-21
\item[Feria II] Hebr \textbf{9}, 15.24-28 / Mc \textbf{3}, 22-30
\item[Feria III] Hebr \textbf{10}, 1-10 / Mc \textbf{3}, 31-35
\item[Feria IV] Hebr \textbf{10}, 11-18 / Mc \textbf{4}, 1-20
\item[Feria V] Hebr \textbf{10}, 19-25 / Mc \textbf{4}, 21-25
\item[Feria VI] Hebr \textbf{10}, 32-39 / Mc \textbf{4}, 26-34
\item[Sabbato] Hebr \textbf{11}, 1-2.8-19 / Mc \textbf{4}, 35-40}
\ApplyPrefaceFeries{
\item in feriis~: præfatio communis III, nisi aliter notetur.}
\ApplyAnniv{Cras recurrit anniversarium approbationis Constitutionum Sancti Ioseph Claræ Vallis ab episcopo Divionensi (1988).}

\ApplyHeader{\textbf{28} & Feria II - \textsc{S. Thomæ de Aquino}, presbyteri et Ecclesiæ doctoris - \textbf{memoria maior} - \textit{Alb.} (olim die 7 martii).}
\ApplyBody{
\item ad Benedictus~: ø \textit{Sapientia clamitat} (AM 581).\item \textit{in ML~: \emph{ß} Alleluia ut in Communi doctorum.}\item in MC~: Commune doctorum Ecclesiæ (MR 943)~; præfatio I de sanctis.}
\ApplyHeader{\textbf{29} & Feria III - de ea - \textit{Vir.}}
\ApplyHeader{\textbf{30} & Feria IV ł - S. Bathildis, monialis - \textit{memoria minor} - \textit{Vir.} }
\ApplyBody{
\item ad Benedictus~: ø \textit{Simile est} (AM 685)~; oratio in supplemento 91.
\item \textit{in ML (Alb.)~: Missa pro nec virgine nec martyre.}
\item in MC \textit{(Alb.)}~: collecta in MP~; Commune sanctorum et sanctarum (MR 960).}
\ApplyHeader{\textbf{31} & Feria V - \textsc{S. Ioannis Bosco}, presbyteri - \textbf{memoria maior} - \textit{Alb.}}
\ApplyBody{
\item in Officio~: oratio in supplemento 92 vel in variationibus 19.
\item ad Vigilias~: lectio de memoria in supplemento 91.
\item ad Benedictus~: ø \textit{Fili, præbe mihi} (AM 566).
\item in MC~: Commune sanctorum et sanctarum (MR 965)~; præfatio II de sanctis.
\item ad Magnificat~: ø \textit{Amen dico vobis} in tono 1 f (AM 829).}

\ApplyNewMonthTitles{Februarius}
\ApplyHeader{\textbf{1} & Feria VI µ £ - de ea - \textit{Vir.}}
\ApplyBody{
\item \textit{in ML (Alb.)~: Missa de sacratissimo Corde Iesu \emph{(Gloria)}.}
\item in MC (Alb.) : Missa votiva de Dei Misericordia (MR 1158) ; præfatio communis II (in tono simplici).}
\ApplyAnniv{Cras celebratur dies ad vitam religiosam atque ad vocationes ecclesiasticas petendas dedicata.}

\ApplyHeader{\textbf{2} & Sabbato § - \textbf{\textsc{In Præsentatione Domini}} (vel Purificationis Beatæ Mariæ Virginis) - \textbf{festum} - \textit{Alb.}}
\ApplyBody{
\item in MC~: ante Missam conventualem peragitur benedictio ac processio candelarum. Post processionem cantantur introitus Missæ et psalmodia Tertiæ~; lectiones propriæ~: Mal \textbf{3}, 1-4 / Lc \textit{2}, 22-40~; præfatio propria.
\item post Completorium dicitur ø \textit{Ave Regina Cælorum}.}
\ApplyHebdoPsalt{- hebdomada II psalterii -}
\ApplyHeader{\textbf{3} & ŧ \textbf{\textsc{Dominica IV per annum}} (IV post Epiphaniam) - de ea - \textit{Vir.}}
\ApplyBody{
\item in MC~: præfatio III de dominicis.}
\ApplyLectHeader{Hebdomada IV per Annum~: lectiones de tempore}
\ApplyLectBody{
\item[Dom. C] Ier \textbf{1}, 4-5.17-19 / 1 Cor \textbf{12}, 31–13, 13 / Lc \textbf{4}, 21 30
\item[Feria II] Hebr \textbf{11}, 32-40 / Mc \textbf{5}, 1-20
\item[Feria III] Hebr \textbf{12}, 1-4 / Mc \textbf{5}, 21-43
\item[Feria IV] Hebr \textbf{12}, 4-7.11-15 / Mc \textbf{6}, 1-6
\item[Feria V] Hebr \textbf{12}, 18-19.21-24 / Mc \textbf{6}, 7-13
\item[Feria VI] Hebr \textbf{13}, 1-8 / Mc \textbf{6}, 14-29
\item[Sabbato] Hebr \textbf{13}, 15-17.20-21 / Mc \textbf{6}, 30-34}
\ApplyPrefaceFeries{
\item in feriis~: præfatio communis IV, nisi aliter notetur.}
\ApplyHeader{\textbf{4} & Feria II - de ea - \textit{Vir.}}
\ApplyBody{
\item in MC (\textit{Nigr.})~: Missa defunctorum pro omnibus benefactoribus nostris defunctis (MR 1225)~; lectiones propriæ~: Rom \textbf{5}, 6b-11 / Mt \textbf{5}, 1-12a~; præfatio II de defunctis.}
\ApplyHeader{\textbf{5} & Feria III - \textsc{S. Agathæ}, virginis et martyris - \textbf{memoria maior} - \textit{Rub.}}
\ApplyBody{
\item ad Vigilias~: lectio de memoria in supplemento 93 pro breviario vetere.
\item ad Laudes, Vesperas et Horas minores~: antiphonæ propriæ.
\item in MC~: Commune virginis martyris (MR 924)~; lectiones propriæ~: 1 Co \textbf{1}, 26-31 / Lc \textbf{9}, 23-26~; præfatio de sanctis martyribus.}
\ApplyHeader{\textbf{6} & Feria IV ł - \textsc{Ss. Pauli Miki et Sociorum}, martyrum - \textbf{memoria maior} - \textit{Rub.}}
\ApplyBody{
\item ad Vigilias~: lectio de memoria in supplemento 94.
\item ad Benedictus~: ø \textit{Beati eritis} (AM 1121).
\item \textit{in ML~: Missa pro pluribus martyribus.}
\item in MC~: Commune martyrum (MR 909)~; præfatio de sanctis martyribus.
\item ad Magnificat~: ø \textit{Gaudent in cælis} (AM 653).}
\ApplyHeader{\textbf{7} & Feria V - de ea - \textit{Vir.}}
\ApplyHeader{\textbf{8} & Feria VI ł - S. Hieronymi Emiliani - \textit{memoria minor} - \textit{Vir.}}
\ApplyBody{
\item ad Benedictus~: ø \textit{O viri misericordiæ} in tono 1 d (AM 973)~; oratio in supplemento 96.
\item \textit{in ML (Alb.)~: olim die 20 iulii.}
\item in MC \textit{(Alb.)}~: Commune sanctorum et sanctarum (MR 965).}
\ApplyHeader{\textbf{9} & Sabbato  - \textsc{De Beata} - \textit{\textbf{memoria maior}} - \textit{Alb.}}
\ApplyBody{
\item ad Vigilias~: lectio de memoria sabbato 2 (in supplemento 198 pro breviario vetere).
\item in MC~: \textit{Sanctæ Mariæ, discipulæ Domini} (CM 10)~; præfatio I de Beata Maria Virgine.
\item I Vesperæ dominicæ sequentis.}
\ApplyAnniv{Cras recurrit anniversarium publicæ renovationis votorum nostrorum (1988).}

\ApplyHebdoPsalt{- hebdomada I psalterii -}
\ApplyHeader{\textbf{10} & \textbf{\textsc{Dominica V per annum}} (V post Epiphaniam) - de ea - \textit{Vir.}}
\ApplyBody{
\item in MC~: præfatio IV de dominicis.}
\ApplyLectHeader{Hebdomada V per Annum~: lectiones de tempore}
\ApplyLectBody{
\item[Dom. C] Is \textbf{6}, 1-2a 3-8 / 1 Cor \textbf{15}, 1-11 / Lc \textbf{5}, 1-11
\item[Feria II] Gen \textbf{1}, 1-19 / Mc \textbf{6}, 53-56
\item[Feria III] Gen \textbf{1}, 20–2, 4a / Mc \textbf{7}, 1-13
\item[Feria IV] Gen \textbf{2}, 4b-9.15-17 / Mc \textbf{7}, 14-23
\item[Feria V] Gen \textbf{2}, 18-25 / Mc \textbf{7}, 24-30
\item[Feria VI] Gen \textbf{3}, 1-8 / Mc \textbf{7}, 31-37
\item[Sabbato] Gen \textbf{3}, 9-24 / Mc \textbf{8}, 1-10}
\ApplyPrefaceFeries{
\item in feriis~: præfatio communis V, nisi aliter notetur.}
\ApplyHeader{\textbf{11} & Feria II - \textsc{Beatæ Mariæ Virginis de Lourdes} - \textbf{\textit{memoria maior}} - \textit{Alb.}}
\ApplyBody{
\item in Officio~: oratio in supplemento 98 vel in variationibus 38.
\item ad Vigilias~: in supplemento 96~; invitatorium in supplemento 58.
\item ad Benedictus~: ø \textit{Præclara} in variationibus 39.
\item in MC~: omnia in MP~; præfatio I de Beata Maria Virgine (\textit{Et te in Conceptione immaculata}).
\item ad Magnificat~: ø \textit{Ave Maria} (AM 862), ß \textit{Immaculata} in variationibus 39.}
\ApplyHeader{\textbf{12} & Feria III - \textsc{S. Benedicti Anianensis}, abbatis - \textbf{\textit{memoria maior}} - \textit{Alb.}}
\ApplyBody{
\item in Officio~: oratio in supplemento 99.
\item ad Vigilias~: lectio de memoria in supplemento 98.
\item \textit{in ML~: Missa pro abbate.}
\item in MC~: collecta in MP~; Commune sanctorum et sanctarum (MR 958)~; præfatio de sanctis pastoribus.}
\ApplyHeader{\textbf{13} & Feria IV ł - de ea - \textit{Vir.}}
\ApplyHeader{\textbf{14} & Feria V - \textsc{\textbf{Ss. Cyrilli, monachi, et Methodii, episcopi}, Europæ patronorum} - \textbf{festum} - \textit{Alb.} (olim die 7 iulii).}
\ApplyBody{
\item ad Vigilias~: de Communi confessoris pontificis, præter lectiones et orationem in supplemento 99.
\item ad Laudes, Vesperas et Horas minores~: ut in variationibus 25 et sequentibus.
\item \textit{in ML~: præfatio de sanctis.}
\item in MC~: lectiones propriæ~: Act \textbf{13}, 46-49 / Lc \textbf{10}, 1-9~; præfatio I de sanctis.}
\ApplyHeader{\textbf{15} & Feria VI ł - de ea - \textit{Vir.}}
\ApplyHeader{\textbf{16} & Sabbato  - \textsc{De Beata} - \textit{\textbf{memoria maior}} - \textit{Alb.}}
\ApplyBody{
\item ad Vigilias~: lectio de memoria sabbato 3 (in supplemento 198 pro breviario vetere).
\item in MC~: \textit{Beatæ Mariæ Virginis, iuxta crucem Domini I} (CM 11)~; præfatio I de Beata Maria Virgine.
\item I Vesperæ dominicæ sequentis. Omnia ut in psalterio, præter ad Magnificat~: ø \textit{Dixit Dominus} (AM 311)~; dicitur \textit{Benedicamus Domino} sine \textit{Alleluia, Alleluia}.}
\ApplyHebdoPsalt{- hebdomada II psalterii -}
\ApplyHeader{\textbf{17} & \textbf{\textsc{Dominica VI per annum}} (Septuagesima) - de ea - \textit{Vir.}}
\ApplyBody{\item in Officio~: omnia dicuntur sicut in psalterio, præter antiphonas ad Benedictus et Magnificat.
\item {\itshape in ML (Viol.)~: non dicitur \emph{Gloria} in Missis de tempore ab hac Dominica usque ad feriam IV Hebdomadæ Sanctæ inclusive.}
\item in MC~: præfatio V de dominicis.}
\newpage
\ApplyGenerSubTitle{In Officio~:}
\ApplyGenerList{
\item in feriis usque ad tempus quadragesimæ~: antiphona ad Benedictus ut in psalterio~; ad Magnificat vero antiphona propria (AM 316 et sequentibus).}
\ApplyGenerSubTitle{in ML~:}
\ApplyGenerList{
\item post graduale dicitur tractus, præterquam in feriis quando resumitur Missa dominicæ.
\item In feriis Septuagesimæ, quando sacerdos celebrat paramentis violaceis, dicere debet vel Missam dominicæ præcedentis vel Missam votivam ex eis quae celebrantur colore violaceo.}
\ApplyLectHeader{Hebdomada VI per Annum~: lectiones de tempore}
\ApplyLectBody{
\item[Dom. C] Ier \textbf{17}, 5-8 / Lc \textbf{6}, 17.20-26
\item[Feria II] Gen \textbf{4}, 1-15.25 / Mc \textbf{8}, 11-13
\item[Feria III] Gen \textbf{6}, 5-8~; \textbf{7}, 1-5.10 / Mc \textbf{8}, 14-21
\item[Feria IV] Gen \textbf{8}, 6-13.20-22 / Mc \textbf{8}, 22-26
\item[Feria V] Gen \textbf{9}, 1-13 / Mc \textbf{8}, 27-33
\item[Feria VI] Gen \textbf{11}, 1-9 / Mc \textbf{8}, 34-39
\item[Sabbato] Hebr \textbf{11}, 1-7 / Mc \textbf{9}, 1-12}
\ApplyPrefaceFeries{
\item in feriis~: præfatio communis VI, nisi aliter notetur.}
\ApplyHeader{\textbf{18} & Feria II - S. Mariæ Bernardæ Soubirous, virginis - \textit{memoria minor} - \textit{Vir.} }
\ApplyBody{
\item ad Benedictus~: ø \textit{Candor} cum suo ß in variationibus 39~; oratio in variationibus 40.
\item \textit{in ML (Alb.)~: Missa pro virgine.}
\item in MC \textit{(Alb.)}~: collecta in MP~; Commune virginum (MR 947).}
\ApplyHeader{\textbf{19} & Feria III - de ea - \textit{Vir.}}
\ApplyBody{
\item \textit{in ML~: Viol.}}
\ApplyHeader{\textbf{20} & Feria IV ł - de ea - \textit{Vir.}}
\ApplyBody{
\item \textit{in ML~: Viol.}}
\ApplyHeader{\textbf{21} & Feria V - \textsc{S. Petri Damiani}, episcopi et Ecclesiæ doctoris - \textbf{memoria maior} - \textit{Alb.} (olim die 23 huius).}
\ApplyBody{
\item ad Vigilias~: lectio in supplemento 104.
\item in MC~: Commune doctorum Ecclesiæ (MR 944)~; præfatio de sanctis pastoribus.}
\ApplyAnniv{Cras recurrit anniversarium fundationis Monasterii nostri (1972).}

\ApplyHeader{\textbf{22} & Feria VI ł - \textbf{\textsc{Cathedræ S. Petri, apostoli}} - \textbf{festum} - \textit{Alb.}}
\ApplyBody{
\item ad Vigilias pro breviario 62~: in supplemento 105~; invitatorium proprium.
\item in MC~: lectiones propriæ~: 1 Petr \textbf{5}, 1-4 / Mt \textbf{16}, 13-19~; præfatio I de Apostolis.
\item ad Magnificat~: ø \textit{Tu es pastor} (AM 823).}
\ApplyHeader{\textbf{23} & Sabbato - \textsc{S. Polycarpi}, episcopi et martyris - \textbf{memoria maior} - \textit{Rub.} (olim die 26 ianuarii).}
\ApplyBody{
\item ad Vigilias~: lectio in supplemento 109.
\item in MC~: Commune martyrum (MR 915)~; præfatio de sanctis martyribus.
\item I Vesperæ dominicæ sequentis ut in psalterio, præter antiphonam ad Magnificat.}
\ApplyHebdoPsalt{- hebdomada I psalterii -}
\ApplyHeader{\textbf{24} & \textbf{\textsc{Dominica VII per annum}} (Sexagesima) - de ea - \textit{Vir.}}
\ApplyBody{\item in Officio~: omnia dicuntur sicut in psalterio, præter antiphonas ad Benedictus et Magnificat.
\item {\itshape in ML~: Viol.}
\item in MC~: præfatio VI de dominicis.
\item ad benedictionem Sanctissimi Sacramenti, post canticum expositionis, cantatur \textit{Ubi Caritas} in Besnier 275 (hebdomada pro christianorum unitate).}
\ApplyLectHeader{Hebdomada VII per Annum~: lectiones de tempore}
\ApplyLectBody{
\item[Dom. C] 1 Sam \textbf{26}, 2.7-9.12-13.22-23 / 1 Cor \textbf{15}, 45-49 / Lc \textbf{6}, 27-38
\item[Feria II] Sir \textbf{1}, 1-10 / Mc \textbf{9}, 13-28
\item[Feria III] Sir \textbf{2}, 1-13 / Mc \textbf{9}, 29-36
\item[Feria IV] Sir \textbf{4}, 12-22 / Mc \textbf{9}, 37-39
\item[Feria V] Sir \textbf{5}, 1-10 / Mc \textbf{9}, 40-49
\item[Feria VI] Sir \textbf{6}, 5-17 / Mc \textbf{10}, 1-12
\item[Sabbato] Sir \textbf{17}, 1-13 / Mc \textbf{10}, 13-16}
\ApplyPrefaceFeries{
\item in feriis~: præfatio communis I, nisi aliter notetur.}
\ApplyHeader{\textbf{25} & Feria II - de ea - \textit{Vir.}}
\ApplyBody{
\item \textit{in ML~: Viol.}}
\ApplyHeader{\textbf{26} & Feria III - de ea - \textit{Vir.}}
\ApplyBody{
\item \textit{in ML~: Viol.}}
\ApplyHeader{\textbf{27} & Feria IV ł - de ea - \textit{Vir.}}
\ApplyBody{
\item \textit{in ML~: Viol.}}
\ApplyHeader{\textbf{28} & Feria V - de ea - \textit{Vir.}}
\ApplyBody{
\item \textit{in ML~: Viol.}}
\ApplyAnniv{Cras recurrit anniversarium fundationis Monasterii Dominæ Nostræ Mayliliensis (1946)~; anniversarium benedictionis Ecclesiæ Sanctissimi Cordis Iesu (1979)~; \textup{†} anniversarium obitus RP Ludovici Mariæ \textsc{Barrielle}, benefactoris (1983)~; \textup{†} anniversarium obitus Eminentissimi Cardinalis Ioannis \textsc{Balland}, benefactoris (1998).}


\ApplyNewMonthTitles{Martius}
\ApplyNewMonthSubTitles{Sancto Joseph consecratus}
\ApplyHeader{\textbf{1} & Feria VI µ £ - de ea - \textit{Vir.}}
\ApplyBody{
\item \textit{in ML (Alb.)~: Missa de sacratissimo Corde Iesu \emph{(Gloria)}.}
\item in MC (Alb.) : Missa votiva de sacratissimo Corde Iesu (MR 492 - GR 660) ; præfatio propria.}
\ApplyHeader{\textbf{2} & Sabbato §  - \textsc{De Beata} - \textit{\textbf{memoria maior}} - \textit{Alb.}}
\ApplyBody{
\item ad Vigilias~: lectio de memoria sabbato 1 (in supplemento 199 pro breviario vetere).
\item ad Benedictus~: ø \textit{Beata es} (AM 1074).
\item \textit{in ML~: Immaculati Cordis Beatæ Mariæ Virginis.}
\item in MC~: \textit{Immaculati Cordis Beatæ Mariæ Virginis} (CM 28)~; præfatio I de Beata Maria Virgine.
\item I Vesperæ dominicæ sequentis ut in psalterio, præter antiphonam ad Magnificat.}
\ApplyHebdoPsalt{- hebdomada II psalterii -}
\ApplyHeader{\textbf{3} & ŧ \textbf{\textsc{Dominica VIII per annum}} (Quinquagesima) - de ea - \textit{Vir.}}
\ApplyBody{\item in Officio~: omnia dicuntur sicut in psalterio, præter antiphonas ad Benedictus et Magnificat.
\item {\itshape in ML~: Viol.}
\item in MC~: præfatio VII de dominicis.}
\ApplyLectHeader{Hebdomada VIII per Annum~: lectiones de tempore}
\ApplyLectBody{
\item[Dom. C] Sir \textbf{27}, 5-8 / 1 Cor \textbf{15}, 54-58 / Lc \textbf{6}, 39-45
\item[Feria II] Sir \textbf{17}, 20-28 / Mc \textbf{10}, 17-27
\item[Feria III] Sir \textbf{35}, 1-15 / Mc \textbf{10}, 28-31}
\ApplyPrefaceFeries{
\item in feriis~: præfatio communis II, nisi aliter notetur.}
\ApplyHeader{\textbf{4} & Feria II - de ea - \textit{Vir.}}
\ApplyBody{
\item \textit{in ML~: Viol.}}
\ApplyHeader{\textbf{5} & Feria III - de ea - \textit{Vir.}}
\ApplyBody{
\item \textit{in ML~: Viol.}}

\ApplyParBox{1cm}{\begin{center}\large{\textit{Post Completorium explicit}}\par\large{\textit{pars prior temporis per annum.}}\end{center}}
\newpage
\ApplyParBox{1cm}{\ApplyGenerTitleHuge{Tempus Quadragesimæ}
\medskip
\ApplyGenerTitleLarge{Usque ad Sabbatum}
\ApplyGenerTitleLarge{Hebdomadæ quartæ}}
\ApplyGenerSubTitle{in Officio~:}
\ApplyGenerList{\item ab initio Vigiliarum Feriæ IV Cinerum sumitur ordinarium officii ferialis temporis Quadragesimæ (in breviario post Dominicam I in Quadragesima vel in AM 336).
\item item omittitur ubique \textit{Alleluia} usque ad Sabbatum Sanctum. Post \textit{Deus in adiutorium}, ubi dicebatur \textit{Alleluia}, dicitur \textit{Laus tibi, Domine, Rex æternæ gloriæ}.
\item in Officio feriali~: \textit{Kyrie}, \textit{Pater} et oratio dicuntur ad omnes Horas flexis genibus.
\item ad Vigilias~: in feriis Quadragesimæ leguntur tres lectiones de homilia cum ¶ n. 3.
\item ad Laudes~: dicuntur cantica ferialia.
\item ad Benedictus et Magnificat~: antiphonæ propriæ.
\item memoriæ obligatoriæ occurentes in feriis Quadragesimæ habendæ sunt uti memoriæ ad libitum. Ad Laudes adduntur post orationem conclusivam, in qua omittitur conclusio \textit{Per Dominum}, antiphona, versiculus et oratio de sancto.}
\ApplyGenerSubTitle{in MC~:}
\ApplyGenerList{\item Missa propria singulis diebus.
\item in feriis Quadragesimæ, occurente memoria alicuius sancti, dicitur collecta de sancto, loco collectæ de feria.
\item tractus dicitur in Feria IV Cinerum, dominicis et sollemnitatibus.
\item in feriis~: \textit{Pater} in tono B~; \textit{Mysterium fidei} et \textit{Per ipsum} in tono simplici.
\item Feria IV Cinerum \& dominicis in Quadragesima~: ad ritum conclusionis missæ, post \textit{Dominus vobiscum}, diaconus cantat \textit{Inclinate vos ad benedictionem}, postquam omnes profunde se inclinant dum sacerdos cantat, manibus extensis super eos, orationem super populum. Hac expleta, omnes surgunt et respondent \textit{Amen}, postquam iterum se inclinant sub benedictionem sacerdotis, more solito, quam sequitur dimissio.}
\ApplyGenerSubTitle{in ML~:}
\ApplyGenerList{\item Missa propria singulis diebus (Missæ votivæ et Missæ defunctorum «cotidianæ» non permittuntur).
\item dicitur præfatio de Quadragesima in Missis de tempore et in ceteris Missis quæ celebrantur eodem tempore et præfatione propria carent.
\medskip
\item cras ieiunium ecclesiasticum cum abstinentia carnium.}
\medskip
\ApplyHeader{\textbf{6} & µ \textsc{Feria IV Cinerum} - \textit{Viol.}}
\ApplyBody{
\item in MC~: post Evangelium, benedictio et impositio cinerum~; Or. n. 8~; præfatio IV de Quadragesima (et sic in feriis sequentibus). Missa concluditur oratione super populum.}
\ApplyLectHeader{Feria IV Cinerum~: lectiones de tempore}
\ApplyLectBody{
\item[Feria IV] Ioel \textbf{2}, 12-18 / 2 Cor \textbf{5}, 20–6, 2 / Mt \textbf{6}, 1-6.16-18
\item[Feria V] Deut \textbf{30}, 15-20 / Lc \textbf{9}, 22-25
\item[Feria VI] Is \textbf{58}, 1-9a / Mt \textbf{9}, 14-15
\item[Sabbato] Is \textbf{58}, 9b-14 / Lc \textbf{5}, 27-32}
\ApplyHeader{\textbf{7} & Feria V - de ea - \textit{Viol.}}
\ApplyBody{
\item ad Laudes~: pro commemoratione Ss. Perpetuæ et Felicitatis, ø \textit{Istarum} (AM 748)~; oratio vide ad diem 6 martii.
\item in MC~: collecta de sanctis.}
\ApplyHeader{\textbf{8} & Feria VI µ - de ea - \textit{Viol.}}
\ApplyBody{
\item \textit{Christifideli, qui orationem \emph{En ego, o bone et dulcissime Iesu} coram Iesu Christi Crucifixi imagine, post communionem, pie recitet, conceditur indulgentia plenaria qualibet feria sexta temporis Quadragesimæ et temporis Passionis \emph{(Enchiridion Indulgentiarum, concessio n. 22)}.}}
\ApplyHeader{\textbf{9} & Sabbato - de eo - \textit{Viol.}}
\ApplyBody{
\item ad Laudes~: pro commemoratione S. Franciscæ Romanæ, ø \textit{Simile est} (AM 685).
\item in MC~: collecta de sancta.\item I Vesperæ dominicæ sequentis.}
\ApplyHebdoPsalt{- hebdomada I psalterii -}
\ApplyHeader{\textbf{10} & \textbf{\textsc{Dominica I in Quadragesima}} - de ea - \textit{Viol.}}
\ApplyBody{
\item in MC~: præfatio propria. In fine celebrationis missæ, sacerdos cantat orationem super populum.}
\ApplyLectHeader{Hebdomada I in Quadragesima~: lectiones de tempore}
\ApplyLectBody{
\item[Dom. C] Deut \textbf{26}, 4-10 / Rom \textbf{10}, 8-13 / Lc \textbf{4}, 1-13
\item[Feria II] Lev \textbf{19}, 1-2.11-18  / Mt \textbf{25}, 31-46
\item[Feria III] Is \textbf{55}, 10-11 / Mt \textbf{6}, 7-15
\item[Feria IV] Ion \textbf{3}, 1-10 / Lc \textbf{11}, 29-32
\item[Feria V] Est \textbf{14}, 1.3-5.12-14 / Mt \textbf{7}, 7-12
\item[Feria VI] Ez \textbf{18}, 21-28 / Mt \textbf{5}, 20-26
\item[Sabbato] Deut \textbf{26}, 16-19 / Mt \textbf{5}, 43-48}
\ApplyPrefaceFeries{
\item in feriis~: præfatio I de Quadragesima, nisi aliter notetur.}
\ApplyHeader{\textbf{11} &Feria II - de ea - \textit{Viol.}}
\ApplyHeader{\textbf{12} &Feria III - de ea - \textit{Viol.}}
\ApplyAnniv{Cras recurrit anniversarium electionis S.S. D.N. Francisci, quem Dominus vivificet et beatum faciat (2013).}

\ApplyHeader{\textbf{13} &Feria IV ł - de ea - \textit{Viol.}}
\ApplyHeader{\textbf{14} &Feria V - de ea - \textit{Viol.}}
\ApplyHeader{\textbf{15} &Feria VI µ - de ea - \textit{Viol.}}
\ApplyBody{\item \textit{indulgentia plenaria pro recitatione orationis \emph{En ego, o bone et dulcissime Iesu}.}}
\ApplyHeader{\textbf{16} &Sabbato - de eo - \textit{Viol.}}
\ApplyBody{
\item I Vesperæ dominicæ sequentis.}
\ApplyHebdoPsalt{- hebdomada II psalterii -}
\ApplyHeader{\textbf{17} & \textbf{\textsc{Dominica II in Quadragesima}} - de ea - \textit{Viol.}}
\ApplyBody{
\item in MC~: præfatio propria. Missa concluditur oratione super populum.}
\ApplyLectHeader{Hebdomada II in Quadragesima~: lectiones de tempore}
\ApplyLectBody{
\item[Dom. C] Gen \textbf{15}, 5-12.17-18 / Phil \textbf{3}, 17–4, 1 / Lc \textbf{9}, 28b-36
\item[Feria II] Dan \textbf{9}, 4b-10 / Lc \textbf{6}, 36-38
\item[Feria III] Is \textbf{1}, 10.16-20 / Mt \textbf{23}, 1-12
\item[Feria IV] Ier \textbf{18}, 18-20 / Mt \textbf{20}, 17-28
\item[Feria V] Ier \textbf{17}, 5-10 / Lc \textbf{16}, 19-31
\item[Feria VI] Gen \textbf{37}, 3-4.12-13a 17b-28 / Mt \textbf{21}, 33-43.45-46
\item[Sabbato] Mic \textbf{7}, 14-15.18-20 / Lc \textbf{15}, 1-3.11-32}
\ApplyPrefaceFeries{
\item in feriis~: præfatio II de Quadragesima, nisi aliter notetur.}
\ApplyHeader{\textbf{18} &Feria II - de ea - \textit{Viol.}}
\ApplyBody{
\item ad Laudes~: pro commemoratione S. Cyrilli Hierosolymitani, ø \textit{Amavit} (AM 663).
\item in MC~: collecta de sancto.\item I Vesperæ sollemnitatis sequentis.}
\ApplyHeader{\textbf{19} &Feria III - ¬ \MakeUppercase{\textbf{S. Ioseph, sponsi Beatæ Mariæ Virginis}}, \textsc{Ecclesiæ universæ et huius monasterii patroni} - \textbf{sollemnitas maior} - \textit{Alb.}}
\ApplyBody{
\item in MC~: lectiones propriæ~: 2 Sam \textbf{7}, 4-5a.12-14a.16 / Rom \textbf{4}, 13.16-18.22 / Mt \textbf{1}, 16.18-21.24a~; præfatio propria (\textit{Et te in sollemnitate}).
\item Vesperæ sollemnitatis~; benedictio Sanctissimi Sacramenti.}
\ApplyHeader{\textbf{20} &Feria IV ł - de ea - \textit{Viol.}}
\ApplyHeader{\textbf{21} &Feria V - \textsc{\textbf{Transitus S.P.N. Benedicti}, abbatis} - \textbf{festum} - \textit{Alb.}}
\ApplyBody{
\item \textit{hodie in Ecclesiis ordinis nostri, indulgentia plenaria acquiri una potest pia ecclesiæ visitatione, in qua recitatur oratio dominica et fidei symbolum.}
\item ad Vigilias~: antiphonæ et psalmi hebdomadæ I.
\item \textit{in ML~: Missa et præfatio propria in supplemento OSB~; sequentia~; non dicitur \emph{Credo}.}
\item in MC~: omnia in MP~; lectiones propriæ~: Gen \textbf{12}, 1-4a / Io \textbf{17}, 20-26~; præfatio propria.
\item ad Vesperas~: ¶ breve, tono simplici (AM 961).}
\ApplyHeader{\textbf{22} &Feria VI µ - de ea - \textit{Viol.}}
\ApplyBody{\item \textit{indulgentia plenaria pro recitatione orationis \emph{En ego, o bone et dulcissime Iesu}.}}
\ApplyHeader{\textbf{23} &Sabbato - de eo - \textit{Viol.}}
\ApplyBody{
\item I Vesperæ dominicæ sequentis.}
\ApplyHebdoPsalt{- hebdomada I psalterii -}
\ApplyHeader{\textbf{24} & \textbf{\textsc{Dominica III in Quadragesima}} - de ea - \textit{Viol.}}
\ApplyBody{
\item in MC~: tres lectiones Anno A~; præfatio propria. Missa concluditur oratione super populum.}
\ApplyLectHeader{Hebdomada III in Quadragesima~: lectiones de tempore}
\ApplyLectBody{
\item[Dom. A] Ex \textbf{17}, 3-7 / Rom \textbf{5}, 1-2.5-8 / Io \textbf{4}, 5-42
\item[Dom. C] Ex \textbf{3}, 1-8a 13-15 / 1 Cor \textbf{10}, 1-6.10-12 / Lc \textbf{13}, 1-9
\item[Feria II] 2 Reg \textbf{5}, 1-15a / Lc \textbf{4}, 24-30
\item[Feria III] Dan \textbf{3}, 25.34-43 / Mt \textbf{18}, 21-35
\item[Feria IV] Deut \textbf{4}, 1.5-9 / Mt \textbf{5}, 17-19
\item[Feria V] Ier \textbf{7}, 23-28 / Lc \textbf{11}, 14-23
\item[Feria VI] Os \textbf{14}, 2-10 / Mc \textbf{12}, 28b-34
\item[Sabbato] Os \textbf{6}, 1b-6 / Lc \textbf{18}, 9-14}
\ApplyPrefaceFeries{
\item in feriis~: præfatio III de Quadragesima, nisi aliter notetur.}
\ApplyHeader{\textbf{25} &Feria II - þ \textbf{\MakeUppercase{In annuntiatione Domini}} - \textbf{sollemnitas minor} - \textit{Alb.}}
\ApplyBody{
\item in MC~: lectiones propriæ~: Is \textbf{7}, 10-14~; \textbf{8}, 10 / Hebr \textbf{10}, 4-10 / Lc \textbf{1}, 26-38~; ad verba symboli \textit{Et incarnatus est} omnes genua flectunt~; præfatio propria.
\item Vesperæ sollemnitatis.}
\ApplyHeader{\textbf{26} &Feria III - de ea - \textit{Viol.}}
\ApplyHeader{\textbf{27} &Feria IV ł - de ea - \textit{Viol.}}
\ApplyHeader{\textbf{28} &Feria V - de ea - \textit{Viol.}}
\ApplyHeader{\textbf{29} &Feria VI µ - de ea - \textit{Viol.}}
\ApplyBody{\item \textit{indulgentia plenaria pro recitatione orationis \emph{En ego, o bone et dulcissime Iesu}.}}
\ApplyHeader{\textbf{30} &Sabbato - de eo - \textit{Viol.}}
\ApplyBody{
\item I Vesperæ dominicæ sequentis.}
\ApplyAnniv{\textup{†} Cras recurrit anniversarium obitus Reverendissimi ac dilectissimi Patris Emmanuel Mariæ \textsc{André}, abbatis Dominæ Nostræ Sanctæ Spei Mesnili, qui die 31 martii 1903 obdormivit in Domino.}

\ApplyHebdoPsalt{- hebdomada II psalterii -}
\ApplyHeader{\textbf{31} & \textbf{\textsc{Dominica IV in Quadragesima}, Lætare} - de ea - \textit{Viol.}}
\ApplyBody{
\item hodie pulsantur organa ad Missam, non vero in aliis Horis.
\item in MC~: tres lectiones Anno A~; præfatio propria. Missa concluditur oratione super populum.}
\ApplyLectHeader{Hebdomada IV in Quadragesima~: lectiones de tempore}
\ApplyLectBody{
\item[Dom. A] 1 Sam \textbf{16}, 1b 6-7.10-13a / Eph \textbf{5}, 8-14 / Io \textbf{9}, 1-41
\item[Dom. C] Ios \textbf{5}, 9a 10-12 / 2 Cor \textbf{5}, 17-21 / Lc \textbf{15}, 1-3.11-32
\item[Feria II] Is \textbf{65}, 17-21 / Io \textbf{4}, 43-54
\item[Feria III] Ez \textbf{47}, 1-9.12 / Io \textbf{5}, 1-3a 5-16
\item[Feria IV] Is \textbf{49}, 8-15 / Io \textbf{5}, 17-30
\item[Feria V] Ex \textbf{32}, 7-14 / Io \textbf{5}, 31-47
\item[Feria VI] Sap \textbf{2}, 1a.12-22 / Io \textbf{7}, 1-2.10.25-30
\item[Sabbato] Ier \textbf{11}, 18-20 / Io \textbf{7}, 40-53}
\ApplyPrefaceFeries{
\item in feriis~: præfatio IV de Quadragesima, nisi aliter notetur.}

\ApplyNewMonthTitles{Aprilis}
\ApplyHeader{\textbf{1} &Feria II - de ea - \textit{Viol.}}
\ApplyHeader{\textbf{2} &Feria III - de ea - \textit{Viol.}}
\ApplyHeader{\textbf{3} &Feria IV ł - de ea - \textit{Viol.}}
\ApplyHeader{\textbf{4} &Feria V - de ea - \textit{Viol.}}
\ApplyHeader{\textbf{5} &Feria VI µ £ - de ea - \textit{Viol.}}
\ApplyBody{\item \textit{indulgentia plenaria pro recitatione orationis \emph{En ego, o bone et dulcissime Iesu}.}}
\ApplyHeader{\textbf{6} &Sabbato § - de eo - \textit{Viol.}}
\ApplyBody{
\item I Vesperæ dominicæ sequentis.}

\medskip
\ApplyGenerList{
\item a I Vesperis dominicæ V in Quadragesima sumitur ordinarium Officii temporis Passionis (AM 382).
\item ß \textit{Gloria Patri} dicitur de more (pro cantu huius versiculi in responsorio brevi, vide in AM 1044 vel 336).
\item ad Vigilias~: invitatorium \textit{Hodie si vocem}~; quarta stropha psalmi 94 incipit a verbis \textit{Sicut in exacerbatione}.
\item cruces et imagines per ecclesiam cooperiuntur.
\item in ML de tempore~: non dicitur psalmus \textit{Iudica} ante confessionem, neque \textit{Gloria} ad introitum et post psalmum \textit{Lavabo}.}
\ApplyHebdoPsalt{- hebdomada I psalterii -}
\ApplyHeader{\textbf{7} & ŧ \textbf{\textsc{Dominica V in Quadragesima}} (I Passionis) - de ea - \textit{Viol.}}
\ApplyBody{\item in MC~: tres lectiones Anno A~; præfatio propria. Missa concluditur oratione super populum.}
\ApplyLectHeader{Hebdomada V in Quadragesima~: lectiones de tempore}
\ApplyLectBody{
\item[Dom. A] Ez \textbf{37}, 12-14 / Rom \textbf{8}, 8-11 / Io \textbf{11}, 1-45
\item[Dom. C] Is \textbf{43}, 16-21 / Phil \textbf{3}, 8-14 / Io \textbf{8}, 1-11
\item[Feria II] Dan \textbf{13}, 41c-62 / Io \textbf{8}, 12-20
\item[Feria III] Num \textbf{21}, 4-9 / Io \textbf{8}, 21-30
\item[Feria IV] Dan \textbf{3}, 14-20.91-92.95 / Io \textbf{8}, 31-42
\item[Feria V] Gen \textbf{17}, 3-9 / Io \textbf{8}, 51-59
\item[Feria VI] Ier \textbf{20}, 10-13 / Io \textbf{10}, 31-42
\item[Sabbato] Ez \textbf{37}, 21-28 / Io \textbf{11}, 45-56}
\ApplyPrefaceFeries{
\item in feriis~: præfatio I de Passione Domini, nisi aliter notetur.}
\ApplyHeader{\textbf{8} & Feria II - de ea - \textit{Viol.}}
\ApplyHeader{\textbf{9} & Feria III - de ea - \textit{Viol.}}
\ApplyHeader{\textbf{10} & Feria IV ł - de ea - \textit{Viol.}}
\ApplyHeader{\textbf{11} & Feria V - de ea - \textit{Viol.}}
\ApplyHeader{\textbf{12} & Feria VI µ - de ea - \textit{Viol.}}
\ApplyBody{\item \textit{indulgentia plenaria pro recitatione orationis \emph{En ego, o bone et dulcissime Iesu}.}}
\ApplyHeader{\textbf{13} & Sabbato - S. Martini I, papæ et martyris - \textit{memoria minor} - \textit{Alb.} (olim die 12 novembris).}
\ApplyBody{\item ad Benedictus~: ø \textit{Qui odit} (AM 642)~; oratio in supplemento 115.\item \textit{in ML~: Rub.}
\item in MC \textit{(Rub.)}~: Commune pastorum (MR 927).\item I Vesperæ dominicæ sequentis.}

\newpage
\ApplyParBox{1cm}{\ApplyGenerTitleHuge{Hebdomada Sancta}}
\ApplyHebdoPsalt{- hebdomada II psalterii -}
\ApplyHeader{\textbf{14} & \textbf{\textsc{Dominica in Palmis de Passione Domini}} - de ea - \textit{Rub.}}
\ApplyBody{\item \textit{in ML (Viol)~: ante lectionem historiæ Passionis omittuntur \emph{Dominus vobiscum} et cætera (uti indicatur in Missali)~; post lectionem non respondetur \emph{Laus tibi Christe} et celebrans non osculatur librum (quod servatur etiam feria III et IV)~; legitur in fine (loco Evangelii sancti Ioannis) Evangelium ut in benedictione ramorum.}
\item in MC~: ante Missam, benedictio et processio palmarum~; omittitur antiphona ad introitum~; circa finem historiæ Passionis, genuflectitur et pausatur aliquantulum~; præfatio propria. Missa concluditur oratione super populum.}
\ApplyLectHeader{Dominica in Palmis de Passione Domini~: lectiones}
\ApplyLectBody{
\item[Dom. C] \textit{Ante processionem~:} Lc \textbf{19}, 28-40
\item[Dom. C] Is \textbf{50}, 4-7 / Phil \textbf{2}, 6-11 / Lc \textbf{22}, 14–\textbf{23}, 56
\item[Feria II] Is \textbf{42}, 1-7 / Io \textbf{12}, 1-11
\item[Feria III] Is \textbf{49}, 1-6 / Io \textbf{13}, 21-33.36-38
\item[Feria IV] Is \textbf{50}, 4-9a / Mt \textbf{26}, 14-25}
\ApplyPrefaceFeries{
\item in feriis~: præfatio II de Passione Domini, nisi aliter notetur.}
\ApplyHeader{\textbf{15} & \textbf{\textsc{Feria II Hebdomadæ Sanctæ}} - de ea - \textit{Viol.}}
\ApplyBody{\item ad Laudes et Horas minores~: antiphonæ propriæ.}
\ApplyHeader{\textbf{16} & \textbf{\textsc{Feria III Hebdomadæ Sanctæ}} - de ea - \textit{Viol.}}
\ApplyBody{\item ad Laudes et Horas minores~: antiphonæ propriæ.}
\ApplyHeader{\textbf{17} & ł \textbf{\textsc{Feria IV Hebdomadæ Sanctæ}} - de ea - \textit{Viol.}}
\ApplyBody{\item ad Laudes et Horas minores~: antiphonæ propriæ.}
\ApplyHeader{\textbf{18} & \textbf{\textsc{Feria V Hebdomadæ Sanctæ}} - de ea - \textit{Viol.}}
\ApplyBody{
\item a Vigiliis usque ad Nonam~: omnia ut in feriis præcedentibus.
\item ad Vigilias~: in I nocturno, lectiones de eadem feria V e II nocturno cum ¶ de III nocturno~; in II nocturno, ut in ordinario officii ferialis temporis Passionis.
\item ad Laudes et Horas minores~: antiphonæ de feria V in Cena Domini cum psalmis de feria V in psalterio~; reliqua ut in ordinario temporis Passionis.}
\ApplyParBox{1cm}{\begin{center}\large{\textit{Cum Missa Vespertina in Cena Domini}}\par\large{\textit{incipit Sacrum Triduum Paschale.}}\end{center}}
\vspace{6cm}
\ApplyParBox{2cm}{\ApplyGenerTitleHuge{Sacrum Triduum Paschale}
\ApplyGenerList{
\item ß \textit{Gloria Patri} dicitur de more.
\item Feria VI in Passione Domini et Sabbato Sancto~: post Sextam, commemoratio pro defunctis omittitur.}
\ApplyLectHeader{Sacrum Triduum Paschale~: lectiones}
\ApplyLectHeader{Feria V in Cena Domini in Missa}
\ApplyLectTriduum{Ex \textbf{12}, 1-8.11-14 / 1 Cor \textbf{11}, 23-26 / Io \textbf{13}, 1-15}
\ApplyLectHeader{Feria VI in Passione Domini in officio Passionis}
\ApplyLectTriduum{Is \textbf{52}, 13–12, 1-12 / Heb \textbf{4}, 14-16~; \textbf{5}, 7-9 / Io \textbf{18}, 1–19, 1-42}
\ApplyLectHeader{Vigilia Paschalis in celebratione Vigiliæ}
\ApplyLectTriduum{(1L) Gen \textbf{1}–\textbf{2}, 1-2 / (2L) Gen \textbf{22}, 1-18 / (3L) Ex \textbf{14}, 15–15, 1a / (4L) Is \textbf{54}, 5-14 / (5L) Is \textbf{55}, 1-11 / (6L) Bar \textbf{3}, 9-15.32–4, 1-4 / (7L) Ez \textbf{36}, 16-17a.18-28 / (8L) Rom \textbf{6}, 3-11 / (Ev) Lc \textbf{24}, 1-12}
\ApplyLectHeader{In Resurrectione Domini Nostri in Missa}
\ApplyLectTriduum{Act \textbf{10}, 34a 37-43 / Col \textbf{3}, 1-4 / Io \textbf{20}, 1-9}}
\newpage
\ApplyHeader{\textbf{18} & \textbf{\MakeUppercase{Missa in Cena Domini}} - Alb.}{
\ApplyBody{
\item in Missa~: \textit{Gloria} (pulsantur campanæ)~; post Evangelium fit lotio pedum quæ concluditur oratione fidelium n. 10~; pro offertorio cantatur \textit{Ubi Caritas} (in GR 168)~; præfatio I de Sanctissima Eucharistia~; \textit{Communicantes}, \textit{Hanc igitur} et \textit{Qui pridie} propria~; in fine Missæ, omittuntur ritus conclusionis~; post missam, sollemnis translatio ac repositio Sanctissimi Sacramenti~: cantatur hymnus \textit{Pange Lingua} (GR 170), cum \textit{Tantum ergo}~: indulgentia plenaria.
\item Vesperæ omittuntur ab his qui Missam vespertinam intersunt.
\item expleta celebratione, denudatur altare et aufertur crux ab ecclesia.
\item ad cenam~: ß \textit{Christus factus est}.
\item ad Completorium~: superior, signo dato, indicat initiandum esse examen conscientiæ, quo absoluto dicitur \textit{Confiteor}~; oratio \textit{Visita} quæ concluditur sub silentio. Hodie et cras omittitur aspersio.}

\ApplyGenerList{\item cras ieiunium ecclesiasticum cum abstinentia carnium.}
\ApplyAnniv{\textit{Cras incipiunt preces novendiales ad misericordiam divinam.}}

\ApplyHeader{\textbf{19} & µ \textbf{\MakeUppercase{Feria VI In Passione Domini}} - \textit{Rub}.}
\ApplyBody{
\item \textit{indulgentia plenaria pro recitatione orationis \emph{En ego, o bone et dulcissime Iesu}.}
\item in Officio~: altare omnino nudum sit, sine cruce et sine candelabris. In Vigiliis et Laudibus, candelabrum triangulare cum quindecim cereis adhibetur. Non pulsantur organa ad cantum sustentandum nec in Officio nec in celebratione Passionis.
\item ad mensam~: ß \textit{Christus factus est}.
\item celebratio Passionis Domini incipienda est circa horam nonam diei (Adoratio Sanctæ Crucis~: indulgentia plenaria)~; expleta celebratione denudatur altare, relictis tamen cruce et candelabris.
\item hodie post Crucis detectionem et cras usque ad Vigiliam paschalem exclusive, cruci genuflectitur.
\item Vesperæ omittuntur ab his qui sollemnem actionem liturgicam intersunt.
\item ad Completorium~: ut heri, sed ante Officium cantatur hymnus \textit{Stabat Mater} (Besnier 174).}
\ApplyHeader{\textbf{20} & ł \textbf{\MakeUppercase{Sabbato Sancto}} - \textit{Viol.}}
\ApplyBody{
\item in Officio~: altare nudum sit, cum cruce tamen et quatuor candelabris accensis. In Vigiliis et Laudibus, candelabrum triangulare cum quindecim cereis adhibetur. Non pulsantur organa usque ad hymnum \textit{Gloria} in Missa Vigiliæ paschalis.
\item ad mensam~: ß \textit{Principes}.
\item Vesperæ celebrantur hora consueta.
\item Completorium necnon et Vigiliæ omittuntur ab his qui Vigiliam paschalem intersunt.}
\bigskip
\ApplyHeader{ &\MakeUppercase{\textbf{Vigilia paschalis}} - \textit{Alb.}}
\ApplyBody{
\item celebratio huius Vigiliæ peragenda est nocte.
\item in Missa~: ad hymnum \textit{Gloria} pulsantur campanæ~; post Evangelium, renovatio promissionum baptismalium~: indulgentia plenaria~; præfatio paschalis I (\textit{in hac potissimum nocte})~; \textit{Communicantes} et \textit{Hanc igitur} propria~; \textit{Ite Missa est} cum duobus \textit{Alleluia}.}

\medskip
\ApplyGenerSubTitle{a dominica Paschæ usque ad Pentecosten}
\ApplyGenerList{
\item loco \textit{Angelus Domini} dicitur \textit{Regina Cæli} stando.
\item ad aspersionem aquæ benedictæ, dominica~: ø \textit{Vidi aquam}.
\item post prandium~: loco psalmi \textit{Miserere mei} dicitur psalmus \textit{Confitemini}.}
\ApplyGenerSubTitle{ad mensam~:}
\ApplyGenerList{
\item ad benedictionem, dicitur ß \textit{Hæc dies} (usque ad dominicam sequentem inclusive).}
\ApplyHebdoPsalt{- hebdomada I psalterii -}
\ApplyHeader{\textbf{21} & \textbf{\MakeUppercase{Dominica Paschæ in Resurrectione Domini}} - \textbf{sollemnitas maior cum octava} - \textit{Alb}.}
\ApplyBody{
\item ad Laudes~: Officium dominicæ Resurrectionis incipitur ab invitatorio Paschæ~: \textit{Domine labia mea aperies}, et statim psalmus 66 \textit{Deus misereatur nostri}, cum sua antiphona \textit{Surrexit Dominus vere alleluia}, quod sequuntur Laudes~; \textit{Benedicamus Domino} cum duobus \textit{Alleluia}.
\item in MC~: sequentia~; præfatio paschalis I (\textit{in hac potissimum die})~; \textit{Communicantes} et \textit{Hanc igitur} propria~; \textit{Ite Missa est} cum duobus \textit{Alleluia}.
\item Vesperæ sollemnitatis~; \textit{Benedicamus Domino} cum duobus \textit{Alleluia}~; benedictio Sanctissimi Sacramenti.
\item ad Completorium~: ø \textit{Regina Cæli}.}
\newpage
\ApplyGenerTitleHuge{Tempus paschale I}
\smallskip
\ApplyGenerTitleLarge{Usque ad Ascensionem Domini}
\ApplyGenerSubTitle{in Officio~:}
\ApplyGenerList{
\item ordinarium invenitur in breviario vel in AM 466.
\item additur unus \textit{Alleluia} ad invitatorium et versiculum ubi iam non habetur.
\item in Officio feriali~: ad Benedictus et Magnificat antiphonæ propriæ.
\item ad Laudes~: pro antiphona ad Benedictus sabbato, sumitur antiphona \textit{Regina cæli} (AM 718), nisi aliter notetur.
\item ad Horas minores et Completorium~: tonus paschalis etiam in memoriis et festis quæ non sunt de Beata Maria Virgine.}
\ApplyGenerSubTitle{in ML~:}
\ApplyGenerList{
\item in omnibus Missis, loco gradualis, dicuntur quatuor Alleluia cum ß tempori Paschali assignatis, adduntur duo \textit{Alleluia} ad introitum, unus ad offertorium et communionem ubi iam non adsunt.
\item præfatio paschalis dicitur etiam in ceteris Missis quæ præfatione propria carent.
\item infra hebdomadam~: quando resumitur Missa dominicæ præcedentis dicuntur \textit{Gloria} et præfatio paschalis.}
\ApplyGenerSubTitle{in MC~:}
\ApplyGenerList{
\item in feriis~: orationes propriæ.\par}
\ApplyGenerSubTitle{infra octavam~:}
\ApplyGenerList{
\item ad Laudes, Vesperas et Horas minores~: omnia ut in die Paschæ, præter antiphonas ad Benedictus, Magnificat et orationem.
\item in MC~: \textit{Gloria}~; quando cantatur \textit{Alleluia}, dicitur sequentia \textit{Victimæ paschali}~; præfatio paschalis I (\textit{in hac potissimum die})~; \textit{Communicantes} et \textit{Hanc igitur} propria~; \textit{Ite Missa est} cum duobus \textit{Alleluia}.}
\ApplyLectHeader{Infra octavam Paschæ~: lectiones}
\ApplyLectBody{
\item[Feria II] Act \textbf{2}, 14.22-32 / Mt \textbf{28}, 8-15
\item[Feria III] Act \textbf{2}, 36-41 / Io \textbf{20}, 11-18
\item[Feria IV] Act \textbf{3}, 1-10 / Lc \textbf{24}, 13-35
\item[Feria V] Act \textbf{3}, 11-26 / Lc \textbf{24}, 35-48
\item[Feria VI] Act \textbf{4}, 1-12 / Io \textbf{21}, 1-14
\item[Sabbato] Act \textbf{4}, 13-21 / Mc \textbf{16}, 9-15}
\ApplyHeader{\textbf{22} &Feria II - \textbf{\textsc{De Octava Paschæ}} - \textit{Alb}.}
\ApplyBody{
\item ad Vigilias~: psalmi de festo hebdomada I~; pro breviario vetere in supplemento 44.
\item \textit{postmeridiani laboris datur vacatio}.}
\ApplyHeader{\textbf{23} &Feria III - \textbf{\textsc{De Octava Paschæ}} - \textit{Alb}.}
\ApplyBody{
\item ad Vigilias~: psalmi de festo hebdomada II~; pro breviario vetere in supplemento 45.}
\ApplyHeader{\textbf{24} &Feria IV ł - \textbf{\textsc{De Octava Paschæ}} - \textit{Alb}.}
\ApplyBody{
\item ad Vigilias~: psalmi de feria cum antiphonis propriis et sic usque ad sabbatum in Albis.}
\ApplyHeader{\textbf{25} &Feria V - \textbf{\textsc{De Octava Paschæ}} - \textit{Alb}.}
\ApplyHeader{\textbf{26} &Feria VI ł - \textbf{\textsc{De Octava Paschæ}} - \textit{Alb}.}
\ApplyHeader{\textbf{27} &Sabbato - \textbf{\textsc{De Octava Paschæ}} (Sabbato in Albis) - \textit{Alb.}}
\ApplyBody{

\item ad Vesperas~: antiphonæ et psalmi ut in die Paschæ~; capitulum de dominica \textit{in Albis}~; ß \textit{Hæc dies quam fecit Dominus}~; ad Magnificat~: ø \textit{Cum esset sero} (AM 475)~; ß \textit{Benedicamus Domino} cum duobus \textit{Alleluia}.}
\ApplyAnniv{Indulgentia plenaria conceditur, suetis sub condicionibus (nempe Sacramentali Confessione, Eucharistica Communione et Oratione ad mentem Summi Pontificis) christifideli, qui, die Dominica II Paschæ seu de Divina Misericordia, in quacumque ecclesia vel oratorio, animo quidem omnino elongato ab affectu erga quodcumque peccatum, etiam veniale, pium exercitium in honorem Divinæ Misericordiæ participaverit, vel saltem coram SS.mo Eucharistiæ Sacramento, publice exposito vel etiam in tabernaculo adservato, Orationem Dominicam ac Symbolum Fidei recitaverit, addita pia aliqua invocatione ad Misericordem Iesum (e.g. “Misericors Iesu in Te confido”).}

\ApplyHebdoPsalt{- hebdomada II psalterii -}
\ApplyHeader{\textbf{28} & \textbf{\textsc{Dominica II Paschæ}} seu \textbf{\textsc{De Divina Misericordia}} (Dominica in Albis) - de ea - \textit{Alb.}}
\ApplyBody{
\item Officium huius diei persolvitur ut in die Paschæ præter ea quæ sequuntur.
\item ad Vigilias~: lectiones, ¶ et oratio e dominica \textit{in Albis}~; \textit{Benedicamus Domino} de tempore paschali.
\item ad Laudes, Horas et Vesperas~: capitula et oratio e dominica \textit{in Albis}~; ß \textit{Hæc dies quam fecit Dominus}.
\item ad Benedictus~: ø \textit{Cum esset sero} (AM 475)~; ß \textit{Benedicamus Domino} cum duobus \textit{Alleluia}.
\item in MC~: sequentia~; præfatio paschalis I (\textit{in hac potissimum die})~; \textit{Communicantes} et \textit{Hanc igitur} propria~; \textit{Ite Missa est} cum duobus \textit{Alleluia}.
\item Vesperæ dominicæ ut in die Paschæ~: ß \textit{Hæc dies quam fecit Dominus}~; ad Magnificat~: ø \textit{Post dies octo} (AM 477)~; ß \textit{Benedicamus Domino} cum duobus \textit{Alleluia}.}
\ApplyLectHeader{Hebdomada II Paschæ~: lectiones de tempore}
\ApplyLectBody{
\item[Dom. C] Act \textbf{5}, 12-16 / Ap \textbf{1}, 9-11a.12-13.17-19 / Io \textbf{20}, 19-31
\item[Feria II] Act \textbf{4}, 23-31 / Io \textbf{3}, 1-8
\item[Feria III] Act \textbf{4}, 32-37 / Io \textbf{3}, 7-15
\item[Feria IV] Act \textbf{5}, 17-26 / Io \textbf{3}, 16-21
\item[Feria V] Act \textbf{5}, 27-33 / Io \textbf{3}, 31-36
\item[Feria VI] Act \textbf{5}, 34-42 / Io \textbf{6}, 1-15
\item[Sabbato] Act \textbf{6}, 1-7 / Io \textbf{6}, 16-21}
\ApplyPrefaceFeries{
\item ad Vigilias~: in Officio feriali, lectio brevis \textit{De Osea}, vel in memoriis, lectio unica de sancto, et sic usque ad dominicam I novembris.
\item in feriis~: præfatio paschalis II, nisi aliter notetur.}
\ApplyHeader{\textbf{29} &Feria II - \textbf{\textsc{S. Catharinæ Senensis, Europæ Patronæ, virginis et Ecclesiæ doctoris}} - \textbf{festum} - \textit{Alb.} (olim die 30 huius).}
\ApplyBody{
\item in Officio~: omnia de Communi virginis, præter sequentia.
\item oratio in supplemento 118*.
\item ad Vigilias~: lectiones II nocturni in supplemento 117*.
\item ad Benedictus~: ø \textit{O beata anima} cum \textit{alleluia} (AM 1131).
\item \textit{in ML~: præfatio de sanctis.}
\item in MC~: lectiones propriæ~: 1 Io \textbf{1}, 5 – \textbf{2}, 2 / Mt \textbf{11}, 25-30~; præfatio de sanctis virginibus et religiosis.
\item ad Magnificat~: ø \textit{O dignissima} cum \textit{alleluia} (AM 1133).}
\ApplyHeader{\textbf{30} &Feria III - S. Pii V, Papæ - \textit{memoria minor} - \textit{Alb.} (olim die 5 maii).}
\ApplyBody{
\item ad Benedictus~: ø \textit{Dum esset} (AM 663).
\item in MC~: Commune pastorum (MR 927).}

\ApplyNewMonthTitles{Maius}
\ApplyNewMonthSubTitles{Beatæ Mariæ Virgini consecratus}
\ApplyHeader{\textbf{1} &Feria IV ł - \textsc{S. Ioseph Opificis} - \textbf{memoria maior} - \textit{Alb.}}
\ApplyBody{
\item in Officio~: oratio \textit{Rerum conditor} in supplemento 118* vel in breviario 62 (Officium schema I).
\item ad Vigilias~: Officium schema I, præter sequentia~: antiphonæ, psalmi et ß de feria~; lectio in supplemento 118~; in II nocturno lectio brevis et ß ut ad Sextam (in breviario 62~: Officium schema I).
\item ad Laudes, Vesperas et Horas minores~: omnia ut in AM 886, præter orationem ut supra.
\item in MC~: lectiones propriæ~: Gen \textbf{1}, 26–\textbf{2}, 3/ Mt \textbf{13}, 54-58~; præfatio propria \textit{(Et te in commemoratione)}.}
\ApplyHeader{\textbf{2} &Feria V - \textsc{S. Athanasii}, episcopi et Ecclesiæ doctoris - \textbf{memoria maior} - \textit{Alb.}}
\ApplyBody{
\item ad Benedictus~: ø \textit{Unus est} cum \textit{alleluia} (AM 356).
\item in MC~: præfatio de sanctis pastoribus.}
\ApplyHeader{\textbf{3} &Feria VI ł £ - \textbf{\textsc{Ss. Philippi et Iacobi, apostolorum}} - \textbf{festum} - \textit{Rub.} (olim die 1 maii [AM] et 11 maii [ML]).}
\ApplyBody{
\item in MC~: lectiones propriæ~: 1 Cor \textbf{15}, 1-8 / Io \textbf{14}, 6-14~; præfatio I de Apostolis.
\item ad Magnificat~: ø \textit{Non turbetur} (AM 892).}
\ApplyHeader{\textbf{4} &Sabbato § - de eo - \textit{Alb}.}
\ApplyBody{
\item \textit{in ML~: Immaculati Cordis Beatæ Mariæ Virginis.}
\item I Vesperæ dominicæ sequentis.}
\ApplyHebdoPsalt{- hebdomada I psalterii -}
\ApplyHeader{\textbf{5} & ŧ \textbf{\textsc{Dominica III Paschæ}} (seu Dominica II post Pascha) - de ea - \textit{Alb.}}
\ApplyBody{
\item ad Vigilias~: in nocturno I~: lectiones 3 et 4 cum ¶ lectionis 4.
\item in MC~: præfatio pascalis III.}
\ApplyLectHeader{Hebdomada III Paschæ~: lectiones de tempore}
\ApplyLectBody{
\item[Dom. C] Act \textbf{5}, 27b-32.40b-41 / Ap \textbf{5}, 11-14 / Io \textbf{21}, 1-19
\item[Feria II] Act \textbf{6}, 8-15 / Io \textbf{6}, 22-29
\item[Feria III] Act \textbf{7}, 51-59 / Io \textbf{6}, 30-35
\item[Feria IV] Act \textbf{8}, 1-8 / Io \textbf{6}, 35-40
\item[Feria V] Act \textbf{8}, 26-40 / Io \textbf{6}, 44-52
\item[Feria VI] Act \textbf{9}, 1-20 / Io \textbf{6}, 53-60
\item[Sabbato] Act \textbf{9}, 31-42 / Io \textbf{6}, 61-70}
\ApplyPrefaceFeries{
\item in feriis~: præfatio paschalis III, nisi aliter notetur.}
\ApplyHeader{\textbf{6} &Feria II - de ea - \textit{Alb}.}
\ApplyHeader{\textbf{7} &Feria III - de ea - \textit{Alb}.}
\ApplyHeader{\textbf{8} &Feria IV ł - de ea - \textit{Alb}.}
\ApplyHeader{\textbf{9} &Feria V - de ea - \textit{Alb}.}
\ApplyHeader{\textbf{10} &Feria VI ł - de ea - \textit{Alb}.}
\ApplyBody{

\item ad Magnificat~: ø \textit{Crucem sanctam} in tono II d (AM 468).}
\ApplyHeader{\textbf{11} &Sabbato - \textsc{Ss. Odonis, Maioli, Odilonis, Hugonis et B. Petri Venerabilis}, abbatum Cluniacensium - \textbf{memoria maior} - \textit{Alb.} (olim die 29 aprilis).}
\ApplyBody{
\item ad Vigilias~: ut in breviario, die 29 aprilis~; invitatorium proprium.
\item ad Laudes et Horas minores~: omnia ut in AM 877.
\item ad Benedictus~: ø \textit{O viri misericordiæ} cum \textit{alleluia} in tono I g (AM 973).
\item \textit{in ML~: Missa in supplemento OSB.}
\item in MC~: collecta in MP~; Commune sanctorum et sanctarum (MR 954)~; lectiones propriæ~: Apoc \textbf{19}, 1. 5-9a / Io \textbf{15}, 9-17~; præfatio de sanctis virginibus et religiosis.
\item I Vesperæ dominicæ sequentis.
\item \textit{ante Completorium, cantantur litaniæ lauretanæ Beatæ Mariæ Virginis ad vocationes sacerdotales ac religiosas impetrandas pro die universali vocationum.}}
\ApplyHebdoPsalt{- hebdomada II psalterii -}
\ApplyHeader{\textbf{12} & \textbf{\textsc{Dominica IV Paschæ}} (seu Dominica III post Pascha) - de ea - \textit{Alb.}}
\ApplyBody{
\item in MC~: præfatio pascalis IV.}
\ApplyLectHeader{Hebdomada IV Paschæ~: lectiones de tempore}
\ApplyLectBody{
\item[Dom. C] Act \textbf{13}, 14.43-52 / Ap \textbf{7}, 9.14b-17 / Io \textbf{10}, 27-30
\item[Feria II] Act \textbf{11}, 1-18 / Io \textbf{10}, 1-10
\item[Feria III] Act \textbf{11}, 19-26 / Io \textbf{10}, 22-30
\item[Feria IV] Act \textbf{12}, 24–13, 5a / Io \textbf{12}, 44-50
\item[Feria V] Act \textbf{13}, 13-25 / Io \textbf{13}, 16-20
\item[Feria VI] Act \textbf{13}, 26-30 / Io \textbf{14}, 1-6
\item[Sabbato] Act \textbf{13}, 44-52 / Io \textbf{14}, 7-14}
\ApplyPrefaceFeries{
\item in feriis~: præfatio paschalis IV, nisi aliter notetur.}
\ApplyHeader{\textbf{13} &Feria II - Beatæ Mariæ Virginis de Fatima - \textit{memoria minor} - \textit{Alb.}}
\ApplyBody{
\item ad Benedictus~: ø \textit{Ait Dominus} (AM 765)~; oratio in supplemento 118*.\item \textit{in ML~: Missa de die 11 februarii.}\item in MC~: Commune Beatæ Mariæ Virginis (MR 903)~; præfatio I de Beata Maria Virgine.}
\ApplyHeader{\textbf{14} &Feria III - \textbf{\textsc{S. Matthiæ, Apostoli}} - \textbf{festum} - \textit{Rub.} (olim die 24 februarii).}
\ApplyBody{\item in Officio~: omnia ut in Communi Apostolorum tempore paschali præter lectiones trium nocturnorum ad Vigilias in breviario cum responsoriis tamen tempore paschali.\item ad Benedictus~: ø \textit{Sancti et iusti} (AM 637a).\item \textit{in ML~: Missa tempore paschali (antiphona ad introitum, versus alleluiatici, antiphonæ ad offertorium et ad communionem sumuntur ex missa \emph{Protexisti}).}\item in MC~: lectiones propriæ~: Act \textbf{1}, 15-17.20a.20c-26 / Io \textbf{15}, 9-17~; præfatio II de Apostolis.}
\ApplyHeader{\textbf{15} &Feria IV ł - \textsc{S. Pacomii}, abbatis - \textbf{memoria maior} - \textit{Alb.}}
\ApplyBody{
\item in Officio~: oratio in supplemento 120 vel in variationibus 21.
\item ad Vigilias~: lectio in supplemento 119.
\item \textit{in ML~: Missa pro abbate.}
\item in MC~: collecta in MP~; Commune sanctorum et sanctarum (MR 958)~; præfatio I de sanctis.}
\ApplyHeader{\textbf{16} &Feria V - de ea - \textit{Alb}.}
\ApplyHeader{\textbf{17} &Feria VI ł - de ea - \textit{Alb}.}
\ApplyBody{

\item ad Magnificat~: ø \textit{Erit sanguis} cum \textit{alleluia} (AM 951).}
\ApplyHeader{\textbf{18} &Sabbato - S. Ioannis I, papæ et martyris - \textit{memoria minor} - \textit{Alb.}}
\ApplyBody{\item ad Benedictus~: ø \textit{Lux perpetua} (AM 632)~; oratio in supplemento 120.\item \textit{in ML (Rub.)~: Missa de Communi summorum pontificum.}
\item in MC \textit{(Rub.)}~: Commune pastorum (MR 927).
\item I Vesperæ dominicæ sequentis.}
\ApplyHebdoPsalt{- hebdomada I psalterii -}
\ApplyHeader{\textbf{19} & \textbf{\textsc{Dominica V Paschæ}} (seu Dominica IV post Pascha) - de ea - \textit{Alb.}}
\ApplyBody{
\item in MC~: præfatio pascalis V.}
\ApplyLectHeader{Hebdomada V Paschæ~: lectiones de tempore}
\ApplyLectBody{
\item[Dom. C] Act \textbf{14}, 20b-26 / Ap \textbf{21}, 1-5a / Io \textbf{13}, 31-33a 34-35
\item[Feria II] Act \textbf{14}, 5-17 / Io \textbf{14}, 21-26
\item[Feria III] Act \textbf{14}, 18-27 / Io \textbf{14}, 27-31a
\item[Feria IV] Act \textbf{15}, 1-6 / Io \textbf{15}, 1-8
\item[Feria V] Act \textbf{15}, 7-21 / Io \textbf{15}, 9-11
\item[Feria VI] Act \textbf{15}, 22-31 / Io \textbf{15}, 12-17
\item[Sabbato] Act \textbf{16}, 1-10 / Io \textbf{15}, 18-21}
\ApplyPrefaceFeries{
\item in feriis~: præfatio paschalis V, nisi aliter notetur.}
\ApplyHeader{\textbf{20} &Feria II - de ea - \textit{Alb}.}
\ApplyAnniv{\textup{†} Cras recurrit anniversarium obitus RP Bernardi Mariæ \textsc{Dewilde}, religiosis, qui die 21 maii 2014 obdormivit in Domino.}

\ApplyHeader{\textbf{21} &Feria III - de ea - \textit{Alb}.}
\ApplyHeader{\textbf{22} &Feria IV ł - de ea - \textit{Alb}.}
\ApplyHeader{\textbf{23} &Feria V - de ea - \textit{Alb}.}
\ApplyHeader{\textbf{24} &Feria VI ł - de ea - \textit{Alb}.}
\ApplyBody{

\item ad Magnificat~: ø \textit{Crucifixus} in tono VI f (AM 471).}
\ApplyHeader{\textbf{25} &Sabbato - \textsc{S. Bedæ venerabilis}, presbyteri et Ecclesiæ doctoris - \textbf{memoria maior} - \textit{Alb.} (olim die 27 huius).}
\ApplyBody{
\item ad Benedictus~: ø \textit{Qui verbum} cum \textit{alleluia} (AM 325).
\item in MC~: Commune doctorum Ecclesiæ (MR 943)~; præfatio de sanctis virginibus et religiosis.
\item I Vesperæ dominicæ sequentis.}
\ApplyHebdoPsalt{- hebdomada II psalterii -}
\ApplyHeader{\textbf{26} & \textbf{\textsc{Dominica VI Paschæ}} (seu Dominica V post Pascha) - de ea - \textit{Alb.}}
\ApplyBody{
\item in MC~: præfatio pascalis II.}
\ApplyLectHeader{Hebdomada VI Paschæ~: lectiones de tempore}
\ApplyLectBody{
\item[Dom. C] Act \textbf{15}, 1-2.22-29 / Ap \textbf{21}, 10-14  22-23 / Io \textbf{14}, 23-29
\item[Feria II] Act \textbf{16}, 11-15 / Io \textbf{15}, 26–16, 4
\item[Feria III] Act \textbf{16}, 22-34 / Io \textbf{16}, 5b-11
\item[Feria IV] Act \textbf{17}, 15.22–18, 1 / Io \textbf{16}, 12-15}

\ApplyLectHeader{In Ascensione Domini~: lectiones}
\ApplyLectBody{
\item[Anno C] Act \textbf{1}, 1-11 / Heb \textbf{9}, 24-28~; \textbf{10}, 19-23 / Lc \textbf{24}, 46-53\smallskip
\item[Feria VI] Act \textbf{18}, 9-18 / Io \textbf{16}, 20-23a
\item[Sabbato] Act \textbf{18}, 23-28 / Io \textbf{16}, 23b-28}
\ApplyPrefaceFeries{
\item in feriis~: præfatio paschalis II, nisi aliter notetur.}
\ApplyHeader{\textbf{27} & Feria II - S. Augustini Cantuariensis, episcopi - \textit{memoria minor} - \textit{\textit{Alb.}} (olim die 26 huius, vel 28 in ML).}
\ApplyBody{
\item in Officio et in ML nihil fit de Rogationibus, præter antiphonam ad \textit{Magnificat}.
\item ad Benedictus~: ø \textit{Amavit eum} (AM 663).\item in MC~: Commune pastorum (MR 938).}
\ApplyHeader{\textbf{28} & Feria III - de ea - \textit{Alb}.}
\ApplyBody{
\item in Officio et in ML nihil fit de Rogationibus, præter antiphonas ad \textit{Benedictus} et \textit{Magnificat}.}
\ApplyHeader{\textbf{29} & Feria IV ł - de ea - \textit{Alb}.}
\ApplyBody{
\item in Officio et in ML nihil fit de Rogationibus, præter antiphonam ad \textit{Benedictus}.\item \textit{in ML~: Missa de Vigilia Ascensionis.}
\item in MC~: Missa \textit{In conserendis agris} (MR 1127 A - MS 2134, n. 27 - GR 654)~; præfatio V de dominicis per annum.
\item I Vesperæ solemnitatis sequentis.}

\newpage
\ApplyGenerSubTitle{ad mensam~:}
\ApplyGenerList{
\item benedictio de Ascensione.
\vspace{1cm}}
\ApplyHeader{\textbf{30} & Feria V - ¬ \textbf{\MakeUppercase{In Ascensione Domini}} - \textbf{sollemnitas major} - \textit{Alb}.}
\ApplyBody{
\item in MC~: præfatio I de Ascensione Domini~; \textit{Communicantes} proprium.
\item Vesperæ sollemnitatis~; benedictio Sanctissimi Sacramenti.}

\newpage
\ApplyParBox{1cm}{\ApplyGenerTitleHuge{Tempus paschale II}}

\ApplyGenerTitleLarge{post Ascensionem Domini}
\ApplyGenerSubTitle{in Officio~:}
\ApplyGenerList{
\item ordinarium invenitur in variationibus 17 et in supplemento 46 pro breviario vetere.
\item ad Benedictus et Magnificat~: antiphonæ ut in die Ascensionis, nisi aliter notetur.
\item ad Vesperas~: in Officio dominicali et feriali, cantantur ¶ \textit{Spiritus Paraclitus} et hymnus \textit{Veni Creator} (AM 518).}
\ApplyGenerSubTitle{in ML et MC~:}
\ApplyGenerList{
\item præfatio de Ascensione dicitur, etiam in ceteris Missis quæ præfatione propria carent.
\item in feriis~: præfatio I de Ascensione, nisi aliter notetur.}
\medskip
\ApplyHeader{\textbf{31} & Feria VI ł - \textbf{\textsc{In Visitatione Beatæ Mariæ Virginis}} - \textbf{festum} - \textit{Alb.} (olim die 2 iulii).}
\ApplyBody{\item ad Vigilias~: in supplemento 122~; invitatorium proprium in supplemento 58. \item in MC~: lectiones propriæ~: Soph \textbf{3}, 14-18 / Lc \textbf{1}, 39-56~; præfatio II de Beata Maria Virgine.}

\ApplyNewMonthTitles{Iunius}
\ApplyNewMonthSubTitles{SS.MO Cordi Jesu consecratus}
\ApplyHeader{\textbf{1} & Sabbato § - \textsc{S. Iustini}, martyris - \textbf{memoria maior} - \textit{Rub.} (olim die 14 aprilis).}
\ApplyBody{
\item in Officio~: oratio in supplemento 130.
\item Ad Vigilias~: lectio in supplemento 130.
\item ad Benedictus~: ø \textit{Beati eritis} cum \textit{alleluia} (AM 1121).
\item in MC~: præfatio de sanctis martyribus.
\item I Vesperæ dominicæ sequentis.}
\ApplyHebdoPsalt{- hebdomada I psalterii -}
\ApplyHeader{\textbf{2} & ŧ \textbf{\textsc{Dominica VII Paschæ}} (seu Dominica post Ascensionem vel infra octavam Ascensionis) - de ea - \textit{Alb}.}
\ApplyBody{
\item ad Vigilias~: invitatorium de Ascensione~; pro breviario vetere~: hymnus ut in die Ascensionis, in unoquoque nocturno ø \textit{Alleluia, Alleluia, Alleluia}, psalmi de dominica in psalterio, ßß (ex unoquoque nocturno Ascensionis) \textit{Ascendit Deus}, \textit{Ascendens Christus}, \textit{Ascendo ad Patrem}.
\item ad Laudes et Horas minores~: antiphonæ et psalmi de dominica in psalterio tempore paschali.
\item in MC~: præfatio II de Ascensione.
\item Vesperæ dominicæ~: antiphonæ et psalmi de dominica in psalterio tempore paschali~; ¶ et hymnus in AM 518.}
\ApplyLectHeader{Hebdomada VII Paschæ~: lectiones de tempore}
\ApplyLectBody{
\item[Dom. C] Act \textbf{7}, 55-60 / Ap \textbf{22}, 12-14.16-17.20 / Io \textbf{17}, 20-26
\item[Feria II] Act \textbf{19}, 1-8 / Io \textbf{16}, 29-33
\item[Feria III] Act \textbf{20}, 17-27 / Io \textbf{17}, 1-11a
\item[Feria IV] Act \textbf{20}, 28-38 / Io \textbf{17}, 11b-19
\item[Feria V] Act \textbf{22}, 30~; \textbf{23}, 6-11 / Io \textbf{17}, 20-26
\item[Feria VI] Act \textbf{25}, 13-21 / Io \textbf{21}, 15-19
\item[Sabbato] Act \textbf{28}, 16-20.30-31 / Io \textbf{21}, 20-25}
\ApplyPrefaceFeries{
\item in feriis~: præfatio II de Ascensione, nisi aliter notetur.}
\ApplyAnniv{\textup{†} Cras recurrit anniversarium obitus RP Francisci Xavier Mariæ \textsc{Gaillot-Drevon}, sacerdotis, qui die 3 iunii 2018 obdormivit in Domino.}

\ApplyHeader{\textbf{3} & Feria II - \textsc{Ss. Caroli Lwanga et Sociorum}, martyrum - \textbf{memoria maior} - \textit{Rub.}}
\ApplyBody{
\item in Officio~: oratio in supplemento 132.
\item ad Vigilias~: lectio in supplemento 131.
\item ad Benedictus~: ø \textit{Lux perpetua} (AM 632).
\item \textit{in ML~: Missa pro pluribus martyribus tempore paschali.}\item in MC~: præfatio de sanctis martyribus.
\item ad Magnificat~: ø \textit{Sancti tui} (AM 632).}
\ApplyHeader{\textbf{4} & Feria III - de ea - \textit{Alb}.}
\ApplyHeader{\textbf{5} & Feria IV ł - \textsc{S. Bonifatii}, episcopi et martyris - \textbf{memoria maior} - \textit{Rub.}}
\ApplyBody{
\item in MC~: Commune martyrum (MR 921)~; præfatio de sanctis martyribus.}
\ApplyHeader{\textbf{6} & Feria V - de ea - \textit{Alb}.}
\ApplyHeader{\textbf{7} & Feria VI ł £ - de ea - \textit{Alb}.}
\ApplyBody{
\item \textit{in ML~: Missa de sacratissimo Corde Iesu (\emph{Gloria}).}}
\ApplyHeader{\textbf{8} & Sabbato - de eo - \textit{Alb}.}
\ApplyBody{
\item ad Benedictus~: ø \textit{Caritas Dei} (AM 531).
\item \textit{in ML (Rub)~: Missa de Vigilia Pentecostes.}
\item I Vesperæ sollemnitatis sequentis~; ß \textit{Benedicamus Domino} cum duobus \textit{Alleluia}.}
\ApplyGenerSubTitle{Ad mensam~:}
\ApplyGenerList{
\item benedictio de Pentecoste.}
\ApplyHebdoPsalt{- hebdomada II psalterii -}
\ApplyHeader{\textbf{9} & \textbf{\MakeUppercase{Dominica Pentecostes}} - \textbf{sollemnitas major} - \textit{Rub}.}
\ApplyBody{
\item Ad Laudes~: ß \textit{Benedicamus Domino} cum duobus \textit{Alleluia}
\item in MC~: tres lectiones Anno A~; sequentia~; præfatio et \textit{Communicantes} propria~; \textit{Ite Missa est, Alleluia Alleluia} ut in die Paschæ.
\item Vesperæ sollemnitatis~: hymnus \textit{Veni, Creator} ~: indulgentia plenaria~; ß \textit{Benedicamus Domino} cum duobus \textit{Alleluia}~; benedictio Sanctissimi Sacramenti.}
\ApplyLectHeader{Dominica Pentecostes~: lectiones}
\ApplyLectBody{
\item[Dom.] Act \textbf{2}, 1-11 / 1 Cor \textbf{12}, 3b-7.12-13 / Io \textbf{20}, 19-23}

\ApplyParBox{1cm}{\begin{center}\large{\textit{Post Completorium extinguitur cereus paschalis}}\par\large{\textit{et explicit tempus paschale.}}\end{center}}
\newpage
\ApplyParBox{1cm}{\ApplyGenerTitleHuge{Tempus per annum}
\medskip
\ApplyGenerTitleLarge{ab hebdomada X}}
\ApplyGenerList{
\item dicitur \textit{Angelus Domini}.
\item ad Completorium~: ø \textit{Salve Regina}.}
\ApplyGenerSubTitle{in Officio~:}
\ApplyGenerList{
\item in feriis~: oratio de octava, nisi aliter notetur.}
\ApplyGenerSubTitle{ad mensam~:}
\ApplyGenerList{
\item cras benedictio de Pentecoste~; a feria III benedictio de tempore per annum.}
\ApplyLectHeader{Hebdomada X per Annum~: lectiones de tempore}
\ApplyLectBody{
\item[Feria II] 2 Cor \textbf{1}, 1-7 / Mt \textbf{5}, 1-12
\item[Feria III] 2 Cor \textbf{1}, 18-22 / Mt \textbf{5}, 13-16
\item[Feria IV] 2 Cor \textbf{3}, 4-11 / Mt \textbf{5}, 17-19
\item[Feria V] 2 Cor \textbf{3}, 15–4, 1.3-6 / Mt \textbf{5}, 20-26
\item[Feria VI] 2 Cor \textbf{4}, 7-15 / Mt \textbf{5}, 27-32
\item[Sabbato] 2 Cor \textbf{5}, 14-21 / Mt \textbf{5}, 33-37}
\ApplyGenerList{
\item in feriis~: præfatio communis IV , nisi aliter notetur.
\medskip}
\ApplyHeader{\textbf{10} & Feria II - \textsc{Beatæ Mariæ Virginis, Ecclesiæ Matris} - \textbf{memoria maior} - \textit{Alb}.}
\ApplyBody{
\item de communi beatæ Mariæ Virginis, præter sequentia. 
\item oratio propria in supplemento 52* vel in AM 527*.
\item ad Vigilias : lectio in supplemento 49*.
\item ad Laudes : hymnus proprius (AM 525*) ; ad Benedictus : ø \textit{Perseverabant unanimiter} (AM 526*).
\item \textit{in ML 1962 (Rub) : Missa infra octavam} (Credo).
\item in MC : Missa de Beata Maria Ecclesiæ Matre (MR 1172); lectiones propriæ : Act \textbf{1}, 12-14 / Io \textbf{19}, 25-34 ; præfatio I de Beata Maria Virgine.
\item ad Magnificat : ø \textit{Dixit Dominus} (AM 527*).}
\ApplyHeader{\textbf{11} & Feria III - \textsc{S. Barnabæ, apostoli} - \textbf{memoria maior} - \textit{Rub.}}
\ApplyBody{\item ad Vigilias~: invitatorium et hymnus de Communi Apostolorum extra tempus paschale~; lectio in supplemento 132 cum ¶ extra tempus paschale~; in II nocturno lectio brevis in supplemento 133 vel ut ad Sextam in Communi Apostolorum.\item \textit{in ML 1962 (Rub) : Missa infra octavam} (Credo).
\item in MC~: lectiones propriæ~: Act \textbf{11}, 21b-26~; \textbf{13}, 1-3 / Mt \textbf{10}, 7-13~; præfatio I de Apostolis.
\item ad Magnificat~: ø \textit{Tradent enim vos} (AM 621).}
\ApplyHeader{\textbf{12} & Feria IV ł - de ea - \textit{Vir.}}
\ApplyBody{
\item ad Benedictus~: ø \textit{Ego sum panis} 1 (AM 528).
\item \textit{in ML 1962 (Rub)~: Missa infra octavam \emph{(Credo)}.}
\item ad Magnificat~: ø \textit{Ego sum panis} 2 (AM 528).}
\ApplyHeader{\textbf{13} & Feria V - \textsc{S. Antonii de Padova}, presbyteri et Ecclesiæ doctoris - \textbf{memoria maior} - \textit{Alb.}}
\ApplyBody{\item in Officio~: oratio in supplemento 134.\item ad Vigilias~: lectio in supplemento 133.
\item ad Benedictus~: ø \textit{Quod autem} (AM 324).
\item \textit{in ML 1962 (Rub) : Missa infra octavam} (Credo).
\item in MC~: Commune Sanctorum et Sanctarum (MR 962)~; præfatio I de sanctis.}
\ApplyHeader{\textbf{14} & Feria VI ł - de ea - \textit{Vir.}}
\ApplyBody{
\item ad Benedictus~: ø \textit{Dixit Jesus} (AM 530).
\item \textit{in ML 1962 (Rub)~: Missa infra octavam \emph{(Credo)}.}
\item ad Magnificat~: ø \textit{Paraclitus} (AM 531).}
\ApplyHeader{\textbf{15} & Sabbato  - \textsc{De Beata} - \textit{\textbf{memoria maior}} - \textit{Alb.}}
\ApplyBody{
\item ad Vigilias~: lectio sabbato 3 (in supplemento 202).
\item \textit{in ML 1962 (Rub) : Missa infra octavam} (Credo).
\item in MC~: Beatæ Mariæ Virginis, Matris et mediatricis gratiæ (CM 30)~; præfatio I de Beata Maria Virgine.
\item I Vesperæ solemnitatis sequentis
\medskip
\ApplyHebdoPsalt{\textbf{- pro breviario 62~: tomus alter -}}}
\ApplyHebdoPsalt{- hebdomada I psalterii -}
\ApplyHeader{\textbf{16} & \textbf{\textsc{Dominica XI per annum} - \MakeUppercase{Sanctissimæ Trinitatis}} (I post Pentecosten)  - \textbf{sollemnitas minor} - \textit{Alb}.}
\ApplyBody{
\item in MC~: MR 485~; præfatio propria.
\item Vesperæ sollemnitatis~; benedictio Sanctissimi Sacramenti.}
\ApplyLectHeader{Dominica Sanctissimæ Trinitatis~: lectiones}
\ApplyLectBody{
\item[Anno C] Prov \textbf{8}, 22-31 / Rom \textbf{5}, 1-5 / Io \textbf{16}, 12-15
\smallskip}
\ApplyLectHeader{Hebdomada XI per Annum~: lectiones de tempore}
\ApplyLectBody{
\item[Feria II] 2 Cor \textbf{6}, 1-10 / Mt \textbf{5}, 38-42
\item[Feria III] 2 Cor \textbf{8}, 1-9 / Mt \textbf{5}, 43-48
\item[Feria IV] 2 Cor \textbf{9}, 6-11 / Mt \textbf{6}, 1-6.16-18
\item[Feria V] 2 Cor \textbf{11}, 1-11 / Mt \textbf{6}, 7-15
\item[Feria VI] 2 Cor \textbf{11}, 18.21b-30 / Mt \textbf{6}, 19-23
\item[Sabbato] 2 Cor \textbf{12}, 1-10 / Mt \textbf{6}, 24-34}
\ApplyPrefaceFeries{
\item in feriis~: præfatio communis V , nisi aliter notetur.}
\ApplyHeader{\textbf{17} & Feria II - de ea - \textit{Vir.}}
\ApplyHeader{\textbf{18} & Feria III - de ea - \textit{Vir.}}
\ApplyHeader{\textbf{19} & Feria IV ł - \textsc{S. Romualdi}, abbatis - \textbf{memoria maior} - \textit{Alb.} (olim die 7 februarii).}
\ApplyBody{
\item ad Vigilias~: lectio et oratio in supplemento 134.
\item in MC~: Commune sanctorum et sanctarum (MR 958)~; præfatio de sanctis virginibus et religiosis.
\item I Vesperæ sollemnitatis sequentis.}
\ApplyHeader{\textbf{20} & Feria V - ¬ \textbf{\MakeUppercase{Ss.mi Corporis et Sanguinis Christi} - sollemnitas major} - \textit{Alb}.}
\ApplyBody{
\item \textit{in ML~: præfatio de Sanctissimo Sacramento.}
\item in MC~: MR 489~; lectiones~: Gen \textbf{14}, 18-20 / 1 Cor \textbf{11}, 23-26 / Lc \textbf{9}, 11b-17~; sequentia~; præfatio II de Sanctissima Eucharistia.
\item Vesperæ sollemnitatis~; post Vesperas, sollemnis processio Sanctissimi Sacramenti~; benedictio Sanctissimi Sacramenti~: \textit{Tantum ergo} cum ß \textit{Panem de cælo} cum \textit{Alleluia} (indulgentia plenaria).}
\ApplyHeader{\textbf{21} & Feria VI ł - S. Aloisii Gonzaga, religiosi - \textit{memoria minor} - \textit{Vir.}}
\ApplyBody{
\item ad Benedictus~: ø \textit{Sanctorum velut} (AM 652).
\item \textit{in ML~: Alb.}
\item in MC~: \textit{Alb.}}
\ApplyHeader{\textbf{22} & Sabbato  - \textsc{De Beata} - \textit{\textbf{memoria maior}} - \textit{Alb.}}
\ApplyBody{
\item ad Vigilias~: lectio sabbato 4 (in supplemento 202 pro breviario vetere).
\item in MC~: \textit{Sanctæ Mariæ, titulo Fons lucis et vitæ} (CM 16)~; præfatio I de Beata Maria Virgine.
\item I Vesperæ dominicæ sequentis.}
\ApplyHebdoPsalt{- hebdomada II psalterii -}
\ApplyHeader{\textbf{23} & \textbf{\textsc{Dominica XII per annum}} (II post Pentecosten vel infra octavam Ss. Corporis Christi) - de ea - \textit{Vir.}}
\ApplyBody{
\item Officium totum dicitur ut in dominicis per annum, præter antiphonas ad Benedictus et Magnificat in AM 557-558. 
\item ad Vigilias pro breviario vetere~: lectiones trium nocturnorum ut in breviario (dominica infra octavam Ss. Corporis Christi) sed cum ¶ dominicæ IV post Pentecosten in supplemento 50.
\item in MC~: præfatio III de dominicis.\item I Vesperæ sollemnitatis sequentis.}
\ApplyLectHeader{Hebdomada XII per Annum~: lectiones de tempore}
\ApplyLectBody{
\item[Dom. C] Zac \textbf{12}, 10-11 / Gal \textbf{3}, 26-29 / Lc \textbf{9}, 18-24
\item[Feria II] Gen \textbf{12}, 1-9 / Mt \textbf{7}, 1-5
\item[Feria III] Gen \textbf{13}, 2.5-18 / Mt \textbf{7}, 6.12-14
\item[Feria IV] Gen \textbf{15}, 1-12.17-18 / Mt \textbf{7}, 15-20
\item[Feria V] Gen \textbf{16}, 1-12.15-16 / Mt \textbf{7}, 21-29
\item[Feria VI] Gen \textbf{17}, 1.9-10.15-22 / Mt \textbf{8}, 1-4
\item[Sabbato] Gen \textbf{18}, 1-15 / Mt \textbf{8}, 5-17}
\ApplyPrefaceFeries{
\item in feriis~: præfatio communis VI, nisi aliter notetur.}
\ApplyHeader{\textbf{24} & Feria II - þ \textbf{\MakeUppercase{In nativitate S. Ioannis Baptistæ}} - \textbf{sollemnitas minor} - \textit{Alb.}}
\ApplyBody{
\item \textit{in ML~: præfatio propria.}
\item in MC~: lectiones propriæ~: Is \textbf{49}, 1-6 / Act \textbf{13}, 22-26 / Lc \textbf{1}, 57-66.80~; præfatio propria.
\item Vesperæ sollemnitatis.}
\ApplyAnniv{\textup{†} Cras recurrit anniversarium obitus RP Ioannis Mariæ \textsc{Pouchet}, sacerdotis, qui die 25 iunii 2012, in Abbatia Dominæ Nostræ Mayliliensis, obdormivit in Domino.}

\ApplyHeader{\textbf{25} & Feria III - de ea - \textit{Vir.}}
\ApplyHeader{\textbf{26} & Feria IV ł - de ea - \textit{Vir.}}
\ApplyHeader{\textbf{27} & Feria V - S. Cyrilli Alexandrini, episcopi et Ecclesiæ doctoris - \textit{memoria minor} - \textit{Vir.} (olim die 9 februarii).}
\ApplyBody{\item ad Benedictus~: ø \textit{Maternitas tua} (AM 1085)~; oratio in AM 797 vel in supplemento 135.\item \textit{in ML~: Alb.}
\item in MC \textit{(Alb.)}~: Commune pastorum (MR 929).
\item I Vesperæ sollemnitatis sequentis.}
\ApplyHeader{\textbf{28} & Feria VI ł - þ \textbf{\MakeUppercase{Sacratissimi Cordis Jesu}} - \textbf{sollemnitas minor} - \textit{Alb}.}
\ApplyBody{
\item in MC~: MR 492~; lectiones~: Ez \textbf{34}, 11-16 / Rom \textbf{5}, 5-11 / Lc \textbf{15}, 3-7~; præfatio propria.
\item Vesperæ sollemnitatis~; ad benedictionem Sanctissimi Sacramenti recitetur \textit{Reparationis actus Sacratissimo Cordi Iesu}~: indulgentia plenaria.}
\ApplyHeader{\textbf{29} & Sabbato - þ \textbf{\MakeUppercase{Ss. Petri et Pauli, Apostolorum}} - \textbf{sollemnitas minor} - \textit{Rub.}}
\ApplyBody{
\item ad Benedictus~: ø \textit{Petrus} (AM 959).
\item in MC~: lectiones propriæ~: Act \textbf{12}, 1-11 / 1 Tim \textbf{4}, 6-8.16-18 / Mt \textbf{16}, 13-19~; præfatio propria.
\item Vesperæ sollemnitatis ut in AM 939, præter antiphonas et capitulum de commemoratione S. Pauli Apostoli (AM 941), cum psalmis ut in II Vesperis de Communi Apostolorum.}
\ApplyHebdoPsalt{- hebdomada I psalterii -}
\ApplyHeader{\textbf{30} & \textbf{\textsc{Dominica XIII per annum}} (III post Pentecosten vel infra octavam Ss. Cordis Iesu) - de ea - \textit{Vir.}}
\ApplyBody{
\item Officium totum dicitur ut in dominicis per annum præter antiphonas ad Benedictus et Magnificat. 
\item ad Vigilias pro breviario vetere~: lectiones ut in breviario (dominica infra octavam Ss. Cordis Iesu), sed lectiones II nocturni in supplemento 49~; ¶ dominicæ IV post Pentecosten in supplemento 50.
\item in MC~: præfatio IV de dominicis.}
\ApplyLectHeader{Hebdomada XIII per Annum~: lectiones de tempore}
\ApplyLectBody{
\item[Dom. C] 1 Reg \textbf{19}, 16b.19-21 / Gal \textbf{4}, 31b–5, 1.13-18 / Lc \textbf{9}, 51-62
\item[Feria II] Gen \textbf{18}, 16-33 / Mt \textbf{8}, 18-22
\item[Feria III] Gen \textbf{19}, 15-29 / Mt \textbf{8}, 23-27
\item[Feria IV] Gen \textbf{21}, 5.8-20 / Mt \textbf{8}, 28-34
\item[Feria V] Gen \textbf{22}, 1-19 / Mt \textbf{9}, 1-8
\item[Feria VI] Gen \textbf{23}, 1-4.19~; \textbf{24}, 1-8.62-67 / Mt \textbf{9}, 9-13
\item[Sabbato] Gen \textbf{27}, 1-5.15-29 / Mt \textbf{9}, 14-17}
\ApplyPrefaceFeries{
\item in feriis~: præfatio communis I, nisi aliter notetur.}

\ApplyNewMonthTitles{Iulius}
\ApplyHeader{\textbf{1} & Feria II - de ea - \textit{Vir.}}
\ApplyBody{
\item in MC (\textit{Nigr.})~: Missa defunctorum pro omnibus benefactoribus nostris defunctis (MR 1225)~; lectiones propriæ~: Rom \textbf{8}, 14-17 / Lc \textbf{12}, 35-38.40~; præfatio III de defunctis.}
\ApplyHeader{\textbf{2} & Feria III - de ea - \textit{Vir.}}
\ApplyHeader{\textbf{3} & Feria IV ł - \textbf{\textsc{S. Thomæ, apostoli}} - \textbf{festum} - \textit{Rub.} (olim die 21 decembris).}
\ApplyBody{
\item ad Vigilias~: in supplemento 138.
\item ad Benedictus~: ø \textit{Quia vidisti} (AM 479).
\item in MC~: lectiones propriæ~: Eph \textbf{2}, 19-22 / Io \textbf{20}, 24-29~; præfatio I de Apostolis.
\item ad Magnificat~: ø \textit{Misi digitum} (AM 480).}
\ApplyHeader{\textbf{4} & Feria V - de ea - \textit{Vir.}}
\ApplyHeader{\textbf{5} & Feria VI µ £ - de ea - \textit{Vir.}}
\ApplyBody{
\item \textit{in ML (Alb.)~: Missa de sacratissimo Corde Iesu \emph{(Gloria)}.}
\item in MC (Alb.) : Missa votiva de Dei Misericordia (MR 1158) ; præfatio communis II (in tono simplici).}
\ApplyHeader{\textbf{6} & Sabbato §  - \textsc{De Beata} - \textit{\textbf{memoria maior}} - \textit{Alb.}}
\ApplyBody{
\item ad Vigilias~: lectio sabbato 1 (in supplemento 203 pro breviario vetere).
\item ad Benedictus~: ø \textit{Beata es} (AM 1074).
\item \textit{in ML~: Immaculati Cordis Beatæ Mariæ Virginis.}
\item in MC~: \textit{Immaculati Cordis Beatæ Mariæ Virginis} (CM 28)~; præfatio I de Beata Maria Virgine.
\item I Vesperæ dominicæ sequentis.}
\ApplyHebdoPsalt{- hebdomada II psalterii -}
\ApplyHeader{\textbf{7} & ŧ \textbf{\textsc{Dominica XIV per annum}} (IV post Pentecosten) - de ea - \textit{Vir.}}
\ApplyBody{
\item in MC~: præfatio V de dominicis.}
\ApplyLectHeader{Hebdomada XIV per Annum~: lectiones de tempore}
\ApplyLectBody{
\item[Dom. C] Is \textbf{66}, 10-14c / Gal \textbf{6}, 14-18 / Lc \textbf{10}, 1-12.17-20
\item[Feria II] Gen \textbf{28}, 10-22a / Mt \textbf{9}, 18-26
\item[Feria III] Gen \textbf{32}, 22-32 / Mt \textbf{9}, 32-38
\item[Feria IV] Gen \textbf{41}, 55-57~; \textbf{42}, 5-7a 17-24a / Mt \textbf{10}, 1-7
\item[Feria V] Gen \textbf{44}, 18-21.23b-29~; \textbf{45}, 1-5 / Mt \textbf{10}, 7-15
\item[Feria VI] Gen \textbf{46}, 1-7.28-30 / Mt \textbf{10}, 16-23
\item[Sabbato] Gen \textbf{49}, 29-32~; \textbf{50}, 15-24 / Mt \textbf{10}, 24-33}
\ApplyPrefaceFeries{
\item in feriis~: præfatio communis II, nisi aliter notetur.}
\ApplyHeader{\textbf{8} & Feria II - de ea - \textit{Vir.}}
\ApplyHeader{\textbf{9} & Feria III - de ea - \textit{Vir.}}
\ApplyHeader{\textbf{10} & Feria IV ł - de ea - \textit{Vir.}}
\ApplyBody{

\item I Vesperæ sollemnitatis sequentis.}
\ApplyHeader{\textbf{11} & Feria V - ¬ \textbf{\MakeUppercase{S. P. N. Benedicti, Abbatis,}} \textsc{Europæ patroni} - \textbf{sollemnitas maior} - \textit{Alb.}}
\ApplyBody{
\item ad Vigilias~: in nocturno III~: lectio 9.
\item \textit{in ML~: Missa in supplemento OSB~; præfatio propria.}
\item in MC~: omnia in MP~; lectiones propriæ~: Pr \textbf{2}, 1-9 / Ep \textbf{4}, 1-6 / Mt \textbf{19}, 27-29~; sequentia~; præfatio propria.
\item Vesperæ sollemnitatis~; benedictio Sanctissimi Sacramenti.}
\ApplyHeader{\textbf{12} & Feria VI ł - S. Ioannis Gualberti, abbatis - \textit{memoria minor} - \textit{Vir.}}
\ApplyBody{
\item ad Benedictus~: ø \textit{Estote} in tono I f (AM 538).
\item \textit{in ML~: Alb.}
\item in MC \textit{(Alb.)}~: collecta in MP~; Commune sanctorum et sanctarum (MR 962).}
\ApplyHeader{\textbf{13} & Sabbato  - \textsc{De Beata} - \textit{\textbf{memoria maior}} - \textit{Alb.}}
\ApplyBody{
\item ad Vigilias~: lectio sabbato 2 (in supplemento 203 pro breviario vetere).
\item in MC~: \textit{Sanctæ Mariæ, Reginæ et Matris misericordiæ} (CM 39)~; præfatio I de Beata Maria Virgine.
\item I Vesperæ dominicæ sequentis.}
\ApplyHebdoPsalt{- hebdomada I psalterii -}
\ApplyHeader{\textbf{14} & \textbf{\textsc{Dominica XV per annum}} (V post Pentecosten) - de ea - \textit{Vir.}}
\ApplyBody{
\item in MC~: præfatio VI de dominicis.}
\ApplyLectHeader{Hebdomada XV per Annum~: lectiones de tempore}
\ApplyLectBody{
\item[Dom. C] Deut \textbf{30}, 10-14 / Col \textbf{1}, 15-20 / Lc \textbf{10}, 25-37
\item[Feria II] Ex \textbf{1}, 8-14.22 / Mt \textbf{10}, 34–\textbf{11}, 1
\item[Feria III] Ex \textbf{2}, 1-15a / Mt \textbf{11}, 20-24
\item[Feria IV] Ex \textbf{3}, 1-6.9-12 / Mt \textbf{11}, 25-27
\item[Feria V] Ex \textbf{3}, 13-20 / Mt \textbf{11}, 28-30
\item[Feria VI] Ex \textbf{11}, 10–12, 14 / Mt \textbf{12}, 1-8
\item[Sabbato] Ex \textbf{12}, 37-42 / Mt \textbf{12}, 14-21}
\ApplyPrefaceFeries{
\item in feriis~: præfatio communis III, nisi aliter notetur.}
\ApplyHeader{\textbf{15} & Feria II - \textsc{S. Bonaventuræ}, episcopi et Ecclesiæ doctoris - \textbf{memoria maior} - \textit{Alb.} (olim die 14 huius).}
\ApplyBody{
\item in MC~: Commune pastorum (MR 931)~; præfatio de sanctis pastoribus.}
\ApplyAnniv{\textup{†} Cras recurrit anniversarium obitus RP Benedicti Mariæ \textsc{Constantin}, sacerdotis, qui die 16 iulii 1965, in Abbatia Dominæ Nostræ Mayliliensis, obdormivit in Domino.}

\ApplyHeader{\textbf{16} & Feria III - Beatæ Mariæ Virginis de Monte Carmelo - \textit{memoria minor} - \textit{Vir.}}
\ApplyBody{\item ad Benedictus~: ø \textit{Caput tuum} (AM 971).\item \textit{in ML~: Alb.}
\item in MC \textit{(Alb.)}~: CM 32~; præfatio I de Beata Maria Virgine.}
\ApplyAnniv{\textup{†} Cras recurrit anniversarium obitus Reverendissimi ac dilectissimi Patris Augustini Mariæ \textsc{Gorce}, abbatis et fundatoris Dominæ Nostræ Mayliliensis, qui die 17 iulii 1962 obdormivit in Domino.}

\ApplyHeader{\textbf{17} & Feria IV ł - de ea - \textit{Vir.}}
\ApplyHeader{\textbf{18} & Feria V - de ea - \textit{Vir.}}
\ApplyHeader{\textbf{19} & Feria VI ł - de ea - \textit{Vir.}}
\ApplyHeader{\textbf{20} & Sabbato  - \textsc{De Beata} - \textit{\textbf{memoria maior}} - \textit{Alb.}}
\ApplyBody{
\item ad Vigilias~: lectio sabbato 3 (in supplemento 204 pro breviario vetere).
\item in MC~: \textit{Sanctæ Mariæ, Ancillæ Domini} (CM 22)~; præfatio I de Beata Maria Virgine.
\item I Vesperæ dominicæ sequentis.}
\ApplyHebdoPsalt{- hebdomada II psalterii -}
\ApplyHeader{\textbf{21} & \textbf{\textsc{Dominica XVI per annum}} (VI post Pentecosten) - de ea - \textit{Vir.}}
\ApplyBody{
\item in MC~: præfatio VII de dominicis.}
\ApplyLectHeader{Hebdomada XVI per Annum~: lectiones de tempore}
\ApplyLectBody{
\item[Dom. C] Gen \textbf{18}, 1-10a / Col \textbf{1}, 24-28 / Lc \textbf{10}, 38-42
\item[Feria II] Ex \textbf{14}, 5-18 / Mt \textbf{12}, 38-42
\item[Feria III] Ex \textbf{14}, 21–15, 1 / Mt \textbf{12}, 46-50
\item[Feria IV] Ex \textbf{16}, 1-5.9-15 / Mt \textbf{13}, 1-9
\item[Feria V] Ex \textbf{19}, 1-2.9-11.16-20b / Mt \textbf{13}, 10-17
\item[Feria VI] Ex \textbf{20}, 1-17 / Mt \textbf{13}, 18-23
\item[Sabbato] Ex \textbf{24}, 3-8 / Mt \textbf{13}, 24-30}
\ApplyPrefaceFeries{
\item in feriis~: præfatio communis IV, nisi aliter notetur.}
\ApplyHeader{\textbf{22} & Feria II - \textbf{\textsc{S. Mariæ Magdalenæ}} - \textbf{festum} - \textit{Alb.}}
\ApplyBody{
\item in Officio~: omnia de Communi nec virginis nec martyris præter sequentia.
\item oratio propria.
\item ad Vigilias~: invitatorium proprium~; hymnus proprius~; lectio in secundo nocturno in supplemento 143*.
\item ad Laudes~: hymnus proprius~; ad Benedictus~: ø \textit{In diebus illis} (AM 975).
\item in MC~: lectiones propriæ~: 2 Cor \textbf{5}, 14-17 / Io \textbf{20}, 1.11-18~; præfatio propria.
\item ad Vesperas~: hymnus proprius~; ad Magnificat~: ø \textit{Mulier} (AM 978).}
\ApplyHeader{\textbf{23} & Feria III - \textbf{\textsc{S. Birgittæ, Europæ Patronæ, Religiosæ}} - \textbf{festum} - \textit{Alb.} (olim die 8 octobris).}
\ApplyBody{
\item in Officio~: omnia de Communi nec virginis nec martyris præter sequentia.
\item oratio in supplemento 146*.
\item ad Vigilias~: lectio II nocturni in supplemento 144*.
\item \textit{in ML~: præfatio de sanctis.}
\item in MC~: Commune sanctorum et sanctarum (MR 966)~; lectiones propriæ~: Iudt \textbf{8}, 2-8 / Mt \textbf{5}, 13-16~; præfatio I de sanctis.}
\ApplyHeader{\textbf{24} & Feria IV ł - de ea - \textit{Vir.}}
\ApplyHeader{\textbf{25} & Feria V - \textbf{\textsc{S. Iacobi, apostoli}} - \textbf{festum} - \textit{Rub.}}
\ApplyBody{
\item ad Benedictus~: ø \textit{Assumpsit Iesus} (AM 998).
\item in MC~: lectiones propriæ~: 2 Cor \textbf{4}, 7-15 / Mt \textbf{20}, 20-28~; præfatio II de Apostolis.}
\ApplyHeader{\textbf{26} & Feria VI ł - \textsc{Ss. Ioachim et Annæ}, parentum Beatæ Mariæ Virginis - \textbf{memoria maior} - \textit{Alb.}}
\ApplyBody{
\item ad Vigilias pro breviario vetere~: in supplemento 144.
\item ad Laudes, Vesperas et Horas minores~: antiphonæ propriæ.
\item \textit{in ML~: Missa in supplemento OSB.}
\item in MC~: lectiones propriæ~: Si \textbf{44}, 1.10-15 / Mt \textbf{13}, 16-17~; præfatio II de sanctis.
\item ad Magnificat~: ø \textit{Benedictionem} (AM 979).}
\ApplyHeader{\textbf{27} & Sabbato  - \textsc{De Beata} - \textit{\textbf{memoria maior}} - \textit{Alb.}}
\ApplyBody{
\item ad Vigilias~: lectio sabbato 4 (in supplemento 204 pro breviario vetere).
\item in MC~: \textit{Beatæ Mariæ Virginis, titulo Templum Domini} (CM 23)~; præfatio I de Beata Maria Virgine.
\item I Vesperæ dominicæ sequentis.}
\ApplyHebdoPsalt{- hebdomada I psalterii -}
\ApplyHeader{\textbf{28} & \textbf{\textsc{Dominica XVII per annum}} (VII post Pentecosten) - de ea - \textit{Vir.}}
\ApplyBody{
\item in MC~: præfatio VIII de dominicis.}
\ApplyLectHeader{Hebdomada XVII per Annum~: lectiones de tempore}
\ApplyLectBody{
\item[Dom. C] Gen \textbf{18}, 20-32 / Col \textbf{2}, 12-14 / Lc \textbf{11}, 1-13
\item[Feria II] Ex \textbf{32}, 15-24.30-34 / Mt \textbf{13}, 31-35
\item[Feria III] Ex \textbf{33}, 7-11~; \textbf{34}, 5b-9.28 / Mt \textbf{13}, 36-43
\item[Feria IV] Ex \textbf{34}, 29-35 / Mt \textbf{13}, 44-45
\item[Feria V] Ex \textbf{40}, 14-19.32-36 / Mt \textbf{13}, 47-53
\item[Feria VI] Lev \textbf{23}, 1.4-11.15-16.27.34b-37 / Mt \textbf{13}, 54-58
\item[Sabbato] Lev \textbf{25}, 1.8-17 / Mt \textbf{14}, 1-12}
\ApplyPrefaceFeries{
\item in feriis~: præfatio communis V, nisi aliter notetur.}
\ApplyHeader{\textbf{29} & Feria II - \textsc{Ss. Marthæ, Mariæ et Lazari}, hospitum Domini - \textbf{memoria maior} - \textit{Alb.}}
\ApplyBody{
\item in Officio~: oratio in supplemento 148.
\item ad Vigilias~: in supplemento 146~; invitatorium \textit{Mirabilem} in supplemento 59 vel in variationibus 35.
\item ad Laudes~: a capitulo ut in variationibus 26 (die 5 octobris)~; ad Benedictus~: ø \textit{Maria ergo} (AM 977).
\item ad Horas~: antiphonæ, capitula et ß ut in variationibus 25.
\item \textit{in ML~: S. Marthæ, virginis.}
\item in MC~: omnia in MP~; lectiones propriæ~: 1 Io \textbf{4}, 7-16 / Lc \textbf{10}, 38-42~; præfatio II de sanctis.
\item ad Vesperas~: a capitulo ut in variationibus 29, præter ad Magnificat~: ø \textit{Domine si} (AM 380).}
\ApplyHeader{\textbf{30} & Feria III - S. Petri Chrysologi, episcopi et Ecclesiæ doctoris - \textit{memoria minor} - \textit{Vir.} (olim die 4 decembris).}
\ApplyBody{
\item ad Benedictus~: ø \textit{Sacerdos} (AM 656)~; oratio in supplemento 148.
\item \textit{in ML~: Alb.}
\item in MC \textit{(Alb.)}~: Commune doctorum Ecclesiæ (MR 944).}
\ApplyHeader{\textbf{31} & Feria IV ł - \textsc{S. Ignatii de Loyola}, presbyteri - \textbf{memoria maior} - \textit{Alb.}}
\ApplyBody{
\item ad Laudes, Vesperas et Horas minores~: antiphonæ e Communi confessoris non pontificis.
\item ad Benedictus~: ø \textit{Ignem} (AM 564).
\item in MC~: lectiones propriæ~: 1 Co \textbf{10}, 31–\textbf{11}, 1 / Lc \textbf{14}, 25-33~; præfatio de sanctis virginibus et religiosis.
\item ad Magnificat~: ø \textit{Estote} (AM 628).}

\ApplyNewMonthTitles{Augustus}
\ApplyHeader{\textbf{1} & Feria V - \textsc{S. Alfonsi de Liguori}, episcopi et Ecclesiæ doctoris - \textbf{memoria maior} - \textit{Alb.} (olim die 2 huius).}
\ApplyBody{
\item ad Vigilias~: lectio in supplemento 149.
\item ad Benedictus~: ø \textit{Observa} (AM 582).
\item in MC~: præfatio de sanctis pastoribus.}
\ApplyHeader{\textbf{2} & Feria VI µ £ - S. Petri Iuliani Eymard, presbyteri - \textit{memoria minor} - \textit{Vir.}}
\ApplyBody{\item ad Benedictus~: ø \textit{Ego sum} 2 (AM 553)~; oratio in supplemento 150*.
\item \textit{in ML (Alb.)~: Missa pro confessore non pontifice.}
\item in MC \textit{(Alb.)}~: Commune sanctorum et sanctarum (MR 961).}
\ApplyHeader{\textbf{3} & Sabbato §  - \textsc{De Beata} - \textit{\textbf{memoria maior}} - \textit{Alb.}}
\ApplyBody{
\item ad Vigilias~: lectio sabbato 1 (in supplemento 205 pro breviario vetere).
\item ad Benedictus~: ø \textit{Beata es} (AM 1074).
\item \textit{in ML~: Immaculati Cordis Beatæ Mariæ Virginis.}
\item in MC~: \textit{Immaculati Cordis Beatæ Mariæ Virginis} (CM 28)~; præfatio I de Beata Maria Virgine.
\item I Vesperæ dominicæ sequentis.}
\ApplyHebdoPsalt{- hebdomada II psalterii -}
\ApplyHeader{\textbf{4} & ŧ \textbf{\textsc{Dominica XVIII per annum}} (VIII post Pentecosten - I augusti) - de ea - \textit{Vir.}}
\ApplyBody{
\item ad Vigilias~: lectiones nocturnorum I et II sumendæ sunt e dominica I augusti, lectiones nocturni III e dominica post Pentecosten, et sic usque ad Adventum.
\item in MC~: præfatio I de dominicis.}
\ApplyLectHeader{Hebdomada XVIII per Annum~: lectiones de tempore}
\ApplyLectBody{
\item[Dom. C] Qoh \textbf{1}, 2~; \textbf{2}, 21-23 / Col \textbf{3}, 1-5.9-11 / Lc \textbf{12}, 13-21
\item[Feria II] Num \textbf{11}, 4b-15 / Mt \textbf{14}, 22-36
\item[Feria III] Num \textbf{12}, 1-13 / Mt \textbf{15}, 1-2.10-14
\item[Feria IV] Num \textbf{13}, 2-3a 26–14, 1.26-30.34-35 / Mt \textbf{15}, 21-28
\item[Feria V] Num \textbf{20}, 1-13 / Mt \textbf{16}, 13-23
\item[Feria VI] Deut \textbf{4}, 32-40 / Mt \textbf{16}, 24-28
\item[Sabbato] Deut \textbf{6}, 4-13 / Mt \textbf{17}, 14-19}
\ApplyPrefaceFeries{
\item in feriis~: præfatio communis VI, nisi aliter notetur.}
\ApplyHeader{\textbf{5} & Feria II - Dedicatio basilicæ S. Mariæ - \textit{memoria minor} - \textit{Vir.}}
\ApplyBody{
\item ad Benedictus~: ø \textit{Beata es, Maria} (AM 709).
\item \textit{in ML~: Alb.}
\item in MC \textit{(Alb.)}~: Commune Beatæ Mariæ Virginis (MR 898)~; præfatio I de Beata Maria Virgine.}
\ApplyHeader{\textbf{6} & Feria III - \textbf{\textsc{in Transfiguratione Domini}} - \textbf{festum} - \textit{Alb.}}
\ApplyBody{
\item ad Horas et Completorium, tonus proprius pro hymnis.
\item in MC~: lectiones propriæ~: Dan \textbf{7}, 9-10.13-14 / Lc \textbf{9}, 28b-36~; præfatio propria.
\item ad Magnificat~: ø \textit{Christus Iesus} (AM 997).}
\ApplyHeader{\textbf{7} & Feria IV ł - de ea - \textit{Vir.}}
\ApplyHeader{\textbf{8} & Feria V - \textsc{S. Dominici}, presbyteri - \textbf{memoria maior} - \textit{Alb.} (olim die 4 huius).}
\ApplyBody{
\item ad Vigilias pro breviario vetere~: lectio XII cum ¶ suo.
\item ad Benedictus~: ø \textit{Vos qui reliquistis omnia} (AM 624).
\item in MC~: præfatio de sanctis virginibus et religiosis.
\item ad Magnificat~: ø \textit{Qui me confessus} (AM 640).}
\ApplyHeader{\textbf{9} & Feria VI ł - \textbf{\textsc{S. Teresiæ Benedictæ a Cruce} (Edith Stein), \textsc{virginis, martyris et Europæ patronæ}} - \textbf{festum} - \textit{Rub.}}
\ApplyBody{
\item in Officio~: omnia de Communi virginis martyris, præter sequentia.
\item oratio in supplemento 153*.
\item ad Vigilias~: lectio I nocturni de libro Ecclesiastici \textbf{51}, 1-17 in breviario monastico 172*~; lectio II nocturni in supplemento 151*.
\item ad Benedictus~: ø \textit{Ego sum} (AM 982).
\item \textit{in ML~: Missa pro virgine et martyre~; præfatio de sanctis.}
\item in MC~: Commune virginis martyris (MR 924)~; lectiones propriæ~: Os \textbf{2}, 16b. 17b. 21-22 / Mt \textbf{25}, 1-13~; præfatio de sanctis martyribus.
\item ad Magnificat~: ø \textit{Elevare} (AM 222).}
\ApplyHeader{\textbf{10} & Sabbato - \textbf{\textsc{S. Laurentii, diaconi et martyris}} - \textbf{festum} - \textit{Rub.}}
\ApplyBody{
\item in MC~: lectiones propriæ~: 2 Cor \textbf{9}, 6-10 / Io \textbf{12}, 24-26~; præfatio de sanctis martyribus.
\item I Vesperæ dominicæ sequentis.}
\ApplyHebdoPsalt{- hebdomada I psalterii -}
\ApplyHeader{\textbf{11} & \textbf{\textsc{Dominica XIX per annum}} (IX post Pentecosten - II augusti) - de ea - \textit{Vir.}}
\ApplyBody{
\item in MC~: pro lectione secunda et evangelio sumitur formula brevior (Heb \textbf{11}, 1-2.8-12 / Lc \textbf{12}, 35-40)~; præfatio II de dominicis.}
\ApplyLectHeader{Hebdomada XIX per Annum~: lectiones de tempore}
\ApplyLectBody{
\item[Dom. C] Sap \textbf{18}, 6-9 / Hebr \textbf{11}, 1-2.8-19 / Lc \textbf{12}, 32-48
\item[Feria II] Deut \textbf{10}, 12-22 / Mt \textbf{17}, 21-26
\item[Feria III] Deut \textbf{31}, 1-8 / Mt \textbf{18}, 1-5.10.12-14
\item[Feria IV] Deut \textbf{34}, 1-12 / Mt \textbf{18}, 15-20
\item[Feria V] Ios \textbf{3}, 7-10a.11.13-17 / Mt \textbf{18}, 21–19, 1
\item[Feria VI] Ios \textbf{24}, 1-13 / Mt \textbf{19}, 3-12
\item[Sabbato] Ios \textbf{24}, 14-29 / Mt \textbf{19}, 13-15}
\ApplyPrefaceFeries{
\item in feriis~: præfatio communis I, nisi aliter notetur.}
\ApplyHeader{\textbf{12} & Feria II - S. Ioannæ Franciscæ Frémiot de Chantal, viduæ - memoria minor - \textit{Alb.}}
\ApplyBody{
\item ad Benedictus~: ø \textit{Date ei} (AM 688)~; oratio in supplemento 60*.
\item \textit{in ML~: olim die 21 augusti.}
\item in MC~: Commune sanctorum et sanctarum (MR 962).}
\ApplyHeader{\textbf{13} & Feria III - de ea - \textit{Vir.}}
\ApplyHeader{\textbf{14} & Feria IV ł - \textsc{S. Maximiliani Mariæ Kolbe}, presbyteri et martyris - \textbf{memoria maior} - \textit{Rub.}}
\ApplyBody{
\item in Officio~: oratio in supplemento 153.
\item ad Vigilias~: lectio in supplemento 152.
\item ad Benedictus~: ø \textit{Majorem caritatem} (AM 622).
\item \textit{in ML~: Missa pro martyre non pontifice.}
\item in MC~: præfatio de sanctis martyribus\item I Vesperæ sollemnitatis sequentis.}
\ApplyHeader{\textbf{15} & Feria V - ¬ \textbf{\MakeUppercase{In assumptione Beatæ Mariæ Virginis}}, \textsc{Patronæ Principalis Galliæ} - \textbf{sollemnitas maior} - \textit{Alb.}}
\ApplyBody{
\item ad Vigilias~: in breviario 62 primum Officium. In nocturno I~: lectiones 1 et 2 cum ¶ lectionis 4~; in nocturno II~: lectiones 5 et 6 cum ¶ lectionis 8~; in nocturno III~: lectiones 9, 10, 11 et 12.
\item in MC~: lectiones propriæ~: Apoc \textbf{11}, 19a~; \textbf{12}, 1-6a.10ab / 1 Cor \textbf{15}, 20-27a / Lc \textbf{1}, 39-56~; præfatio de Assumptione.
\item post Vesperas sollemnitatis incipiuntur Litaniæ Beatæ Mariæ Virginis pro sollemni supplicatione iuxta votum Regis Ludovici XIII et fit processio~; benedictio Sanctissimi Sacramenti.
\item ad Completorium~: ø \textit{Ave Regina cælorum} (AM 175).}
\ApplyHeader{\textbf{16} & Feria VI ł - de ea - \textit{Vir.}}
\ApplyHeader{\textbf{17} & Sabbato  - \textsc{De Beata} - \textit{\textbf{memoria maior}} - \textit{Alb.}}
\ApplyBody{
\item ad Vigilias~: lectio sabbato 3 (in supplemento 206 pro breviario vetere).
\item in MC~: \textit{Beatæ Mariæ Virginis, titulo Salus infirmorum} (CM 44)~; præfatio I de Beata Maria Virgine.
\item I Vesperæ dominicæ sequentis.}
\ApplyHebdoPsalt{- hebdomada II psalterii -}
\ApplyHeader{\textbf{18} & \textbf{\textsc{Dominica XX per annum}} (X post Pentecosten - III augusti) - de ea - \textit{Vir.}}
\ApplyBody{
\item in MC~: præfatio III de dominicis.}
\ApplyLectHeader{Hebdomada XX per Annum~: lectiones de tempore}
\ApplyLectBody{
\item[Dom. C] Ier \textbf{38}, 4-6.8-10 / Hebr \textbf{12}, 1-4 / Lc \textbf{12}, 49-53
\item[Feria II] Iudic \textbf{2}, 11-19 / Mt \textbf{19}, 16-22
\item[Feria III] Iudic \textbf{6}, 11-24a / Mt \textbf{19}, 23-30
\item[Feria IV] Iudic \textbf{9}, 6-15 / Mt \textbf{20}, 1-16a
\item[Feria V] Iudic \textbf{11}, 29-39a / Mt \textbf{22}, 1-14
\item[Feria VI] Rut \textbf{1}, 1.3-6.14b-16.22 / Mt \textbf{22}, 34-40
\item[Sabbato] Rut \textbf{2}, 1-3.8-11~; \textbf{4}, 13-17 / Mt \textbf{23}, 1-12}
\ApplyPrefaceFeries{
\item in feriis~: præfatio communis II, nisi aliter notetur.}
\ApplyHeader{\textbf{19} & Feria II - \textsc{S. Bernardi Tolomæi}, abbatis - \textbf{memoria maior} - \textit{Alb.} (olim die 21 huius).}
\ApplyBody{
\item in Officio~: oratio in variationibus 22.
\item ad Vigilias~: lectio in supplemento 154*.
\item ad Benedictus~: ø \textit{Ex domo} (AM 779).
\item \textit{in ML~: Missa in supplemento OSB.}
\item in MC~: collecta in MP~; Commune pastorum (MR 934)~; præfatio de sanctis virginibus et religiosis.
\item ad Magnificat~: ø \textit{Pastor bonus} (AM 483).}
\ApplyHeader{\textbf{20} & Feria III - \textbf{\textsc{S. Bernardi, abbatis et Ecclesiæ doctoris}} - \textbf{festum} - \textit{Alb.}}
\ApplyBody{
\item ad Vigilias~: hymnus proprius~; lectiones in folio supplementi.
\item ad Laudes~: hymnus proprius et antiphona ad Benedictus propria.
\item \textit{in ML~: Missa in PAL.}
\item in MC~: lectiones propriæ~: Sir \textbf{2}, 1-9 / Lc \textbf{6}, 17.20-26~; præfatio de sanctis virginibus et religiosis.
\item ad Vesperas~: hymnus proprius.}
\ApplyHeader{\textbf{21} & Feria IV ł - \textsc{S. Pii X}, papæ - \textbf{memoria maior} - \textit{Alb.} (olim die 3 septembris).}
\ApplyBody{
\item in Officio~: oratio in supplemento 156 vel in variationibus 23.
\item ad Vigilias pro breviario vetere~: lectio in supplemento 156.
\item ad Benedictus~: ø \textit{Homo quidam} (AM 557).
\item in MC~: præfatio de sanctis pastoribus.
\item ad Magnificat~: ø \textit{Dum esset} (AM 663).}
\ApplyHeader{\textbf{22} & Feria V - \textsc{Beatæ Mariæ Virginis Reginæ} - \textbf{\textit{memoria maior}} - \textit{Alb.}}
\ApplyBody{
\item in Officio~: oratio in supplemento 157.
\item ad Vigilias~: lectio, ¶ et capitulum in supplemento 157*.
\item ad Benedictus~: ø \textit{Beata Mater} (AM 713).
\item \textit{in ML~: olim die 31 maii.}
\item in MC~: præfatio I de Beata Maria Virgine.
\item ad Magnificat~: ø \textit{Sancta Maria} (AM 705).}
\ApplyHeader{\textbf{23} & Feria VI ł - de ea - \textit{Vir.}}
\ApplyHeader{\textbf{24} & Sabbato - \textbf{\textsc{S. Bartholomæi, apostoli}} - \textbf{festum} - \textit{Rub.}}
\ApplyBody{
\item in MC~: lectiones propriæ~: Apoc \textbf{21}, 9b-14 / Io \textbf{1}, 45-51~; præfatio I de Apostolis.
\item I Vesperæ dominicæ sequentis.}
\ApplyHebdoPsalt{- hebdomada I psalterii -}
\ApplyHeader{\textbf{25} & \textbf{\textsc{Dominica XXI per annum}} (XI post Pentecosten - IV augusti) - de ea - \textit{Vir.}}
\ApplyBody{
\item in MC~: præfatio IV de dominicis.}
\ApplyLectHeader{Hebdomada XXI per Annum~: lectiones de tempore}
\ApplyLectBody{
\item[Dom. C] Is \textbf{66}, 18-21 / Hebr \textbf{12}, 5-7.11-13 / Lc \textbf{13}, 22-30
\item[Feria II] 1 Th \textbf{1}, 1-5.8b-10 / Mt \textbf{23}, 13-22
\item[Feria III] 1 Th \textbf{2}, 1-8 / Mt \textbf{23}, 23-26
\item[Feria IV] 1 Th \textbf{2}, 9-13 / Mt \textbf{23}, 27-32
\item[Feria V] 1 Th \textbf{3}, 7-13 / Mt \textbf{24}, 42-51
\item[Feria VI] 1 Th \textbf{4}, 1-8 / Mt \textbf{25}, 1-13
\item[Sabbato] 1 Th \textbf{4}, 9-11 / Mt \textbf{25}, 14-30}
\ApplyPrefaceFeries{
\item in feriis~: præfatio communis III, nisi aliter notetur.}
\ApplyHeader{\textbf{26} & Feria II - de ea - \textit{Vir.}}
\ApplyHeader{\textbf{27} & Feria III - \textsc{S. Monicæ, viduæ} - \textbf{memoria maior} - \textit{Alb.} (olim die 4 maii).}
\ApplyBody{
\item ad Vigilias~: invitatorium \textit{Mirabilem} in supplemento 59 vel in variationibus 35, lectio, ¶ et oratio in supplemento 158*.
\item ad Benedictus~: ø \textit{Absterget Deus} (AM 652).
\item in MC~: Commune sanctorum et sanctarum (MR 967)~; præfatio II de sanctis.
\item ad Magnificat~: ø \textit{Tristitia} (AM 490).}
\ApplyHeader{\textbf{28} & Feria IV ł - \textsc{S. Augustini}, episcopi et Ecclesiæ doctoris - \textbf{memoria maior} - \textit{Alb.}}
\ApplyBody{
\item ad Vigilias pro breviario vetere~: lectio XII cum ¶ suo.
\item ad Benedictus~: ø \textit{Omnes autem} (AM 357).
\item \textit{in ML~: Missa in proprio sanctorum vel in PAL.}
\item in MC~: lectiones propriæ~: 1 Io \textbf{4}, 7-16 / Mt \textbf{23}, 8-12~; præfatio de sanctis pastoribus.}
\ApplyHeader{\textbf{29} & Feria V - \textsc{In Passione S. Ioannis Baptistæ} - \textbf{memoria maior} - \textit{Rub.}}
\ApplyBody{
\item ad Vigilias~: in supplemento 159 pro breviario vetere~; In II nocturno, lectio brevis et ß de Communi unius martyris in psalterio.
\item ad Laudes , Vesperas et Horas minores~: antiphonæ propriæ.
\item \textit{in ML~: præfatio propria.}
\item in MC~: lectiones propriæ~: Ier \textbf{1}, 17-19 / Mc \textbf{6}, 17-29~; præfatio propria.}
\ApplyHeader{\textbf{30} & Feria VI ł - de ea - \textit{Vir.}}
\ApplyHeader{\textbf{31} & Sabbato  - \textsc{De Beata} - \textit{\textbf{memoria maior}} - \textit{Alb.}}
\ApplyBody{
\item ad Vigilias~: lectio sabbato 5 (in supplemento 206 pro breviario vetere).
\item in MC~: \textit{Beatæ Mariæ Virginis, titulo Imago et Mater Ecclesiæ I} (CM 25)~; præfatio I de Beata Maria Virgine.
\item I Vesperæ dominicæ sequentis.}

\ApplyNewMonthTitles{September}
\ApplyHebdoPsalt{- hebdomada II psalterii -}
\ApplyHeader{\textbf{1} & ŧ \textbf{\textsc{Dominica XXII per annum}} (XII post Pentecosten - I septembris) - de ea - \textit{Vir.}}
\ApplyBody{
\item in MC~: præfatio V de dominicis.}
\ApplyLectHeader{Hebdomada XXII per Annum~: lectiones de tempore}
\ApplyLectBody{
\item[Dom. C] Sir \textbf{3}, 19-21.30-31 / Hebr \textbf{12}, 18-19.22-24a / Lc \textbf{14}, 1.7-14
\item[Feria II] 1 Th \textbf{4}, 13-17 / Lc \textbf{4}, 16-30
\item[Feria III] 1 Th \textbf{5}, 1-6.9-11 / Lc \textbf{4}, 31-37
\item[Feria IV] Col \textbf{1}, 1-8 / Lc \textbf{4}, 38-44
\item[Feria V] Col \textbf{1}, 9-14 / Lc \textbf{5}, 1-11
\item[Feria VI] Col \textbf{1}, 15-20 / Lc \textbf{5}, 33-39
\item[Sabbato] Col \textbf{1}, 21-23 / Lc \textbf{6}, 1-5}
\ApplyPrefaceFeries{
\item in feriis~: præfatio communis IV, nisi aliter notetur.}
\ApplyHeader{\textbf{2} & Feria II - de ea - \textit{Vir.}}
\ApplyBody{
\item in MC (\textit{Nigr.})~: Missa defunctorum pro omnibus benefactoribus nostris defunctis (MR 1225)~; lectiones propriæ~: 1 Io \textbf{3}, 14.16-20 / Io \textbf{5}, 24-29~; præfatio IV de defunctis.}
\ApplyHeader{\textbf{3} & Feria III - \textbf{S. Gregorii I, papæ et Ecclesiæ doctoris} - \textbf{festum} - \textit{Alb.} (olim die 12 martii).}
\ApplyBody{
\item ad Vigilias~: omnia in supplemento 160.
\item ad Laudes, Vesperas et Horas minores~: antiphonæ propriæ.
\item in MC~: lectiones propriæ~: 2 Cor \textbf{4}, 1-2.5-7 / Lc \textbf{22}, 24-30~; præfatio de sanctis pastoribus.}
\ApplyHeader{\textbf{4} & Feria IV ł - de ea - \textit{Vir.}}
\ApplyHeader{\textbf{5} & Feria V - de ea - \textit{Vir.}}
\ApplyHeader{\textbf{6} & Feria VI µ £ - de ea - \textit{Vir.}}
\ApplyBody{
\item \textit{in ML (Alb.)~: Missa de sacratissimo Corde Iesu \emph{(Gloria)}.}
\item in MC (Alb.) : Missa votiva de sacratissimo Corde Iesu (MR 492 - GR 660) ; præfatio propria.\item I Vesperæ sollemnitatis sequentis. Responsorium \textit{Adiuvabit eam} et versiculum in AM 1178.}
\ApplyHeader{\textbf{7} & Sabbato § - þ \textbf{\MakeUppercase{S. Reginæ, virginis et martyris}} - \textbf{sollemnitas minor} - \textit{Rub.}}
\ApplyBody{
\item in Officio~: omnia de Communi virginum~; oratio in supplemento 168.
\item ad Vigilias~: hymnus proprius et lectiones II nocturni in folio supplemento~; lectio I nocturni de libro Ecclesiastici \textbf{51}, 1-17 in breviario monastico 172*.
\item \textit{in ML~: \emph{Gloria~; Credo}~; præfatio de sanctis.}
\item in MC~: collecta propria~; Commune virginis martyris (MR 924)~; lectiones propriæ~: Ct \textbf{8}, 6-7 / Ep \textbf{6}, 10-13. 18 / Mt \textbf{25}, 1-13 ~; præfatio de sanctis martyribus.
\item Vesperæ sollemnitatis. Responsorium \textit{Adiuvabit eam} et versiculum in AM 1178.}
\ApplyHebdoPsalt{- hebdomada I psalterii -}
\ApplyHeader{\textbf{8} & \textbf{\textsc{Dominica XXIII per annum}} (XIII post Pentecosten - II septembris) - de ea - \textit{Vir.}}
\ApplyBody{
\item in MC~: præfatio VI de dominicis.}
\ApplyLectHeader{Hebdomada XXIII per Annum~: lectiones de tempore}
\ApplyLectBody{
\item[Dom. C] Sap \textbf{9}, 13-19 / Phm 9b-10.12-17 / Lc \textbf{14}, 25-33
\item[Feria II] Col \textbf{1}, 24–2, 3 / Lc \textbf{6}, 6-11
\item[Feria III] Col \textbf{2}, 6-15 / Lc \textbf{6}, 12-19
\item[Feria IV] Col \textbf{3}, 1-11 / Lc \textbf{6}, 20-26
\item[Feria V] Col \textbf{3}, 12-17 / Lc \textbf{6}, 27-38
\item[Feria VI] 1 Tim \textbf{1}, 1-2.12-14 / Lc \textbf{6}, 39-42
\item[Sabbato] 1 Tim \textbf{1}, 15-17 / Lc \textbf{6}, 43-49}
\ApplyPrefaceFeries{
\item in feriis~: præfatio communis V, nisi aliter notetur.}
\ApplyHeader{\textbf{9} & Feria II - de ea - \textit{Vir.}}
\ApplyHeader{\textbf{10} & Feria III - de ea - \textit{Vir.}}
\ApplyHeader{\textbf{11} & Feria IV ł - de ea - \textit{Vir.}}
\ApplyHeader{\textbf{12} & Feria V - \textsc{Ss. Nominis Beatæ Virginis Mariæ} - \textbf{\textit{memoria maior}} - \textit{Alb.}}
\ApplyBody{
\item ad Vigilias~: lectio et oratio in supplemento 169*
\item ad Benedictus~: ø \textit{Sancta Maria} (AM 705).
\item in MC~: præfatio I de Beata Maria Virgine.}
\ApplyHeader{\textbf{13} & Feria VI ł - \textsc{S. Ioannis Chrysostomi}, episcopi et Ecclesiæ doctoris - \textbf{memoria maior} - \textit{Alb.} (olim die 27 ianuarii).}
\ApplyBody{
\item ad Vigilias~: lectio, ¶ et oratio in supplemento 169.
\item ad Benedictus~: ø \textit{Semen est} (AM 324).
\item in MC~: præfatio de sanctis pastoribus.}
\ApplyHeader{\textbf{14} & Sabbato - \textbf{\textsc{In Exaltatione Sanctæ Crucis}} - \textbf{festum} - \textit{Rub.}}
\ApplyBody{
\item in MC~: lectiones propriæ~: Phil \textbf{2}, 6-11 / Io \textbf{3}, 13-17~; præfatio propria.\item ad Vesperas~: omnia ut in I Vesperis.}
\ApplyHebdoPsalt{- hebdomada II psalterii -}
\ApplyHeader{\textbf{15} & \textbf{\textsc{Dominica XXIV per annum}} (XIV post Pentecosten - III septembris) - de ea - \textit{Vir.}}
\ApplyBody{
\item in MC~: præfatio VII de dominicis.}
\ApplyLectHeader{Hebdomada XXIV per Annum~: lectiones de tempore}
\ApplyLectBody{
\item[Dom. C] Ex \textbf{32}, 7-11.13-14 / 1 Tim \textbf{1}, 12-17 / Lc \textbf{15}, 1-32
\item[Feria II] 1 Tim \textbf{2}, 1-8 / Lc \textbf{7}, 1-10
\item[Feria III] 1 Tim \textbf{3}, 1-13 / Lc \textbf{7}, 11-17
\item[Feria IV] 1 Tim \textbf{3}, 14-16 / Lc \textbf{7}, 31-35
\item[Feria V] 1 Tim \textbf{4}, 12-16 / Lc \textbf{7}, 36-50
\item[Feria VI] 1 Tim \textbf{6}, 2c-12 / Lc \textbf{8}, 1-3
\item[Sabbato] 1 Tim \textbf{6}, 13-16 / Lc \textbf{8}, 4-15}
\ApplyPrefaceFeries{
\item in feriis~: præfatio communis VI, nisi aliter notetur.}
\ApplyHeader{\textbf{16} & Feria II - \textsc{Ss. Cornelii}, papæ et \textsc{Cypriani}, episcopi, martyrum - \textbf{memoria maior} - \textit{Rub.}}
\ApplyBody{
\item in Officio~: oratio in supplemento 172 vel in AM 1040.
\item ad Vigilias~: lectio in supplemento 171.
\item ad Benedictus~: ø \textit{Adstiterunt iusti} in tono Ig (AM 929).
\item in MC~: præfatio de sanctis martyribus.
\item ad Magnificat~: ø \textit{Isti sunt sancti} in tono Id (AM 929).}
\ApplyHeader{\textbf{17} & Feria III - S. Hildegardis, virginis et Ecclesiæ doctoris - \textit{memoria minor} - \textit{Vir.}}
\ApplyBody{
\item ad Benedictus~: ø \textit{Sponsa Christi} in tono III g in variationibus 24.
\item \textit{in ML (Alb.)~: Missa in supplemento OSB.}
\item in MC \textit{(Alb.)}~: collecta in MP~; Commune virginum (MR 948).}
\ApplyHeader{\textbf{18} & Feria IV ł - de ea - \textit{Vir.}}
\ApplyBody{
\item \textit{in ML (Viol.)~: Quatuor Temporum Septembris.}
\item in MC~: Missa \textit{post collectos fructus terræ} (MR 1129 - MS 2135 - GR 654)~; præfatio V de dominicis per annum.}
\ApplyHeader{\textbf{19} & Feria V - S. Sequani, abbatis - memoria minor - \textit{Vir.}}
\ApplyBody{
\item ad Benedictus~: ø \textit{Serve bone} (AM 673)~; oratio in supplemento 172.
\item \textit{in ML (Alb.)~: Missa pro abbate.}
\item in MC \textit{(Alb.)}~: collecta propria~; Commune sanctorum et sanctarum (MR 958).}
\ApplyHeader{\textbf{20} & Feria VI ł - S. Iusti de Bretenières, presbyteri et martyris - memoria minor - \textit{Vir.}}
\ApplyBody{
\item ad Benedictus~: ø \textit{Alias oves} (AM 486)~; oratio in supplemento 172.
\item \textit{in ML (Viol.)~: Quatuor Temporum Septembris.}
\item in MC \textit{(Rub.)}~: collecta propria~; Commune martyrum (MR 917).}
\ApplyHeader{\textbf{21} & Sabbato - \textbf{\textsc{S. Matthæi, apostoli et evangelistæ}} - \textbf{festum} - \textit{Rub.}}
\ApplyBody{
\item in MC~: lectiones propriæ~: Eph \textbf{4}, 1-7.11-13 / Mt \textbf{9}, 9-13~; præfatio II de Apostolis.
\item I Vesperæ dominicæ sequentis.}
\ApplyHebdoPsalt{- hebdomada I psalterii -}
\ApplyHeader{\textbf{22} & \textbf{\textsc{Dominica XXV per annum}} (XV post Pentecosten - IV septembris) - de ea - \textit{Vir.}}
\ApplyBody{
\item in MC~: præfatio VIII de dominicis.}
\ApplyLectHeader{Hebdomada XXV per Annum~: lectiones de tempore}
\ApplyLectBody{
\item[Dom. C] Am \textbf{8}, 4-7 / 1 Tim \textbf{2}, 1-8 / Lc \textbf{16}, 1-13
\item[Feria II] Esd \textbf{1}, 1-6 / Lc \textbf{8}, 16-18
\item[Feria III] Esd \textbf{6}, 7-8.12b 14-20 / Lc \textbf{8}, 19-21
\item[Feria IV] Esd \textbf{9}, 5-9 / Lc \textbf{9}, 1-6
\item[Feria V] Ag \textbf{1}, 1-8 / Lc \textbf{9}, 7-9
\item[Feria VI] Ag \textbf{1}, 15b–2, 9 / Lc \textbf{9}, 18-22
\item[Sabbato] Zac \textbf{2}, 1-5.10-11a / Lc \textbf{9}, 44b-45}
\ApplyPrefaceFeries{
\item in feriis~: præfatio communis I, nisi aliter notetur.}
\ApplyHeader{\textbf{23} & Feria II - S. Pii de Pietrelcina, presbyteri - memoria minor - \textit{Vir.}}
\ApplyBody{
\item ad Benedictus~: ø \textit{Vivo autem} (AM 1128)~; oratio in supplemento 172*.
\item \textit{in ML (Alb.)~: Missa pro confessore non pontifice.}
\item in MC \textit{(Alb.)}~: Commune sanctorum et sanctarum (MR 961).}
\ApplyHeader{\textbf{24} & Feria III - de ea - \textit{Vir.}}
\ApplyHeader{\textbf{25} & Feria IV ł - de ea - \textit{Vir.}}
\ApplyHeader{\textbf{26} & Feria V - Ss. Cosmæ et Damiani, martyrum - \textit{memoria minor} - \textit{Vir.} (olim die 27 huius).}
\ApplyBody{
\item ad Benedictus~: ø \textit{Sanctorum precibus} in tono VIII g (AM 829).
\item \textit{in ML~: Rub.}
\item in MC~: \textit{Rub.}}
\ApplyHeader{\textbf{27} & Feria VI ł - \textsc{S. Vincentii de Paul}, presbyteri - \textbf{memoria maior} - \textit{Alb.} (olim die 19 iulii).}
\ApplyBody{
\item ad Vigilias~: lectio in supplemento 172.
\item ad Benedictus~: ø \textit{Amen dico vobis} in tono I f (AM 829).
\item \textit{in ML~: Missa in proprio sanctorum vel in PAL.}
\item in MC~: præfatio II de sanctis.
\item ad Magnificat~: ø \textit{O viri} in tono I d (AM 973).}
\ApplyHeader{\textbf{28} & Sabbato  - \textsc{De Beata} - \textit{\textbf{memoria maior}} - \textit{Alb.}}
\ApplyBody{
\item ad Vigilias~: lectio sabbato 4 (in supplemento 208 pro breviario vetere).
\item in MC~: \textit{Beatæ Mariæ Virginis, titulo Causa nostræ lætitiæ} (CM 34)~; præfatio I de Beata Maria Virgine.
\item I Vesperæ dominicæ sequentis.}
\ApplyHebdoPsalt{- hebdomada II psalterii -}
\ApplyHeader{\textbf{29} & \textbf{\textsc{Dominica XXVI per annum}} (XVI post Pentecosten - V septembris) - de ea - \textit{Vir.}}
\ApplyBody{
\item in MC~: præfatio I de dominicis.}
\ApplyLectHeader{Hebdomada XXVI per Annum~: lectiones de tempore}
\ApplyLectBody{
\item[Dom. C] Am \textbf{6}, 1a 4-7 / 1 Tim \textbf{6}, 11-16 / Lc \textbf{16}, 19-31 
\item[Feria II] Zac \textbf{8}, 1-8 / Lc \textbf{9}, 46-50
\item[Feria III] Zac \textbf{8}, 20-23 / Lc \textbf{9}, 51-56
\item[Feria IV] Neh \textbf{2}, 1-8 / Lc \textbf{9}, 57-62
\item[Feria V] Neh \textbf{8}, 1-4a.5-6.7b-12 / Lc \textbf{10}, 1-12
\item[Feria VI] Bar \textbf{1}, 15-22 / Lc \textbf{10}, 13-16
\item[Sabbato] Bar \textbf{4}, 5-12.27-29 / Lc \textbf{10}, 17-24}
\ApplyPrefaceFeries{
\item in feriis~: præfatio communis II, nisi aliter notetur.}
\ApplyHeader{\textbf{30} & Feria II - \textsc{S. Hieronymi}, presbyteri et Ecclesiæ doctoris - \textbf{memoria maior} - \textit{Alb.}}
\ApplyBody{
\item ad Vigilias pro breviario vetere~: lectio XII cum ¶ suo.
\item ad Benedictus~: ø \textit{Qui verbum} (AM 325).
\item in MC~: præfatio I de sanctis.}

\ApplyNewMonthTitles{October}
\ApplyHeader{\textbf{1} & Feria III - \textsc{S. Teresiæ a Iesu Infante, virginis et Ecclesiæ doctoris}, patronæ secundariæ Galliæ - \textbf{memoria maior} - \textit{Alb.} (olim die 3 huius).}
\ApplyBody{
\item in Officio~: oratio in supplemento 175 vel in variationibus 24.
\item ad Vigilias~: lectio in supplemento 174.
\item ad Benedictus~: ø \textit{Qui sperant} in variationibus 31.
\item \textit{in ML~: Præfatio de Sanctis.}
\item in MC~: lectiones propriæ~: Is \textbf{66} 10-14c / Mt \textbf{18}, 1-5~; præfatio de sanctis virginibus et religiosis.
\item ad Magnificat~: ø \textit{Virgo gloriosa} (AM 1143).}
\ApplyHeader{\textbf{2} & Feria IV ł - \textsc{Ss. Angelorum Custodum} - \textbf{memoria maior} - \textit{Alb.}}
\ApplyBody{
\item ad Vigilias pro breviario vetere~: in supplemento 175.
\item ad Laudes, Vesperas et Horas minores~: antiphonæ propriæ.\item ad Benedictus~: ø \textit{Omnes sunt} (AM 1066).\item in MC~: cantatur hymnus angelicus \textit{Gloria}~; lectiones propriæ~: Ex \textbf{23}, 20-23a / Mt \textbf{18}, 1-5.10~; præfatio de Angelis.}
\ApplyHeader{\textbf{3} & Feria V - de ea - \textit{Vir.}}
\ApplyHeader{\textbf{4} & Feria VI µ £ - \textsc{S. Francisci Assisiensis} - \textbf{memoria maior} - \textit{Alb.}}
\ApplyBody{
\item ad Vigilias pro breviario vetere~: lectio XII cum ¶ suo.
\item ad Benedictus~: ø \textit{Vos qui reliquistis} (AM 624).
\item in MC~: præfatio de sanctis virginibus et religiosis.
\item ad Magnificat~: ø \textit{Vivo autem} (AM 1128).}
\ApplyHeader{\textbf{5} & Sabbato §  - \textsc{De Beata} - \textit{\textbf{memoria maior}} - \textit{Alb.}}
\ApplyBody{
\item ad Vigilias~: lectio sabbato 1 (in supplemento 208 pro breviario vetere).
\item ad Benedictus~: ø \textit{Beata es} (AM 1074).
\item \textit{in ML~: Immaculati Cordis Beatæ Mariæ Virginis.}
\item in MC~: \textit{Immaculati Cordis Beatæ Mariæ Virginis} (CM 28)~; præfatio I de Beata Maria Virgine.
\item I Vesperæ dominicæ sequentis (hymnus tono hiemali).}
\medskip
\ApplyPrefaceFeries{
\item a dominica I octobris usque ad Adventum~: dicitur hymnus hiemalis ad Vigilias, Laudes et Vesperas.}
\ApplyHebdoPsalt{- hebdomada I psalterii -}
\ApplyHeader{\textbf{6} & ŧ \textbf{\textsc{Dominica XXVII per annum}} (XVII post Pentecosten - I octobris) - de ea - \textit{Vir.}}
\ApplyBody{
\item in MC~: præfatio II de dominicis.}
\ApplyLectHeader{Hebdomada XXVII per Annum~: lectiones de tempore}
\ApplyLectBody{
\item[Dom. C] Hab \textbf{1}, 2-3~; \textbf{2}, 2-4 / 2 Tim \textbf{1}, 6-8.13-14 / Lc \textbf{17}, 5-10
\item[Feria II] Ion \textbf{1}, 1–2, 1.11 / Lc \textbf{10}, 25-37
\item[Feria III] Ion \textbf{3}, 1-10 / Lc \textbf{10}, 38-42
\item[Feria IV] Ion \textbf{4}, 1-11 / Lc \textbf{11}, 1-4
\item[Feria V] Mal \textbf{3}, 13–4, 2a / Lc \textbf{11}, 5-13
\item[Feria VI] Ioel \textbf{1}, 13-15~; \textbf{2}, 1-2 / Lc \textbf{11}, 15-26
\item[Sabbato] Ioel \textbf{3}, 12-21 / Lc \textbf{11}, 27-28}
\ApplyPrefaceFeries{
\item in feriis~: præfatio communis III, nisi aliter notetur.}
\ApplyHeader{\textbf{7} & Feria II - \textsc{Beatæ Mariæ Virginis a Rosario} - \textbf{memoria maior} - \textit{Alb.}}
\ApplyBody{
\item ad Vigilias~: invitatorium proprium~; in supplemento 177 pro breviario vetere.
\item ad Laudes, Vesperas et Horas minores~: antiphonæ propriæ.
\item in MC~: lectiones propriæ~: Act \textbf{1}, 12-14 / Lc \textbf{1}, 26-38~; præfatio I de Beata Maria Virgine.}
\ApplyHeader{\textbf{8} & Feria III - de ea - \textit{Vir.}}
\ApplyHeader{\textbf{9} & Feria IV ł - Ss. Dionysii, episcopi et Sociorum, martyrum. - \textit{memoria minor} - \textit{Vir.}}
\ApplyBody{
\item ad Benedictus~: ø \textit{Sanctorum velut aquilæ} (AM 652)~; oratio in supplemento 179.
\item \textit{in ML~: Rub.}
\item in MC \textit{(Rub.)}~: Commune martyrum (MR 913).}
\ApplyHeader{\textbf{10} & Feria V - de ea - \textit{Vir.}}
\ApplyHeader{\textbf{11} & Feria VI ł - de ea - \textit{Vir.}}
\ApplyHeader{\textbf{12} & Sabbato  - \textsc{De Beata} - \textit{\textbf{memoria maior}} - \textit{Alb.}}
\ApplyBody{
\item ad Vigilias~: lectio sabbato 2 (in supplemento 209 pro breviario vetere).
\item in MC~: \textit{Beatæ Mariæ Virginis, Reginæ pacis} (CM 45)~; præfatio I de Beata Maria Virgine.
\item I Vesperæ dominicæ sequentis.}
\ApplyHebdoPsalt{- hebdomada II psalterii -}
\ApplyHeader{\textbf{13} & \textbf{\textsc{Dominica XXVIII per annum}} (XVIII post Pentecosten - II octobris) - de ea - \textit{Vir.}}
\ApplyBody{
\item in MC~: præfatio III de dominicis.}
\ApplyLectHeader{Hebdomada XXVIII per Annum~: lectiones de tempore}
\ApplyLectBody{
\item[Dom. C] 2 Reg \textbf{5}, 14-17 / 2 Tim \textbf{2}, 8-13 / Lc \textbf{17}, 11-19
\item[Feria II] Rom \textbf{1}, 1-7 / Lc \textbf{11}, 29-32
\item[Feria III] Rom \textbf{1}, 16-25 / Lc \textbf{11}, 37-41
\item[Feria IV] Rom \textbf{2}, 1-11 / Lc \textbf{11}, 42-46
\item[Feria V] Rom \textbf{3}, 21-30a / Lc \textbf{11}, 47-54
\item[Feria VI] Rom \textbf{4}, 1-8 / Lc \textbf{12}, 1-7
\item[Sabbato] Rom \textbf{4}, 13.16-18 / Lc \textbf{12}, 8-12}
\ApplyPrefaceFeries{
\item in feriis~: præfatio communis IV, nisi aliter notetur.}
\ApplyHeader{\textbf{14} & Feria II - de ea - \textit{Vir.}}
\ApplyHeader{\textbf{15} & Feria III - \textsc{S. Teresiæ a Iesu, virginis et Ecclesiæ doctoris} - \textbf{memoria maior} - \textit{Alb.}}
\ApplyBody{
\item ad Laudes~: hymnus proprius~; ad Benedictus~: ø \textit{O beata anima} (AM 1131).
\item \textit{in ML~: Missa in proprio sanctorum vel in PAL.}
\item in MC~: præfatio de sanctis virginibus et religiosis.
\item ad Vesperas~: hymnus proprius~; ad Magnificat~: ø \textit{Sanctissima Christi sponsa} (AM 1128).}
\ApplyHeader{\textbf{16} & Feria IV ł - S. Margaritæ Mariæ Alacoque, virginis - \textit{memoria minor} - \textit{Vir.} (olim die 17 huius).}
\ApplyBody{
\item ad Benedictus~: ø \textit{Dum esset rex} (AM 686)~; oratio in supplemento 179.
\item \textit{in ML~: Alb.}
\item in MC \textit{(Alb.)}~: Commune virginum (MR 948).}
\ApplyHeader{\textbf{17} & Feria V - \textsc{S. Ignatii Antiocheni}, episcopi et martyris - \textbf{memoria maior} - \textit{Rub.} (olim die 1 februarii).}
\ApplyBody{
\item ad Vigilias~: lectio, ¶ et oratio in supplemento 180.
\item in MC~: præfatio de sanctis martyribus.}
\ApplyHeader{\textbf{18} & Feria VI ł - \textbf{\textsc{S. Lucæ, evangelistæ}} - \textbf{festum} - \textit{Rub.}}
\ApplyBody{
\item in MC~: lectiones propriæ~: 2 Tim \textbf{4}, 9-17 / Lc \textbf{10}, 1-9~; præfatio II de Apostolis.}
\ApplyHeader{\textbf{19} & Sabbato  - \textsc{De Beata} - \textit{\textbf{memoria maior}} - \textit{Alb.}}
\ApplyBody{
\item ad Vigilias~: lectio sabbato 3 (in supplemento 209 pro breviario vetere).
\item in MC~: \textit{Beatæ Mariæ Virginis iuxta crucem Domini} (CM 12)~; præfatio I de Beata Maria Virgine.
\item I Vesperæ dominicæ sequentis.}
\ApplyHebdoPsalt{- hebdomada I psalterii -}
\ApplyHeader{\textbf{20} & \textbf{\textsc{Dominica XXIX per annum}} (XIX post Pentecosten - III octobris) - de ea - \textit{Vir.}}
\ApplyBody{
\item in MC~: præfatio IV de dominicis.}
\ApplyLectHeader{Hebdomada XXIX per Annum~: lectiones de tempore}
\ApplyLectBody{
\item[Dom. C] Ex \textbf{17}, 8-13 / 2 Tim \textbf{3}, 14–4, 2  / Lc \textbf{18}, 1-8
\item[Feria II] Rom \textbf{4}, 20-25 / Lc \textbf{12}, 13-21
\item[Feria III] Rom \textbf{5}, 12.15b.17-19.20b-21 / Lc \textbf{12}, 35-38
\item[Feria IV] Rom \textbf{6}, 12-18 / Lc \textbf{12}, 39-48
\item[Feria V] Rom \textbf{6}, 19-23 / Lc \textbf{12}, 49-53
\item[Feria VI] Rom \textbf{7}, 18-25a / Lc \textbf{12}, 54-59
\item[Sabbato] Rom \textbf{8}, 1-11 / Lc \textbf{13}, 1-9}
\ApplyPrefaceFeries{
\item in feriis~: præfatio communis V, nisi aliter notetur.}
\ApplyHeader{\textbf{21} & Feria II - de ea - \textit{Vir.}}
\ApplyHeader{\textbf{22} & Feria III - S. Ioannis Pauli II, papæ - \textit{memoria minor} - \textit{Vir.}}
\ApplyBody{
\item ad Benedictus~: ø \textit{Dum esset} (AM 663)~; oratio in supplemento 181*.
\item \textit{in ML~: Missa \emph{Si diligis me} de Communi Summorum Pontificum, præter orationem.}
\item in MC \textit{(Alb.)}~: collecta propria~; Commune pastorum (MR 927).}
\ApplyHeader{\textbf{23} & Feria IV ł - \textsc{Dominæ Nostræ Sanctæ Spei} - \textbf{\textit{memoria maior}} - \textit{Alb.}}
\ApplyBody{
\item in Officio~: oratio in supplemento 182.
\item ad Vigilias~: lectio in supplemento 181.
\item ad Benedictus~: ø \textit{Beata es Maria} (AM 709).
\item \textit{in ML~: Missa Sanctissimi Nominis Mariæ (vide ad diem 12 septembris) præter orationem.}
\item in MC~: CM 37~; præfatio I de Beata Maria Virgine.}
\ApplyHeader{\textbf{24} & Feria V - S. Antonii Mariæ Claret, episcopi - \textit{memoria minor} - \textit{Vir.}}
\ApplyBody{
\item ad Benedictus~: ø \textit{Sacerdos} (AM 656)~; oratio in supplemento 182.
\item \textit{in ML (Alb.)~: olim die 23 huius.}
\item in MC \textit{(Alb.)}~: Commune pastorum (MR 938).}
\ApplyHeader{\textbf{25} & Feria VI ł - de ea - \textit{Vir.}}
\ApplyHeader{\textbf{26} & Sabbato  - \textsc{De Beata} - \textit{\textbf{memoria maior}} - \textit{Alb.}}
\ApplyBody{
\item ad Vigilias~: lectio sabbato 4 (in supplemento 210 pro breviario vetere).
\item in MC~: \textit{Sanctæ Mariæ, Matris Domini} (CM 19)~; præfatio I de Beata Maria Virgine.
\item I Vesperæ dominicæ sequentis.}
\ApplyHebdoPsalt{- hebdomada II psalterii -}
\ApplyHeader{\textbf{27} & \textbf{\textsc{Dominica XXX per annum}} (XX post Pentecosten - IV octobris) - de ea - \textit{Vir.}}
\ApplyBody{
\item in MC~: præfatio V de dominicis.}
\ApplyLectHeader{Hebdomada XXX per Annum~: lectiones de tempore}
\ApplyLectBody{
\item[Dom. C] Sir \textbf{35}, 15b-17.20-22a / 2 Tim \textbf{4}, 6-8.16-18 / Lc \textbf{18}, 9-14
\item[Feria II] Rom \textbf{8}, 12-17 / Lc \textbf{13}, 10-17
\item[Feria III] Rom \textbf{8}, 18-25 / Lc \textbf{13}, 18-21
\item[Feria IV] Rom \textbf{8}, 26-30 / Lc \textbf{13}, 22-30
\item[Feria V] Rom \textbf{8}, 31b-39 / Lc \textbf{13}, 31-35
\item[Feria VI] Rom \textbf{9}, 1-5 / Lc \textbf{14}, 1-6
\item[Sabbato] Rom \textbf{11}, 1-2a.11-12.25-29 / Lc \textbf{14}, 1.7-11}
\ApplyPrefaceFeries{
\item in feriis~: præfatio communis VI, nisi aliter notetur.}
\ApplyHeader{\textbf{28} & Feria II - \textbf{\textsc{Ss. Simonis et Iudæ, apostolorum}} - \textbf{festum} - \textit{Rub.}}
\ApplyBody{
\item in MC~: lectiones propriæ~: Eph \textbf{2}, 19-22 / Lc \textbf{6}, 12-19~; præfatio I de Apostolis.
\item ad Magnificat~: ø \textit{In regeneratione} (AM 1125).}
\ApplyHeader{\textbf{29} & Feria III - de ea - \textit{Vir.}}
\ApplyHeader{\textbf{30} & Feria IV ł - de ea - \textit{Vir.}}
\ApplyHeader{\textbf{31} & Feria V - de ea - \textit{Vir.}}
\ApplyBody{
\item I Vesperæ sollemnitatis sequentis.}
\medskip
\textit{Plenaria indulgentia, animabus in Purgatorio detentis tantummodo applicabilis, conceditur christifideli qui}\par \textit{1° singulis diebus, a primo usque ad octavum novembris, cœmeterium devote visitaverit et, vel mente tantum, pro defunctis exoraverit~;}\par \textit{2° die Commemorationis omnium fidelium defunctorum (vel, de consensu Ordinarii, die Dominico antecedenti aut subsequenti aut die sollemnitatis Omnium Sanctorum) ecclesiam aut oratorium pie visitaverit ibique recitaverit \emph{Pater} et \emph{Credo}.}\begin{flushright}\textit{(Enchiridion Indulgentiarum, concessio n. 29~: Pro fidelibus defunctis).}\end{flushright}

\ApplyNewMonthTitles{November}
\ApplyHeader{\textbf{1} & Feria VI ł £ - ¬ \textbf{\MakeUppercase{Omnium Sanctorum}} - \textbf{sollemnitas maior} - \textit{Alb.}}
\ApplyBody{
\item ad Laudes et Vesperas~: hymnus in folio separato.
\item \textit{in ML~: præfatio de sanctis.}
\item in MC~: lectiones propriæ~: Apoc \textbf{7}, 2-4.9-14 / 1 Io \textbf{3}, 1-3 / Mt \textbf{5}, 1-12a~; præfatio propria.
\item Vesperæ sollemnitatis~; benedictio Sanctissimi Sacramenti.}
\ApplyHeader{\textbf{2} & Sabbato µ § - \textbf{\textsc{In Commemoratione omnium Fidelium Defunctorum}} - \textit{Nigr.}.}
\ApplyBody{
\item ß \textit{Gloria Patri} dicitur in fine omnium psalmorum et canticorum. In responsoriis omittitur ß \textit{Requiem æternam}.
\item ad Vigilias~: absolute incipitur ab invitatorio (psalmus 94)~; psalmi sumuntur e Nocturnis I et III~; lectiones sumuntur e Nocturnis I et II.
\item \textit{hodie, licet omnibus sacerdotibus tres Missas celebrare, ea tamen lege, ut unam tantum libere applicare et pro ea stipem percipere queant~: tenentur vero, nulla stipe percepta, alteram in suffragium omnium fidelium defunctorum, tertiam ad mentem Summi Pontificis applicare.}
\item \textit{ritus in Missis servandus~: In prima et secunda Missa, si immediate sacerdos aliam Missam sit celebraturus, sumpto divino Sanguine, purificat calicem cum aqua tantum.}
\item in MC \textit{(1a Missa)}~: lectiones propriæ~: Is \textbf{25}, 6a-9 / 1 Cor \textbf{15}, 51-54.57~; sequentia \textit{Dies iræ} (Besnier 53) / Io \textbf{6}, 51-59~; præfatio I de defunctis.\item Vesperæ pro defunctis.\item Completorium pro defunctis~: incipitur a \textit{Confiteor}, post examen conscientiæ~; aspersio de more.}
\ApplyHebdoPsalt{- hebdomada I psalterii -}
\ApplyHeader{\textbf{3} & ŧ \textbf{\textsc{Dominica XXXI per annum}} (XXI post Pentecosten - I novembris) - de ea - \textit{Vir.}}
\ApplyBody{
\item in MC~: præfatio VI de dominicis.}
\ApplyLectHeader{Hebdomada XXXI per Annum~: lectiones de tempore}
\ApplyLectBody{
\item[Dom. C] Sap \textbf{11}, 23–12, 2 / 2 Th \textbf{1}, 11–2, 2 / Lc \textbf{19}, 1-10
\item[Feria II] Rom \textbf{11}, 29-36 / Lc \textbf{14}, 12-14
\item[Feria III] Rom \textbf{12}, 5-16a / Lc \textbf{14}, 15-24
\item[Feria IV] Rom \textbf{13}, 8-10 / Lc \textbf{14}, 25-33
\item[Feria V] Rom \textbf{14}, 7-12 / Lc \textbf{15}, 1-10
\item[Feria VI] Rom \textbf{15}, 14-21 / Lc \textbf{16}, 1-8
\item[Sabbato] Rom \textbf{16}, 3-9.16.22-27 / Lc \textbf{16}, 9-15}
\ApplyPrefaceFeries{
\item in feriis~: præfatio communis I, nisi aliter notetur.
\item ad Vigilias~: in feriis, lectiones SO.}
\ApplyHeader{\textbf{4} & Feria II - \textsc{S. Caroli}, episcopi - \textbf{memoria maior} - \textit{Alb.}}
\ApplyBody{
\item ad Vigilias~: lectio de memoria in supplemento 183.
\item ad Benedictus~: ø \textit{Sacerdos et pontifex} (AM 656).
\item \textit{in ML~: Missa in proprio sanctorum vel in PAL.}
\item in MC~: præfatio de sanctis pastoribus.
\item ad Magnificat~: ø \textit{Euge} (AM 661).}
\ApplyHeader{\textbf{5} & Feria III - de ea - \textit{Vir.}}
\ApplyHeader{\textbf{6} & Feria IV ł - de ea - \textit{Vir.}}
\ApplyHeader{\textbf{7} & Feria V - de ea - \textit{Vir.}}
\ApplyHeader{\textbf{8} & Feria VI ł - \textsc{Sanctæ Elisabeth a Trinitate}, virginis - memoria minor - \textit{Vir.}}
\ApplyBody{
\item ad Benedictus~: ø \textit{O Beata} (AM 1131)~; oratio in supplemento 184.
\item \textit{in ML (Alb.)~: Missa pro virgine tantum.}
\item in MC \textit{(Alb.)}~: collecta propria~; Commune virginum (MR 947).}
\ApplyHeader{\textbf{9} & Sabbato - \textbf{\textsc{In Dedicatione Basilicæ Lateranensis}} - \textbf{festum} - \textit{Alb.}}
\ApplyBody{
\item ad Vigilias~: in nocturno II, lectiones de commune Dedicationis ecclesiæ in II nocturno.
\item \textit{in ML~: præfatio de dedicatione ecclesiæ.}
\item in MC~: lectiones propriæ~: Ez \textbf{47}, 1-2.8-9.12 / Io \textbf{2}, 13-22~; præfatio de dedicatione ecclesiæ II.}
\ApplyHebdoPsalt{- hebdomada II psalterii -}
\ApplyHeader{\textbf{10} & \textbf{\textsc{Dominica XXXII per annum}} (XXII post Pentecosten - II novembris) - de ea - \textit{Vir.}}
\ApplyBody{
\item ad Vigilias~: lectiones Nocturnorum I et II sumuntur in supplemento 52*. Et sic in feriis hebdomadæ II novembris.
\item in MC~: præfatio VII de dominicis.}
\ApplyLectHeader{Hebdomada XXXII per Annum~: lectiones de tempore}
\ApplyLectBody{
\item[Dom. C] 2 Mac \textbf{7}, 1-2.9-14 / 2 Th \textbf{2}, 15–3, 5 / Lc \textbf{20}, 27-38
\item[Feria II] Sap \textbf{1}, 1-7 / Lc \textbf{17}, 1-6 
\item[Feria III] Sap \textbf{2}, 23–3, 9 / Lc \textbf{17}, 7-10
\item[Feria IV] Sap \textbf{6}, 2-12 / Lc \textbf{17}, 11-19
\item[Feria V] Sap \textbf{7}, 22–8, 1 / Lc \textbf{17}, 20-25
\item[Feria VI] Sap \textbf{13}, 1-9 / Lc \textbf{17}, 26-37
\item[Sabbato] Sap \textbf{18}, 14-16~; \textbf{19}, 6-9 / Lc \textbf{18}, 1-8}
\ApplyPrefaceFeries{
\item in feriis~: præfatio communis II, nisi aliter notetur.}
\ApplyHeader{\textbf{11} & Feria II - \textbf{\textsc{S. Martini, episcopi}} - \textbf{festum} - \textit{Alb.}}
\ApplyBody{\item ad Vigilias~: in nocturno I~: lectiones 1 et 2 cum ¶ lectionis 4.\item in MC~: lectiones propriæ~: Is \textbf{61}, 1-3a / Mt \textbf{25}, 31-40~; præfatio de sanctis pastoribus.}
\ApplyHeader{\textbf{12} & Feria III - S. Theodori Studitæ, abbatis - \textit{memoria minor} - \textit{Vir.}}
\ApplyBody{
\item ad Benedictus~: ø \textit{Serve bone} (AM 673)~; oratio in supplemento 185.
\item \textit{in ML (Alb.)~: Missa pro abbate.}
\item in MC \textit{(Alb.)}~: collecta in MP~; Commune sanctorum et sanctarum (MR 958).}
\ApplyAnniv{\textup{†} Cras recurrit anniversarium obitus RP Lini Mariæ \textsc{Delbos}, sacerdotis, qui die 13 novembris 2011, in Abbatia Dominæ Nostræ Mayliliensis, obdormivit in Domino.}

\ApplyHeader{\textbf{13} & Feria IV ł - \textbf{\textsc{S. Benigni, martyris}} - \textbf{festum} - \textit{Rub.}}
\ApplyBody{
\item in Officio~: omnia de Communi unius martyris~; oratio in supplemento 188.
\item ad Vigilias~: invitatorium proprium in supplemento 59~; omnia de Communi unius martyris præter hymnum in supplemento 185, lectiones I et II nocturnorum in supplemento 185*~; antiphonæ et psalmi hebdomadæ I.
\item \textit{in ML~: Missa et præfatio propriæ (olim die 20 novembris).}
\item in MC~: oratio propria~; Commune martyrum (MR 915)~; lectiones propriæ~: 1 Thes \textbf{2}, 2-8 / Mc \textbf{16}, 15-18~; præfatio de sanctis martyribus.}
\ApplyAnniv{\textup{†} Cras recurrit anniversarium obitus Reverendissimi Patris Ioannis \textsc{Prou}, abbatis, qui die 14 novembris 1999, in Abbatia Sancti Petri Solesmensis, obdormivit in Domino.}

\ApplyHeader{\textbf{14} & Feria V - de ea - \textit{Vir.}}
\ApplyHeader{\textbf{15} & Feria VI ł - S. Alberti Magni, episcopi et Ecclesiæ doctoris - \textit{memoria minor} - \textit{Vir.}}
\ApplyBody{
\item ad Benedictus~: ø \textit{Omnis sapientia} (AM 581).
\item \textit{in ML~: Alb.}
\item in MC \textit{(Alb.)}~: Commune doctorum Ecclesiæ (MR 943).}
\ApplyHeader{\textbf{16} & Sabbato - \textsc{S. Gertrudis Magnæ}, virginis - \textbf{memoria maior} - \textit{Alb.} (olim die sequenti).}
\ApplyBody{
\item ad Vigilias pro breviario vetere~: in supplemento 188.
\item ad Laudes, Vesperas et Horas minores~: antiphonæ propriæ.
\item \textit{in ML~: Missa in supplemento OSB.}
\item in MC~: Commune virginum (MR 948)~; lectiones propriæ~: Ep \textbf{3}, 14-19 / Io \textbf{15}, 1-8~; præfatio de sanctis virginibus et religiosis.
\item I Vesperæ dominicæ sequentis.}
\ApplyHebdoPsalt{- hebdomada I psalterii -}
\ApplyHeader{\textbf{17} & \textbf{\textsc{Dominica XXXIII per annum}} (XXIII post Pentecosten - III novembris) - de ea - \textit{Vir.}}
\ApplyBody{
\item in MC~: præfatio VIII de dominicis.}
\ApplyLectHeader{Hebdomada XXXIII per Annum~: lectiones de tempore}
\ApplyLectBody{
\item[Dom. C] Mal \textbf{4}, 1-2a / 2 Th \textbf{3}, 7-12 / Lc \textbf{21}, 5-19
\item[Feria II] 1 Mac \textbf{1}, 11-16.43-45.57-60.65-67 / Lc \textbf{18}, 35-43
\item[Feria III] 2 Mac \textbf{6}, 18-33 / Lc \textbf{19}, 1-10
\item[Feria IV] 2 Mac \textbf{7}, 1.20-31 / Lc \textbf{19}, 11-28
\item[Feria V] 1 Mac \textbf{2}, 15-29 / Lc \textbf{19}, 41-44
\item[Feria VI] 1 Mac \textbf{4}, 36-37.52-59 / Lc \textbf{19}, 45-48
\item[Sabbato] 1 Mac \textbf{6}, 1-13 / Lc \textbf{20}, 27-40}
\ApplyPrefaceFeries{
\item in feriis~: præfatio communis III, nisi aliter notetur.}
\ApplyAnniv{Cras recurrit anniversarium ingressus et installationis nostræ in Flaviniacum (1976).}

\ApplyHeader{\textbf{18} & Feria II - Dedicatio Basilicarum Ss. Petri et Pauli, Apostolorum - \textit{memoria minor }- \textit{Vir.}}
\ApplyBody{\item ad Benedictus~: ø \textit{Petrus apostolus} (AM 959).\item \textit{in ML (Alb.)~: præfatio de dedicatione ecclesiæ.}
\item in MC~: lectiones propriæ~: Act \textbf{28}, 11-16.30-31 / Mt \textbf{14}, 22-33~; præfatio I de Apostolis.}
\ApplyHeader{\textbf{19} & Feria III - de ea - \textit{Vir.}}
\ApplyHeader{\textbf{20} & Feria IV ł - de ea - \textit{Vir.}}
\ApplyHeader{\textbf{21} & Feria V - \textsc{In Præsentatione Beatæ Mariæ Virginis} - \textbf{memoria maior} - \textit{Alb.}}
\ApplyBody{
\item ad Vigilias pro breviario vetere~: in supplemento 190.
\item in MC~: lectiones propriæ~: Zac \textbf{2}, 14-17 / Mt \textbf{12}, 46-50~; præfatio I de Beata Maria Virgine.
\item ad Magnificat~: ø \textit{Beata} (AM 1138).}
\ApplyHeader{\textbf{22} & Feria VI ł - \textsc{S. Cæciliæ}, virginis et martyris - \textbf{memoria maior} - \textit{Rub.}}
\ApplyBody{
\item ad Vigilias pro breviario vetere~: in supplemento 191.
\item ad Laudes, Vesperas et Horas minores~: antiphonæ propriæ.
\item in MC~: omnia in MP~; lectiones propriæ~: Os \textbf{2}, 16b.17b.21-22 / Mt \textbf{25}, 1-13~; præfatio de sanctis martyribus.
\item ad Magnificat~: ø \textit{Est secretum} (AM 1139).}
\ApplyHeader{\textbf{23} & Sabbato - \textsc{S. Clementis I}, papæ et martyris - \textbf{memoria maior} - \textit{Rub.}}
\ApplyBody{
\item ad Laudes, Vesperas et Horas minores~: antiphonæ propriæ.
\item in MC~: Commune martyrum (MR 917)~; lectiones propriæ~: 1~P \textbf{5}, 1-4 / Mt \textbf{16}, 13-19~; præfatio de sanctis martyribus.
\item I Vesperæ sollemnitatis sequentis. Officium invenitur in breviario circa finem mensis octobris et in AM 1088.}
\ApplyHebdoPsalt{- hebdomada II psalterii -}
\ApplyHeader{\textbf{24} &\textbf{\textsc{Dominica XXXIV per annum} - \MakeUppercase{Domini nostri Jesu Christi Universorum Regis}} (XXIV post Pentecosten - IV novembris) - \textbf{sollemnitas minor} - \textit{Alb}.}
\ApplyBody{
\item ad Vigilias~: in nocturno II~: lectiones 5 et 6 cum ¶ lectionis 8~; in nocturno III~: lectiones 11 et 12. Pro breviario vetere~: lectiones II nocturni in supplemento 52.
\item in MC~: præfatio propria.
\item Vesperæ sollemnitatis~; ad benedictionem Sanctissimi Sacramenti recitetur \textit{Actus dedicationis humani generis Iesu Christo Regi}~: indulgentia plenaria.
\ApplyLectHeader{Dominica D.N.I.C. Universorum Regis~: lectiones}
\ApplyLectBody{\item[Anno C] 2 Sam \textbf{5}, 1-3 / Col \textbf{1}, 12-20 / Lc \textbf{23}, 35-43}
\smallskip
\ApplyLectHeader{Hebdomada XXXIV per Annum~: lectiones de tempore}
\ApplyLectBody{
\item[Feria II] Dan \textbf{1}, 1-6.8-20 / Lc \textbf{21}, 1-4
\item[Feria III] Dan \textbf{2}, 31-45 / Lc \textbf{21}, 5-11
\item[Feria IV] Dan \textbf{5}, 1-6.13-14.16-17.23-28 / Lc \textbf{21}, 12-19
\item[Feria V] Dan \textbf{6}, 11-27 / Lc \textbf{21}, 20-28
\item[Feria VI] Dan \textbf{7}, 2-14 / Lc \textbf{21}, 29-33
\item[Sabbato] Dan \textbf{7}, 15-27 / Lc \textbf{21}, 34-36}}
\ApplyPrefaceFeries{
\item in feriis~: præfatio communis IV, nisi aliter notetur.}
\ApplyHeader{\textbf{25} & Feria II - S. Catharinæ Labouré, virginis - memoria minor - \textit{Vir.}}
\ApplyBody{
\item ad Benedictus~: ø \textit{Veni sponsa} (AM 678)~; oratio in supplemento 192.
\item \textit{in ML (Alb.)~: Missa pro virgine tantum.}
\item in MC \textit{(Alb.)}~: Collecta propria~; Commune virginum (MR 950).}
\ApplyHeader{\textbf{26} & Feria III - de ea - \textit{Vir.}}
\ApplyBody{
\item in MC (\textit{Nigr.})~: Missa defunctorum pro omnibus benefactoribus nostris defunctis (MR 1225)~; lectiones propriæ~: Ap \textbf{20}, 11 – \textbf{21}, 1 / Io \textbf{14}, 1-6~; præfatio V de defunctis.}
\ApplyHeader{\textbf{27} & Feria IV ł - de ea - \textit{Vir.}}
\ApplyHeader{\textbf{28} & Feria V - de ea - \textit{Vir.}}
\ApplyHeader{\textbf{29} & Feria VI ł - de ea - \textit{Vir.}}
\ApplyBody{
\item ad Benedictus~: ø \textit{Cum videritis} (AM 617).
\item ad Magnificat~: ø \textit{Amen dico vobis} (AM 618).}
\ApplyAnniv{Cras incipiunt preces novendiales ante sollemnitatem Immaculatæ Conceptionis Beatæ Mariæ Virginis.}

\ApplyHeader{\textbf{30} & Sabbato - \textbf{\textsc{S. Andreæ, apostoli}} - \textbf{festum} - \textit{Rub.}}
\ApplyBody{\item in MC~: lectiones propriæ~: Rom \textbf{10}, 9-18 / Mt \textbf{4}, 18-22~; præfatio II de Apostolis.}
\vspace{1cm}
\ApplyHebdoPsalt{\textbf{Post Nonam explicit}}
\ApplyHebdoPsalt{\textbf{Annus liturgicus 2018-2019}}

\end{document}
